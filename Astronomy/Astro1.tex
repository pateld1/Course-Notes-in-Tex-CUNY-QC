\documentclass[12pt]{article}
\usepackage[letterpaper, portrait, margin=1in]{geometry}
\usepackage{amsmath, amsthm, amssymb, mathrsfs}

\usepackage{fancyhdr}
\pagestyle{fancy}
\fancyhf{}
\lhead{Darshan Patel}
\rhead{Astronomy 1: General Astronomy}
\renewcommand{\footrulewidth}{0.4pt}
\cfoot{\thepage}

\begin{document}

\theoremstyle{definition}
\newtheorem{theorem}{Theorem}[section]
\newtheorem{definition}{Definition}[section]
\newtheorem{example}{Example}[section]

\title{Astronomy 1: General Astronomy}
\author{Darshan Patel}
\date{Winter 2017}
\maketitle

\tableofcontents

\section{Our Place in the Universe}
\subsection{Our Modern View of the Universe}
\begin{definition} Cosmic Address: describes our place in the universe \end{definition}
Earth belongs to the Solar System which belongs to the Milky Way Galaxy which belongs to the Local Group which belongs to the Local Supercluster which belongs to the Universe.
\begin{definition} Solar System: consists of the Sun and all the objects that orbit it: the planets and their moons, and countless smaller objects including rocky asteroids and icy comets \end{definition}
\begin{definition} Milky Way Galaxy: a huge, disk-shaped collection of stars \end{definition} 
\begin{definition} Star: a large, glowing ball of gas that generates a heat and light through nuclear fusion \end{definition}
\begin{definition} Planet: a moderately large object that orbits a star; it shines by reflected light; may be rocky, ice or gaseous in compositions \end{definition} 
\begin{definition} Moon: an object that orbits a planet \end{definition} 
\begin{definition} Asteroid: a relatively small and rocky object that orbits a star; leftovers of star formations \end{definition} 
\begin{definition} Comet: a relatively small and icy object that orbits a star; most comets do not actually have tails \end{definition} 
\begin{definition} Star System: a star and all the material that orbit it, including its planets and moons; one example is the solar system \end{definition} 
\begin{definition} Nebula: an interstellar cloud of gas and/or dust \end{definition} 
\begin{definition} Galaxy: a great island of stars in space, all held together by gravity and orbiting a common center, containing more than 100 billion stars \end{definition} 
\begin{definition} Local Group: a collection of galaxies \end{definition} 
\begin{definition} Galaxy Clusters: groups of galaxies with more than a few dozen members \end{definition} 
\begin{definition} Superclusters: regions in which galaxies and galaxy clusters are most lightly packed; essentially clusters of galaxy clusters; our Local group is located in the outskirts of the Local Supercluster \end{definition} 
\begin{definition} Universe: the sum total of all matter and energy; that is, everything within and between all galaxies \end{definition} 
Cosmic Origins \begin{itemize} 
\item Birth of the Universe: The expansion of the universe began with the hot and dense Big Bang. The universe continues to expand, but on smaller scales gravity has pulled matter together to make galaxies.
\item Galaxies as Cosmic Recycling Plants: The early universe contained only two chemical elements: hydrogen and helium. All other elements were made by stars and recycled from one stellar generation to the next within galaxies like our Milky Way. 
\item Life Cycles of Stars: Many generations of stars have lived and died in the Milky Way \begin{itemize} 
\item Stars are born in clouds of gas and dust; planets may form in surrounding disks 
\item Stars shine with energy released by nuclear fusion, which ultimately manufactures all elements heavier than hydrogen and helium
\item Massive stars explode when they die, scattering the elements they've produced into space \end{itemize} 
\item Earth and Life: By the time our solar system was born, 4.5 billion years ago, about 2\% of the original hydrogen and helium had been converted into heavier elements. We are therefore "star stuff," because we and our planet are made from elements manufactured in stars that lived and died long ago 
\end{itemize} 
Note: While the universe as a whole continues to expand, individual galaxies and galaxy clusters do not expand. 
\begin{definition} Nuclear Fusion: the process in which lightweight atomic nuclei smash together and stick (or fuse) to make heavier nuclei \end{definition} 
Light travels at a finite speed (300,000 km/s). Thus, we see objects as they were in the past. The further away we look at distance, the further back we look in time. 
$$ \begin{tabular}{|c|c|} \hline Destination & Light Travel Time \\ \hline
Moon & 1 second \\ \hline
Sun & 8 minutes \\ \hline
Sirius & 8 years \\ \hline
Andromeda Galaxy & 2.5 million years \\ \hline \end{tabular} $$
\begin{definition} Light Year: the distance light can travel in one year; about 10 trillion kilometers; at great distances, we see objects as they were when the universe were much younger,\end{definition} 
Since the observed expansion of the universe implies that it is about 14 billion years ago and that looking far into space means looking far back in time, it places a limit on the portion of the universe that we can see, even in principle. 
\begin{definition} Observable Universe: the portion of the entire universe that we can potentially observe; for us it is 14 billion light years; looking at something beyond means looking back at a time more that 14 billion years ago which is before the universe existed thus there's nothing to see \end{definition} 
Note: This does not put any limit on the size of the entire universe. 

\subsection{The Scale of the Universe}
\begin{definition} Cosmic Calendar: a scale on which we compress the history of the universe into 1 year \end{definition} 
On a scale of 1 to 10 billion, the Sun is about the size of a grapefruit. The Earth is the size of a tip of a ballpoint pen. about 15 m away. The distances between planets are huge compared to their sizes. In addition, Pluto is just a few minutes' walk from the Sun or Earth, the distance to the nearest stars is equivalent to the distance across the United States. On the same scale as above, the stars are thousands of kilometers away. It would take more than 3000 years to count the stars in the Milky Way Galaxy at a rate of one per second. The Milky Way Galaxy is about 100,000 light years across. \newline
The Milky Way is one of about 100 billion galaxies. 
$$ 10^{11} \text{ stars/galaxy} \times 10^{11} \text{galaxies} = 10^{22} \text{stars} $$
It has as many stars as grains of dry sand on all of Earth's beaches. \\~\\ 
The History of the Universe in 1 Year \begin{itemize} 
\item January 1: The Big Bang 
\item February: The Milky Way forms 
\item September 3: Earth forms 
\item September 22: Early life on Earth 
\item December 17: Cambrian explosion
\item December 26: Rise of the dinosaurs 
\item December 30: Extinction of the dinosaurs \end{itemize} 
If we can imagine the 14 billion year history of the universe compressed into 1 year, a human lifetime lasts only a fraction of a second. The entire history of human civilization falls into just the last half minute. 

\subsection{Spaceship Earth} 
Contrary to our perception, we are not ``sitting still." We are moving with the Earth in several ways, and at surprisingly fast speeds. \newline 
Earth rotates (spins) around its axis once every day. 
Earth revolves (orbits) the Sun once every year \begin{itemize} 
\item at an average distance of 1 AU = 150 million km \item with Earth's axis tilted by an 23.5$^\circ$ (pointing to Polaris) \item and rotates in the same direction it orbits, counterclockwise as viewed from above the North Pole \end{itemize} 
\begin{definition} Astronomical Unit (AU): Earth's average orbital distance, equivalent  to about 150 million kilometers (93 million miles) \end{definition} 
\begin{definition} Ecliptic Plane: a flat plane defined by Earth's orbital path \end{definition} 
Earth's axis is tilted by 23.5$^\circ$ from a line perpendicular to the ecliptic plane. This axis tilt happens to be oriented so that it points almost directly at the star Polaris, or the North Star. The idea of ``tilt" by itself has no meaning in space without a reference to a ecliptic plane. \newline
Note: The Earth orbits the Sun in the same direction that it rotates on its axis: counterclockwise. \newline 
Stars in our local solar neighborhood move in essentially random directions relative to each other while the galaxy's rotation carries them around the galactic center at even higher speed. Our Sun moves randomly relative to the other stars. In the local solar neighborhood at typical relative speeds of more than 70,000 km/hr. The Sun orbits the center of the galaxy every 230 million years. \newline
Hubble discovered that \begin{itemize} 
\item All galaxies outside our Local Group are moving away from us 
\item The more distant the galaxy, the faster it is racing away 
\item Therefore, we live in an expanding universe \end{itemize} 
Basic Motions of Earth in the Universe \begin{itemize} 
\item Earth rotates around its axis $>$ 1,000 km/hr 
\item Earth orbits the Sun $>$ 100,000 km/hr
\item The Solar system moves relatively to nearby stars - 70,000 km/hr
\item The Milky Way Galaxy rotates - 800,000 km/hr 
\item Our galaxy moves relative to others in the Local Group - 300,000 km/hr
\item The universe expands \end{itemize} 

\section{Discovering the Universe for Yourself}
\subsection{Patterns in the Night Sky} 
With the naked eye, we can see more than 2000 stars in the Milky Way. 
\begin{definition} Constellation: a region of the sky with well-defined borders; the familiar pattern of stares merely help us locate these constellations; there are a total of 88 constellations that fill the entire sky \end{definition}
\begin{definition} Celestial Sphere: a sphere in which lies the stars and constellations that surround Earth; is an false but useful illusion that allows us to map the sky as seen from Earth; covered by the 88 official constellations \end{definition} 
\begin{itemize} 
\item North Celestial Pole: the point directly over Earth's North Pole
\item South Celestial Pole: the point directly over Earth's South Pole
\item Celestial Equator: a projection of Earth's equator into space, makes a complete circle around the celestial sphere 
\item Ecliptic: the path the Sun follows as it appears to circle around the celestial sphere once each year. It crosses the celestial equator at a 23.5$^\circ$ angle because that is the tilt of Earth's axis \end{itemize} 
The Milky Way traces our galaxy's disk of stars - the galactic plane - as it appears from our location in the outskirts of the galaxy. The Milky Way in the night sky is our view in all directions into the disk of our galaxy. 
\begin{definition} Local Sky: the sky as seen from wherever you happen to be standing - appears to take the shape of hemisphere or dome \end{definition}
\begin{definition} Horizon: the boundary between Earth and the sky \end{definition}
\begin{definition} Zenith: the point directly above the horizon \end{definition}
\begin{definition} Meridian: an imaginary half-circle stretching from the horizon due south, through the zenith, to the horizon due north \end{definition}
We pinpoint an object in the local sky by stating its altitude above the horizon and direction along the horizon. 
\begin{definition} Angular Size: the angle it appears to span in your field of view \end{definition}
The further away an object is, the smaller its angular size.
\begin{definition} Angular Distance: the angle that appears to separate a pair of objects in the sky \end{definition}
\begin{definition} Arcminute: a division of a degree - 1 degree = 60 arcminutes, 1 arcminute = 60 arcseconds \end{definition}
The Paths of Various Stars Through the Local Sky \begin{itemize} 
\item Stars near the north celestial pole do not rise or set; rather, they remain above the horizon and make daily counterclockwise circles around the north celestial pole - these stars are circumpolar 
\item Stars near the south celestial pole never rise above the horizon at all 
\item All other stars have daily circles that are partly above the horizon and partly below it. Because Earth rotates from west to east (counterclockwise as viewed from above the North Pole), these stars appear to rise in the east and set in the west \end{itemize} 
Earth's west to east rotation makes stars appear to move from east to west through the sky as they circle around the celestial poles. 
\begin{definition} Latitude: measures north-south position on Earth \end{definition} 
\begin{definition} Longitude: measures east-west position on Earth \end{definition} 
Latitude is defined to be $0^\circ$ at the equator, increasing to $90^\circ$ N at the North Pole and $90^\circ$ S at the South Pole. Longitude is defined to be $0^\circ$ along a line passing through Greenwich, England. Although the local sky varies with latitude, it does not vary with longitude. Latitude matters because your position on Earth determines which constellations remain below the horizon. Thus, the constellations you see in the sky depend on your latitude only, but not on your longitude. \newline 
The altitude of the celestial pole in your sky is equal to your latitude. 
\begin{definition} Zodiac: the constellations along the ecliptic \end{definition} 
The constellations visible at a particular time of night change as we orbit the Sun. 
\begin{definition} Solar Day: 24 hours \end{definition}
\begin{definition} Sidereal Day (Earth's rotation period): 23 hours, 56 minutes \end{definition} 

\subsection{The Reason for Seasons} 
Earth's seasons are caused by the tilt of its rotation axis, which is why the seasons are opposite in the two hemispheres. The seasons do not depend on Earth's distance from the Sun, which varies only slightly throughout the year. \\~\\
The Seasons \begin{itemize} 
\item Axis Tilt: Earth's axis points in the same direction throughout the year, which causes changes in Earth's orientation relative to the Sun
\item Northern Summer/Southern Winter: In June, sunlight falls more directly on the Northern Hemisphere, which makes it summer there because solar energy is more concentrated and the Sun follows a longer and higher path through the sky. The Southern Hemisphere receives less direct sunlight, making it winter. 
\item Spring/Fall: Spring and Fall begin when sunlight falls equally on both hemispheres, which happens twice a year: in March, when spring begins in the Northern Hemisphere and fall in the Southern Hemisphere; and in September, when fall begins in the Northern Hemisphere and spring in the Southern Hemisphere.
\item Northern Winter/Southern Summer: In December, sunlight falls less directly on the Northern Hemisphere, which makes it winter because solar energy is less concentrated and the Sun follows a shorter and lower path through the sky. The Southern Hemisphere receives more direct sunlight, making it summer. \end{itemize} 
We use the equinoxes and solstices to mark the progression of the seasons. \newline
Solstices and Equinoxes \begin{itemize} 
\item Summer Solstice: occurs around June 21, is the moment when the Northern Hemisphere is tipped most directly toward the Sun (and the Southern Hemisphere is tipped most directly away from it)
\item Winter Solstice: occurs around December 21, is the moment when the Northern Hemisphere is tipped most directly away from the the Sun (and the Southern Hemisphere is tipped most directly toward it) 
\item Spring Equinox: occurs around March 21, is the moment when the Northern Hemisphere goes from being tipped slightly away from the Sun to being tipped slightly toward the Sun
\item Fall Equinox: occurs around September 22, is the moment when the Northern Hemisphere first starts to be tipped away from the Sun \end{itemize} 
Note: For both equinoxes, the Sun shines equally on both hemispheres. The Sun rises precisely due east and sets precisely due west only on the days of the spring and fall equinoxes. \newline
At very high latitudes, the summer Sun remains above the horizon all day long. Seasonal changes are more extreme at high latitudes. 
\begin{definition} Precession: a gradual wobble that changes the orientation of Earth's axis in space \end{definition} 
The tilt of Earth's axis remains close to 23.5$^\circ$, but the direction the axis points in space changes slowly with the 26,000 year cycle of precession. As a result, the constellations associated with the solstices and equinoxes change over time. \\~\\ 
Precession is caused by gravity's effect on a tilted, rotating object that is not a perfect sphere. The spinning Earth precesses because gravitational tugs from the Sun and Moon try to ``straighten out" our planet's bulging equator, which has the same tilt as the axis; it does not succeed in straightening out the tilt but only causes the axis to precess. 

\subsection{The Moon, Our Constant Companion} 
\begin{definition} Lunar Phases: the appearance and the time at which the Moon rises and sets as it moves through the sky; the phase of the Moon on any given day depends on its position relative to the Sun as it orbits Earth; a consequence of the Moon's 27.3 day orbit around Earth \end{definition} 
Half the Moon is always illuminated by the Sun, but the amount of this illuminated half that we see from Earth depends on the Moon's position in its orbit. Each complete cycle of phases, from one new moon to the next, takes about 29.5 days, which is about 2 days longer than the Moon's actual orbital period because of Earth's motion around the Sun during the time the Moon is orbiting around Earth. \\~\\ 
The Moon's phase affects not only its appearance but also its rise and set times. 
$$ \begin{tabular}{|c|c|c|c|} \hline
Phases of the Moon & Rise & Highest & Set \\ \hline 
New Moon & 6:00 am & noon & 6:00 pm \\ \hline
Waxing Crescent & 9:00 am & 3:00 pm & 9:00 pm \\ \hline
First Quarter & noon & 6:00 pm & midnight \\ \hline
Waxing Gibbous & 3:00 pm & 9:00 pm & 3:00 am \\ \hline
Full Moon & 6:00 pm & midnight & 6:00 am \\ \hline 
Waning Gibbous & 9:00 pm & 3:00 am & 9:00 am \\ \hline 
Third Quarter & midnight & 6:00 am & noon \\ \hline 
Waning Crescent & 3:00 am & 9:00 am & 3:00 pm \\ \hline \end{tabular} $$
Phases from new to full are ``waxing" or increasing; Phases from full to new are ``waning" or decreasing. While waxing, the Moon is visible in afternoon/evening and gets ``fuller" or rises later in the day. While waning, the Moon is visible in the late night/morning and gets ``less" and sets later each day. Half the moon's face is seen at first-quarter and third-quarter phases; these phases mark the times when the Moon is one-quarter or three-quarters of the way through its monthly cycle (taken to begin at new moon). The phases just before and after new moon are called crescent while those just before and after full moon are called gibbous. 
\begin{definition} Synchronous Rotation: the Moon rotates exactly once with each orbit; this is why only one side is visible from Earth \end{definition}
\begin{definition} Eclipses: shadows caused when the Sun, Earth and Moon all fall into a straight line \begin{itemize} 
\item Lunar Eclipse: occurs when Earth lies directly between the Sun and the Moon, so that Earth's shadow falls on the Moon 
\item Solar Eclipse: occurs when the Moon lies directly between the Sun and Earth, so that the Moon's shadow falls on Earth \end{itemize} \end{definition} 
We see a lunar eclipse when Earth's shadow falls on the Moon and a solar eclipse when the Moon blocks our view of the Sun. 
\begin{definition} Node: the point on the orbit at which the Moon crosses the surface \begin{itemize} 
\item A lunar eclipse occurs when a full moon occurs at or very near one of the nodes 
\item A solar eclipse occurs when a new moon occurs at or very near one of the nodes \end{itemize}\end{definition} 
\begin{definition} Umbra: the region where sunlight is completely blocked on the shadow of the Moon or Earth \end{definition} 
\begin{definition} Penumbra: the region where sunlight is only partially blocked on the shadow of the Moon or Earth \end{definition} 
\begin{definition} Total Lunar Eclipse: occurs when the Sun, Earth and Moon are nearly perfectly aligned and the Moon passes through Earth's umbra \end{definition} 
\begin{definition} Partial Lunar Eclipse: occurs when the Sun, Earth and Moon are not perfectly aligned and only part of the full moon passes through the umbra and the rest in the penumbra \end{definition} 
\begin{definition} Penumbral Lunar Eclipse: occurs if the Moon passes only through Earth's penumbra \end{definition} 
\begin{definition} Totality: the time during which the Moon is entirely engulfed in the umbra; typically lasts about an hour \end{definition} 
\begin{definition} Total Solar Eclipse: occurs when the Moon is relatively close to Earth in its orbit and the Moon's umbra touches a small area of Earth's surface \end{definition}
\begin{definition} Partial Solar Eclipse: occurs in the lighter area surrounding the umbra where only part of the Sun is blocked from view \end{definition} 
\begin{definition} Annular Eclipse: occurs when the Moon is relatively far from Earth and the umbra does not reach Earth's surface at all; is a ring of sunlight sounding the moon from a position directly behind the umbra \end{definition} 
A total solar eclipse is visible only when the narrow path that the Moon's umbral shadow makes across Earth's surface. \\~\\
The Moon's orbit is tilted 5$^\circ$ to the ecliptic plane. Thus, we have abut two eclipse seasons each year; a lunar at full moon and a solar at new moon. 
\begin{definition} Saros Cycle: the cycle of about 18 years 11.33 days when eclipses recur \end{definition} 

\subsection{The Ancient Mystery of the Planets} 
Planets Known in Ancient Times \begin{itemize} 
\item Mercury (bottom): difficult to see; always close to Sun in sky 
\item Venus (above Mercury): very bright when visible; morning or evening ``star" 
\item Mars (middle): noticeably red 
\item Jupiter (top): very bright
\item Saturn (above Mars): moderately bright \end{itemize} 
Planets usually move eastward through the constellations but sometimes reverse course. 
\begin{definition} Apparent Retrograde Motion: the motion of westward through the zodiac; exhibited by planets for a period that can last from a few weeks to a few month; we see apparent retrograde motion when we pass by a planet in its orbit \end{definition} 
A planet appears to move backward relative to the stars during the period when Earth passes it in its orbit. 
It is easier for us to explain: this occurs when we ``lap" another planet but it is very difficult to explain if you think that Earth is the center of the universe. 
\begin{definition} Stellar Parallax: slight apparent shifts in stellar positions over the course of the year; undetected by the Greeks and thus abandoned the idea of an Earth-centered universe \end{definition} 
The Greeks knew that stellar parallax should occur if Earth orbits the Sun, but they could not detect it because they also believed that all stars lie on the same celestial sphere. They concluded one of the following is true: \begin{itemize} 
\item Earth orbits the Sun, but the stars are so far away that stellar parallax is not detectable to the naked eye 
\item There is no stellar parallax because Earth remains stationary at the center of the universe \end{itemize} 
Today, we can detect stellar parallax with the aid of telescopes, providing direct proof that Earth really does orbit the Sun. Careful measurements of stellar parallax also provide the most reliable means of measuring distances to nearby stars.

\section{The Science of Astronomy} 
\subsection{The Ancient Roots of Science} 
Scientific thinking is based on everyday observations and trial and error experiments. \\~\\
Ancient people used observations of the sky to keep track of the time and seasons and as an aid in navigation. \begin{itemize} 
\item In the daytime, ancient people could tell time by observing the Sun's path through the sky using sundials or obelisks 
\item Many ancient cultures built structures to help them mark seasons, such as the Stonehenge in southern England \end{itemize} 
Some ancient civilizations used lunar phases as a basis for calendars (such as in the Muslim religion) while others used solar calendars that was based on the Sun. \\~\\
Remarkable ancient achievements included accurate calendars, eclipse prediction, navigational tools and elaborate structures for astronomical observations. Days of the week were named for the Sun, Moon and visible planets. 
\begin{definition} Archaeoastronomy: the study of ancient structures in search of astronomical connections \end{definition} 

\subsection{Ancient Greek Science} 
The Greeks developed models of nature that aimed to explain and predict observed phenomena without resorting to myth or the supernatural. In astronomy, the Greeks constructed conceptual models of the universe in an attempt to explain what they observed in the sky, an effort that quickly led them past simplistic ideas of a flat Earth under a dome-shaped sky to a far more sophisticated view of the cosmos. This idea was taught by Pythagoras as early as about 500 BC. He and his followers envisioned Earth as a sphere floating in the center of the celestial sphere. More than a century later, Aristotle cited observations of Earth's curved shadow on the Moon during lunar eclipse as evidence for a spherical Earth. 
\begin{definition} Geocentric Model (of the Universe): a model that shows Earth being the center of the universe; adapted by Greek philosophers \end{definition} 
\begin{definition} Ptolemaic Model: a model that shows planetary motion while preserving Earth's central position and the idea that heavenly objects move in perfect circles \end{definition} 
The essence of the Ptolemaic model was that each planet moves on a small circle (epicycle) whose center moves around Earth on a larger circle (deferent). To make the model agree well with observations, Ptolemy had to include a number of other complexities such as positioning some of the large circles slightly off-center from Earth and thus was mathematically complex and using it to predict planetary positions required long and tedious calculations. However, it did correctly forecast future planetary positions to within a few degrees of arc. \\~\\ 
Much of Greek knowledge was lost with the destruction of the Library of Alexandria and what survived was preserved by the rise of a new center of intellectual inquiry in Baghdad. Scholars of the new religion of Islam sought knowledge of mathematics and astronomy in hopes of better understanding the wisdom of Allah. During the eighth and ninth centuries AD, scholars working in the Muslim empire translated and saved many ancient Greek works. Around 800 AD, Islamic leader Al-Mamun established the ``House of Wisdom" in Baghdad and there, scholars developed the mathematics of algebra and many new instruments and techniques for astronomical observation. The intellectual center in Baghdad achieved a synthesis of the surviving work of the Ancient Greeks and that of the Indians and Chinese. The accumulated knowledge of the Arabs spread through the Byzantine empire. When the Byzantine capital of Constantinople fell to the Turks in 1453, many English scholars headed west to Europe, carrying with them the knowledge that helped ignite the European Renaissance.

\subsection{The Copernican Revolution} 
\begin{definition} Copernican Revolution: the story of the dramatic changes Polish scientist Nicholas Copernicus and other later scientists made to the Ptolemaic model \end{definition} 
Copernicus's Sun-centered model had the right general ideas but its predictions were not substantially better than those of Ptolemy's Earth-centered model. The problem people had with his model was that although he overturned Earth's central place in the cosmos, he had held fast to the belief that heavenly motion must occur in perfect circles and thus by adding numerous complexities like Ptolemy, the model was no more accurate and no less complex than the Ptolemaic model. \\~\\
Better data of proving either models were provided by the Danish noblemen Tycho Brahe. Tycho's accurate naked-eye observations provided the data needed to improve the Copernican system. He was convinced that the planets must orbit the Sun but since he couldn't detect stellar parallax, he led to conclude that Earth must remain stationary and the Sun orbits Earth while all other planets orbit the Sun. \\~\\
Kepler believed that planetary orbits should be perfect circles so he worked to match circular motions to Tycho's data. After years of effort, he found a set of circular orbits that matched most of Tycho's observations quite well. Some positions differed from Tycho's observations by only about 8 arcminutes. The small discrepancies finally led Kepler to abandon the idea of circular orbits and to find the correct solution to the ancient riddle of planetary motion. Kepler's key discovery was that planetary orbits are not circles but instead are a special type of oval called an eclipse. By using elliptical orbits, Kepler created a Sun-centered model that predicted planetary positions with outstanding accuracy. 
\begin{definition} Kepler's First Law: the orbit of each planet about the Sun is an ellipse with the Sun at one focus \end{definition} 
\begin{definition} Perihelion: the point where the planet is closest to the Sun \end{definition} 
\begin{definition} Aphelion: the point where the planet is farthest from the Sun \end{definition} 
The average of a planet's perihelion and aphelion distances is the length of its semimajor axis, or average distance from the Sun. 
\begin{definition} Kepler's Second Law: as a planet moves around its orbit, it sweeps out equal areas in equal times; a planet travels faster when it is nearer to the Sun and slower when it is farther from the Sun \begin{itemize} 
\item Near perihelion, in any particular amount of time, a planet sweeps out an area that is short but wide
\item Near aphelion, in the same amount of time, a planet sweeps out an area that is long but narrow
\end{itemize} \end{definition} 
\begin{definition} Kepler's Third Law: more distant planets orbit the Sun at slower average speeds, obeying the precise mathematical relationship $p^2 = a^3$ where $p$ is the planet's orbital period in years and $a$ is its average distance from the Sun in astronomical units \end{definition} 
Three Basic Objections that Galilei answered \begin{enumerate} 
\item Aristotle had held that Earth could not be moving because, if it were, objects such as birds, falling stones, and clouds would be left behind as Earth moved along its way
\item The idea of noncircular orbits contradicted Aristotle's claim that the heavens - the realm of the Sun, Moon, planets and stars - must be perfect and unchanging
\item No one had detected the stellar parallax that should occur if Earth orbits the Sun \end{enumerate} 
Galileo defused the first observation with experiments that demonstrated that a moving object remains in motion unless a force acts to stop it. The second observation was shattered after he built a telescope in late 1600. He observed that the Sun had sunspots ("imperfections") and that the Moon has mountains and valleys like the "imperfect" Earth by noticing the shadows cast near the diving line between the light and dark portions of the lunar face. If the heavens were in fact not perfect, then the idea of elliptical orbits were not so objectionable. The third observation was not fully proven but but he provided evidence in his favor by seeing through the telescope that the Milky Way resolved into countless individual stars that helped him argue that the stars were far more numerous and more distant that Tycho had believed. \\~\\
Galileo's observation that four moons orbit Jupiter rather than Earth and that Venus goes through phases in a way that proved that it must orbit the Sun and not Earth helped end the idea of the Earth-centered model. In 1633, the Catholic Church ordered Galileo to recant his claim that Earth orbits the Sun. His book on the subject was removed from the Church's Index of banned books in 1824. Galileo was formally vindicated by the Church in 1992. 

\subsection{The Nature of Science} 
\begin{definition} Hypothesis: tentative explanation or an educated guess \end{definition} 
The Scientific Method: \begin{itemize} 
\item Make observations
\item Ask a question
\item Suggest a hypothesis 
\item Make a prediction
\item Perform a test: experiment or additional observation \begin{itemize} 
\item If test supports hypothesis, make additional predictions and test them
\item If test does not support hypothesis, revise hypothesis or make a new one \end{itemize} \end{itemize} 
The scientific method is a useful idealization of scientific thinking, but science rarely progresses in such an orderly way. \newpage
The Hallmarks of Science \begin{itemize} 
\item Modern science seeks explanations for observed phenomena that rely solely on natural causes 
\item Science progresses through the creation and testing of models of nature that explain the observations as simply as possible
\item A scientific model must make testable predictions about natural phenomena that would force us to revise or abandon the model if the predictions do not agree with observations \end{itemize} 
\begin{definition} Occam's Razor: the idea that scientists should prefer the simpler of two models that agree equally well with observations \end{definition} 
\begin{definition} Paradigm: general patterns of thought of the time \end{definition} 
Individual scientists inevitably carry personal biases into their work but the collective action of many scientists should ultimately make science objective and and provide a means of bring people to agreement. \\~\\
A scientific theory is a simple yet powerful model whose predictions have been borne out by repeated and varied testing. A scientific theory must: explain a wide variety of observations with a few simply principles; be supported by a large, compelling body of evident; not have failed any crucial test of its validity.

\section{Making Sense of the Universe: Understanding Motion, Energy and Gravity} 
\subsection{Describing Motion: Examples from Daily Life} 
\begin{definition} Speed: how fast an object will go in a certain amount of time \end{definition} 
\begin{definition} Velocity: the speed and direction of an object \end{definition} 
\begin{definition} Acceleration: the change in velocity in any way, whether in speed or direction or both \end{definition} 
Note: An object is accelerating if either its speed or direction is changing. 
\begin{definition} Acceleration of Gravity: the acceleration of a falling object, abbreviated by $g$; on Earth, the acceleration of gravity causes falling objects to fall faster by 9.8 meters per second with each passing second \end{definition} 
\begin{definition} Momentum: the product of an object's mass and its velocity \end{definition} 
Note: The only way to change an object's momentum is to apply a net force to it. 
\begin{definition} Net Force: overall force acting on an object represented by the combined effect of all the individual forces put together \end{definition} 
Note: An object must accelerate whenever a net force acts on it. 
\begin{definition} Mass: the amount of matter in an object \end{definition} 
\begin{definition} Weight: the force that a scale measures when an object is on it; depends on both mass of the object and on the forces (including gravity) acting on the mass  \end{definition} 
Note: An object's mass is the same no matter where it is but the weight can vary due to force of gravity. 
\begin{definition} Free-fall: falling without any resistance to slow down; can make an object weightless \end{definition} 
People or objects are weightless whenever they are falling freely and astronauts in orbit are weightless because they are in a constant state of free-fall. 

\subsection{Newton's Laws of Motion} 
Newton showed that the same physical laws that operate on Earth also operate in the heavens. 
\begin{definition} Newton's First Law: an object moves at constant velocity if there is no net force acting on it \end{definition} 
\begin{definition} Newton's Second Law: Force = mass $\times$ acceleration ($F = ma$) \end{definition} 
\begin{definition} Newton's Third Law: For any force, there is always an equal and opposite reaction force \end{definition} 

\subsection{Conservation Laws in Astronomy} 
\begin{definition} Angular Momentum: "circling" momentum $$\text{angular momentum} = m \times v \times r $$ where $m$ is the mass, $v$ is its velocity around the orbit and $r$ is the radius of the orbit \end{definition} 
\begin{definition} Conservation of Angular Momentum: an object's angular momentum cannot change unless it transfers angular momentum to or from another object \end{definition} 
Note: The total momentum of interacting objects cannot change unless an external force is acting on them. \newpage
Two Key Facts \begin{itemize}
\item Earth needs no fuel or push of any kind to keep orbiting the Sun - it will keep orbiting as long as nothing comes along to take angular momentum away 
\item Because Earth's angular momentum at any point in its orbit depends on the product of its speed and orbital radius (distance from the Sun), Earth's orbital speed must be faster when it is nearer to the Sun (and the radius is smaller) and slower when it is farther from the Sun (and the radius if larger) \end{itemize} 
Earth is not exchanging substantial angular momentum with any other object so its rotation rate and orbit must stay about the same. 
\begin{definition} Conservation of Energy: energy can be transferred from one object to another or transformed from one type to another but the total amount of energy is always conserved; energy can change form/type \end{definition} 
\begin{definition} Basic Types of Energy \begin{itemize}
\item Kinetic Energy: energy of motion
\item Radiative Energy: energy carried by light 
\item Potential Energy: stored energy \end{itemize} \end{definition} 
\begin{definition} Thermal Energy: the total kinetic energy of many individual particles; depends on both temperature and density \end{definition} 
\begin{definition} Temperature: measures the average kinetic energy of particles \end{definition} 
At lower temperatures, particles are moving relatively slow; at higher temperature, particles are moving relatively fast. 
\begin{definition} Gravitational Potential Energy: depends on an object's mass and how far it can fall as a result of gravity; increases when it moves higher and decreases when it moves lower; equals $mgh$ where $m$ is the mass of the object, $g$ is the strength of gravity and $h$ is the distance the object falls \end{definition} 
Mass itself is a form of potential energy, as described by Einstein's equation $$E = mc^2$$ where $E$ is the amount of potential energy, $m$ is the mass of the object, and $c$ is the speed of light. A small amount of mass can release a great deal of energy. Concentrated energy can spontaneously turn into particles (for example, in particle acceleration).  \\~\\ 
The energy of any object can be traced back to the origin of the universe in the Big Bang. 

\subsection{The Force of Gravity} 
\begin{definition} Universal Law of Gravitation \begin{itemize}
\item Every mass attracts every other mass through the force of gravity 
\item The strength of the gravitational force attracting any two objects is directly proportional to the product of their masses 
\item The strength of gravity between two objects decreases with the square of the distance between their centers; thus the gravitational force follows an inverse square law \end{itemize} These three statements can be combined into the following: $$F_g = G\frac{M_1M_2}{d^2}$$ where $F_g$ is the force of gravitational attraction, $M_1$ and $M_2$ are the masses of the two objects, $d$ is the distance between their centers and $G$ is the gravitational constant ($G = 6.67 \times 10^{-11}$ m$^3$/(kg $\times$ s$^2$))
\end{definition} 
Newton discovered that Kepler's first two laws apply to all orbiting objects and not just to planets going around the Sun. Second, Newton found that ellipses are not the only possible orbital paths; orbits can be bound (ellipses) or unbound (parabola or hyperbola). Kepler was right when he found that ellipses are the only possible shapes for bound orbits but Newton also discovered unbound orbits. 
\begin{definition} Bound Orbits: orbits in which an object goes around another object over and over again \end{definition} 
\begin{definition} Unbound Orbits: paths that bring an object close to another object just once; found by Newton
\end{definition} 
Third, Newton generalized Kepler's third law to account for masses of all distant objects. If a small object orbits a larger one and you measure the orbiting object's orbital period and average orbital distance, then you can calculate the mass of the larger object. 
\begin{definition} Newton's Version of Kepler's Third Law: $$p^2 = \frac{4\pi^2}{G(M_1 + M_2)}a^3 $$ \end{definition} 
\begin{definition} Orbital Energy: the sum of a planet's kinetic and gravitational potential energy; always constant for a planet due to conservation of energy \end{definition} 
Note: Orbits cannot change spontaneously - an object's orbit can change only if it gains or loses orbital energy. 
\begin{definition} Gravitational Encounter: the exchange of orbital energy between two objects where they pass near enough so that each can feel the effects of the other's gravity \end{definition} 
\begin{definition} Escape Velocity: the velocity at which an object has gained enough orbital energy to escape a planet completely \end{definition} 
The Moon's gravity creates a tidal force that stretches Earth along the Earth-Moon line, causing Earth to bulge both toward and away from the Moon. Earth's rotation carries us through the two bulges each day, giving us two daily high tides and two daily low tides.

\section{Light: The Cosmic Messenger} 
\subsection{Basic Properties of Light and Matter} 
\begin{definition} Spectrum: light appearing when passed through a prism or similar device; appears as a rainbow of color stretched in horizontal rows and can include hundreds of dark lines where pieces of the rainbow is missing from the light \end{definition}
\begin{definition} Light: electromagnetic radiation \end{definition}
\begin{definition} Electromagnetic Spectrum: the complete spectrum of light \end{definition}
Visible light is only a tiny part of the complete spectrum of light. \\~\\
A wave is something that can transmit energy without carrying material along with it. 
\begin{definition} Wavelength: the distance between adjacent peaks \end{definition}
\begin{definition} Frequency: the number of times that any piece of a wave moves up and down each second \end{definition}
Because light can affect both electrically charged particles and magnets, light is an electromagnetic wave. That is why light is called electromagnetic radiation and the spectrum of light is called the electromagnetic spectrum. 
\begin{definition} Speed of Light: exhibited by all light traveling in empty space \end{definition}
Note: Due to the constant speed of light, the longer the wavelength of the light, the lower its frequency and energy. 
\begin{definition} Photons: small packets of energy \end{definition}
Light comes in ``pieces" called photons, each with a precise wavelength, frequency and energy. \\~\\
Visible light has wavelengths ranging from about 400 nm at the blue/violet end of the rainbow to about 700 nm at the red end. 
\begin{definition} Infrared: light with wavelengths longer than red light \end{definition}
\begin{definition} Radio Waves: longest wavelength light \end{definition}
\begin{definition} Microwave: the region near the border between infrared and radio waves, where wavelengths range from micrometers to millimeters \end{definition}
\begin{definition} Ultraviolet: light wavelengths shorter than blue \end{definition}
\begin{definition} X Rays: light with even shorter wavelength \end{definition}
\begin{definition} Gamma Rays: light with the shortest wavelength \end{definition}
\begin{definition} Elements: types of atoms \end{definition}
Protons and neutrons are found in the tiny nucleus at the center of an atom. The rest of the atom's volume contains the electrons that surround the nucleus. Although the nucleus is very small compared to the atom as a whole, it contains most of the atom's mass because protons and neutrons are each about 2000 times as massive as an electron. 
\begin{definition} Electrical Charge: a property that describes how strongly an object will interact with electromagnetic fields \end{definition}
Note: Total electrical energy is always conserved. 
\begin{definition} Atomic Number: the number of protons in an atom \end{definition}
\begin{definition} Atomic Mass Number: the combined number of protons and neutrons in an atom \end{definition}
Note: Atoms of different chemical elements have different number of protons. 
\begin{definition} Isotopes: versions of an element with different numbers of neutrons \end{definition}
Note: Isotopes of a particular chemical element all have the same number of protons but different number of neutrons. 
\begin{definition} Molecules: formed by the combination of atoms \end{definition}
Interaction of Light with Matter \begin{itemize} 
\item Emission: A light bulb emits visible light; the energy of the light comes from electrical potential energy supplied to the bulb
\item Absorption: When you place your hand near an incandescent light bulb, your hand absorbs some of the light and this absorbed energy warms your hand 
\item Transmission: Some forms of matter, such as glass or air, transmit light, which means allowing it to pass through 
\item Reflection/Scattering: Light can bounce off matter, leading to what we call reflection (when the bouncing is all in the same general direction) or scattering (when the bouncing is more random) \end{itemize} 
Materials that transmit light and said to be transparent and materials that absorb light are called opaque. 

\subsection{Learning from Light} 
Spectra of electrophysical objects are usually combinations of three basic types. \\~\\
Types of Spectra \begin{itemize} 
\item Continuous Spectrum: spans a broad range of wavelengths without interruption 
\item Emission Line Spectrum: spans wavelengths that are emitted depending on its composition and temperature; consists of bright emission lines against a black background
\item Absorption Line Spectrum: spans wavelengths that are absorbed; consists of dark absorption lines over the background rainbow \end{itemize} 
\begin{definition} Intensity: amount of light at a wavelength in the spectrum \end{definition}
The intensity is high at wavelengths where there is a lot of light and low where there is little light. 
Note: Electrons in atoms can have only particular amounts of energy, at specific energy levels, and not other energies in between. 
\begin{definition} Energy Level Transitions: changes in the energy level due to a rise from a low energy level to a higher one or a fall from a high energy level to a lower one \end{definition}
\begin{definition} Ions: electrically charged atoms \end{definition}
The photons that produce emission lines are created when electrons fall to lower energy levels. Absorption lines occur when photons causes electrons to rise to higher energy levels. \\~\\
Note: Every kind of atom, ion, and molecule produces a unique spectral ``fingerprint;" for example, if we see spectral lines of hydrogen, helium and carbon in the spectrum of a distant star, we know that all three elements are present in the star. It allows us to determine chemical compositions of objects throughout the universe.
\begin{definition} Thermal Radiation Spectrum: a spectrum of thermal radiation - dependency of temperature on light \end{definition}
Planets, stars, rocks and people emit thermal radiation that depends only on temperature. \\~\\
Laws of Thermal Radiation \begin{enumerate} 
\item Stefan - Boltzmann Law: Each square meter of a hotter object's surface emits more light at all wavelengths 
\item Wien's Law: Hotter objects emit photons with a higher average energy thus shorter average wavelength \end{enumerate} 
Hotter objects emit more total light per unit surface area and emit photons with a higher average energy. 
Due to the Doppler effect, spectral lines shift to shorter wavelengths when an object is moving toward us and to longer wavelengths when an object is moving away from us. It tells us how fast an object is moving toward or away from us. 
\begin{definition} Blueshift: a shift to shorter wavelength when an object is moving away from us \end{definition}
\begin{definition} Redshift: a shift to longer wavelength when an object is moving closer to us \end{definition}
\begin{definition} Rest Wavelengths: wavelengths in stationary clouds \end{definition}
If the hydrogen lines in the spectrum of a distant object appear at longer wavelengths, then they are red-shifted. If the hydrogen lines appear at shorter wavelengths, then they are blue-shifted. The larger the shift, the faster the object is moving.
The Doppler shift tells us only the part of an object's full motion that is directed toward or away from us (the object's radial component of motion). It does not give any information about how fast an object is moving across our line of sight (the object's tangential component of motion). 

\subsection{Collecting Light with Telescopes} 
Properties of a Telescope \begin{itemize} 
\item Light-Collecting Area: tells us how much total light the telescope can collect at one time; usually categorize a telescope's "size" as the diameter of its light-collecting area 
\item Angular Resolution: the smallest angle over which two dots - or two stars - are distinct - the smaller, the better \end{itemize} 
Note: Larger lenses are better because larger light collecting area and better angular resolution.\\~\\
Basic Telescope Design \begin{itemize} 
\item Refracting Telescope: operates like an eye, using transparent glass lenses to collect and focus light
\item Reflecting Telescope: uses a precisely curved primary mirror to gather light which reflects the gathered light to a secondary mirror that lies in front of it which then reflects the light to a focus at a place where the eye or instruments can observe it \end{itemize} 
Note: The world's largest reflecting telescopes have primary mirrors 10 meters or more in diameter. Telescopes specialized to observe different wavelengths of light allow us to learn far more than we could learn from visible light alone. \\~\\
Telescopes are placed in space due to a couple of reasons: the brightness of the daytime sky limits visible-light observations, stars cannot be seen on cloudy nights, light pollution (the scattering of bright city lights), and turbulence (ever-changing motion of air in the atmosphere) which bends light in constantly shifting patterns. Thus, telescopes in space are above the distorting effects of Earth's atmosphere. In addition, our atmosphere prevents most forms of light from reaching the ground at all. Only radio waves, visible light and small parts of the infrared spectrum can be observed from the ground. Thus the most important reason for putting telescopes in space is to allow us to observe light that does not penetrate Earth's surface. 
\begin{definition} Adaptive Optics: can eliminate much of the blurring caused by atmospheric turbulence \end{definition}
\begin{definition} Interferometry: allows two or more individual telescopes to achieve the angular resolution of a much large telescope \end{definition}

\section{Formation of Planetary Systems: Our Solar System and Beyond} 
\subsection{A Brief Tour of the Solar System} 
The Solar System \begin{enumerate} 
\item The Sun \begin{itemize} 
\item Over 99.8\% of solar system's mass 
\item Made mostly of hydrogen and helium gas (plasma)
\item Converts 4 million tons of mass into energy per second \end{itemize}
\item Mercury \begin{itemize} 
\item Made of metal and rock; large iron core 
\item Desolate, cratered; long, tall, steep cliffs 
\item Very hot and very cold: 425 degrees C (day), -170 degrees C (night)
\item Ice found in craters very recently \end{itemize} 
\item Venus \begin{itemize} 
\item Nearly identical in size to Earth; surface hidden by clouds 
\item Hellish conditions due to an extreme greenhouse effect 
\item Even hotter than Mercury: 470 degrees C, day and night
\item Densest atmosphere of the terrestrial planets 
\item Sulfuric acid thick clouds; carbon dioxide atmosphere
\item Very weak magnetic field \end{itemize} 
\item Earth \begin{itemize} 
\item An oasis of life 
\item The only surface liquid water in the solar system 
\item A surprisingly large moon \end{itemize} 
\item Mars \begin{itemize}
\item Looks almost Earth-like but it's not 
\item Giant volcanoes, a huge canyon, polar caps
\item Water flowed in the distant past \end{itemize} 
\item Jupiter \begin{itemize} 
\item Much farther from Sun than inner planets 
\item Mostly hydrogen and helium; no solid surface
\item 300 times more massive than Earth
\item Many moons, rings \end{itemize} 
\item The Galilean Moons: Jupiter's moons \begin{itemize} 
\item Io: active volcanoes all over 
\item Europa: possible subsurface ocean 
\item Ganymede: largest moon in solar system 
\item Callisto: a large, cratered ``ice ball" \end{itemize} 
\item Saturn \begin{itemize} 
\item Giant and gaseous like Jupiter 
\item Spectacular rings 
\item Many moons, including cloudy Titan \end{itemize} 
\item Uranus \begin{itemize} 
\item Smaller than Jupiter and Saturn; much larger than Earth 
\item Made of hydrogen and helium gas and hydrogen components (H$_2$O, NH$_3$, CH$_4$)
\item Extreme axis tilt 
\item Moons and rings \end{itemize} 
\item Neptune \begin{itemize} 
\item Similar to Uranus except for axis tilt 
\item Many moons (including Triton)
\item Blue color because of high methane quantity in its atmosphere
\item Strongest winds in the solar system \end{itemize} 
\item Pluto and other Dwarf Planets \begin{itemize} 
\item Was considered a planet for 76 years 
\item Much smaller than other planets 
\item Icy, comet-like composition
\item Pluto's moon Charon is similar in size to Pluto 
\item Dwarf Planets: Ceres, Pluto, Eris, Haumea and Makemake \end{itemize} \end{enumerate} 

\subsection{Clues to the Formation of our Solar System} 
Features of the Solar System \begin{enumerate} 
\item The Sun, planets and large moons generally orbit and rotate in a very organized way \begin{itemize} 
\item All planetary orbits are nearly circular and lie nearly in the same plane 
\item All planets orbit the Sun in the same direction: counterclockwise as viewed from high above Earth's North Pole 
\item Most planets rotate in the same direction in which they orbit, with fairly small axis tilts; the Sun also rotates in this direction
\item Most of the solar system's large moons exhibit similar properties in their orbits around their planets, such as orbiting in their planet's equatorial plane in the same direction that the planet rotates \end{itemize}
\item The eight planets divide clearly into two groups: the small rocky planets that are close together and close to the Sun, and the large gas-rich planets that are farther apart and farther from the Sun \begin{itemize} 
\item Terrestrial Planets: the four inner planets: Mercury, Venus, Earth, Mars; small, rocky and close to the Sun
\item Jovian Planets: the four large planets: Jupiter, Saturn, Uranus, Neptune; large, gas-rich and far from the Sun \end{itemize} 
\item Between and beyond the planets, vast numbers of asteroids and comets orbit the Sun; some are large enough to qualify as dwarf planets; the locations, orbits and compositions of these asteroids and comets follow distinct patterns \begin{itemize} 
\item Asteroids: rocky bodies that orbit the Sun much like planets but they are much smaller; most asteroids are found within the asteroid belt between the orbits of Mars and Jupiter
\item Comets: small objects that orbit the Sun but make largely of ices mixed with rock; the vast majority of comets orbit the Sun in either the Kuiper belt beyond the orbit of Neptune (about 100,000 and with orbit in the same direction and plane as planets), or in the Oort cloud which is much farther from the Sun (about a trillion and with orbits of random tilts and eccentricities) \end{itemize}
\item The generally ordered solar system also has some notable exceptions. For example, only Earth has a large moon along the inner planets and Uranus is tipped on its side \end{enumerate} 
\begin{definition} Nebular Theory: our solar system formed from the gravitational collapse of a great cloud of interstellar gas, explains the general features of our solar system \end{definition}

\subsection{The Birth of the Solar System} 
\begin{definition} Galactic Recycling: elements that formed planets were made in stars and then recycled through interstellar space; we can see stars forming in other interstellar gas clouds, leading support to the nebular theory \end{definition}
The nebular theory begins with the idea that our solar system was born from a cloud of gas, called the solar nebula, that collapsed under its own gravity. The gas that made up the solar nebula contained hydrogen and helium from the Big Bang and heavier elements produced by stars. \\~\\
As the solar nebula shrank in size, three important processes altered its density, temperature and shape, changing it from a large, diffused (spread-out) cloud to a much smaller spinning dish \begin{enumerate} 
\item Heating: the temperature of the solar nebula increased as it collapsed forming the Sun in the center 
\item Spinning: the solar nebula rotated faster and faster as it shrank in radius; the rotation helped insure that not all the material in the solar nebula collapsed into the center; the greater the angular momentum of a rotating cloud, the more spread out it will be 
\item Flattening: the solar nebula flattened into a disk due to collisions between particles in a spinning cloud; collisions between gas particles also reduce up and down motion \end{enumerate} 
The orderly motions of our solar system today are direct result of the solar system's birth in a spinning, flattened cloud of gas. Observations of disks around other stars support the idea that our own solar system was once a spinning disk of gas. \\~\\


\subsection{The Formation of Planets}
The terrestrial planets formed in the warm, inner regions of the swirling disk and the jovian planets formed in the colder, outer regions. Planet formation began around tiny ``seeds" of solid metal, rock or ice. 
\begin{definition} Frost Line: currently lies between the orbits of Mars and Jupiter; inside this, hydrogen compounds are too hot to form ice; outside this, hydrogen compounds are cold enough to form ice \end{definition}
The solid seeds in the inner solar system were made only of metal and rock but in the outer solar system they included the more abundant ices. \newpage
Formation of Terrestrial Planets \begin{itemize} 
\item Small particles of rock and metal were present inside the frost line
\item Planetesimals of rocks and metal built up as these particles collided 
\item Gravity eventually assembled these planetesimals into terrestrial planets 
\item Tiny solid particles stick to form planetesimals 
\item Gravity draws planetesimals together to form planets 
\item This process of assembly is called accretion: the process of where many smaller objects collect into just a few large ones \end{itemize} 
Formation of Jovian Planets \begin{itemize} 
\item Ice could also form small particles outside the frost line
\item Larger planetesimals and planets were able to form
\item The gravity of these larger planets was able to draw in surrounding hydrogen and helium gases 
\item The gravity of rock and ice in jovian planets draws in hydrogen and helium gases 
\item Moons of jovian planets form in miniature disks \end{itemize} 
The solar wind blew away the leftover gases which ended the era of planet formation. \\~\\
Asteroids and Comets \begin{itemize} 
\item Leftovers from the accretion process 
\item Rocky asteroids inside frost line 
\item Icy comets outside frost line \end{itemize} 
In the late stages of solar system formation, first few hundred million years, leftover planetesimals bombarded other objects in a period called Heavy Bombardment. Water may have come to Earth by way of icy planetesimals from the outer solar system. \\~\\
The captured moons of some planets may be captured planetesimals. Collision of two planets, the size of Venus and Mars, resulted in Earth, and a huge amount of debris, which eventually formed the Moon. Giant impacts might also explain the different rotation aces of some planets. \newpage
The Nebular Theory \begin{enumerate} 
\item Cool and dense cloud of gas - contraction 
\item Spinning disk of gas - condensation of materials 
\item Protostar with two zones - accretion of planetesimals 
\item Protoplanetary disk - clearing the nebula 
\item Solar system \end{enumerate} 
\begin{definition} Radiometric Dating: a reliable method for measuring the age of a rock, relying on careful measurements of the proportions of various atoms and isotopes in the rock \end{definition}
A radioactive isotope has a nucleus prone to spontaneous change, or decay, such as breaking apart or having one of its protons turn into a neutron. Decay rates are characterized by its half-life, which is the length of time it would take for half the nuclei in the collection to decay. 
\begin{definition} Carbon Dating: one of the most widely used methods of figuring out when an object of organic material (like wood or leather) formed; it is based on the half-life of carbon 14, which is unstable \end{definition} 
Age dating of meteorites that are unchanged since they condensed and accreted tells us that the solar system is about 4.5 billion years old. Radiometric dating tells us that the oldest moon rocks are 4.4 billion years old, the oldest meteorites are 4.55 billion years old and that planets probably formed 4.5 billion years ago. 


\section{Earth and the Terrestrial Worlds} 
\subsection{Earth as a Planet}
Earth's Interior Structure \begin{itemize}
\item Core: highest-density material, consisting primarily of metals such as nickel and iron, resides in the central core 
\item Mantle: rocky material of moderate density - mostly minerals that contain silicon, oxygen and other elements - forms the thick mantle that surrounds the core 
\item Curst: lowest - density rock, such as granite and basalt (volcanic rocks), forms the thin crust, essentially representing the world's outer skin \end{itemize} 
\begin{definition} Lithosphere: Earth's outer layer consisting of relatively cool and rigid rock; ``floats" on warmer, softer rock beneath; encompasses the crust and part of the upper mantle on Earth and extends deeper into the mantle on smaller worlds \end{definition} 
\begin{definition} Differentiation: the process of layering where gravity pulls the denser material to the bottom, driving the less dense material to the top; results in layers made of different materials \end{definition} 
Note: Earth and the other terrestrial worlds were once hot enough inside for their interiors to melt, allowing material to settle into layers of differing density. \\~\\
Interior heat is the primary driver of geological activity because the heat supplies the energy needed to move rocks and reshape surfaces. 
\begin{definition} Convection: the process by which hot material expands and rises while cooler material contracts and falls; at the typical rate of mantle convection on Earth - a few centimeters per year - it would take 100 million years fear a piece of rock to be carried from the base of the mantle to the top \end{definition} 
Larger planets retain internal heat longer than smaller ones and this heat drives geological activity. \\~\\
Over time, the heating of interiors was done by accretion and differentiation when planets were young. Today, radioactive decay is the mist important heat source. 
\begin{definition} Conduction: transfers heat from a hot material to a cooler material \end{definition} 
\begin{definition} Magnetic Field: created by moving electrical charges \end{definition} 
Earth's magnetic field is generated by the motions of molten metal in its liquid outer core: electrical conduction, convection and rotation. Internal heat causes liquid metal to rise and fall (convection), while Earth's rotation twists and distorts the convection pattern, resulting in electrons in the molten metal moving within the outer core. 
\begin{definition} Magnetosphere: a kind of protective bubble that surrounds our planet; deflects most of the charged particles from the Sun around our planet \end{definition} 
The relatively few charged particles that make it through the magnetosphere tend to be be channelled toward the poles, where they collide with atoms and molecules in our atmosphere and produce aurora (``Northern Light"). \newpage
Processes that Shape Earth's Surface \begin{itemize} 
\item Impact Cratering: the excavation of bowl-shaped impact craters by asteroids or comets crashing into a planet's surface \begin{itemize} 
\item Most cratering happened soon after the solar system formed 
\item Craters are about 10 times wider than the objects that made them 
\item There are more small craters than larger ones \end{itemize}
\item Volcanism: the eruption of molten rock, or lava, from a planet's interior onto its surface \begin{itemize} 
\item Happens when molten rock (magma) finds a path through the lithosphere to the surface 
\item Molten rock is called lava when it reaches Earth's surface \end{itemize} 
\item Tectonics: the disruption of a planet's surface by internal stresses \begin{itemize} 
\item Convection of the mantle creates stresses in the crust called tectonic forces 
\item Compression forces make mountains while valleys are formed the crust is pulled apart \end{itemize}
\item Erosion: the wearing down or building up of geological features by wind, water, ice and other phenomena of planetary weather \end{itemize} 
Earth's atmosphere and oceans were made from gases released from the interior by volcanic outgassing (release of gas into the atmosphere). Tectonics and volcanism generally occur together because both require internal heat and therefore depend on a planet's size. Erosion can both break down and build up geological features. 
\begin{definition} Erosion: a variety of processes that break down or transport rock through the action of ice, liquid or gas; processes that cause it include glaciers, rivers and wind \end{definition} 
\begin{definition} Ozone: a rare gas that absorbs ultraviolet light \end{definition} 
Ozone absorbs the Sun's dangerous ultraviolet radiation while the X rays are absorbed by atoms and molecules higher up in Earth's atmosphere. 
\begin{definition} Greenhouse Effect: the effect of the atmosphere trapping additional heat; keeps Earth's surface much warmer than it would be otherwise, allowing water to stay liquid over most of the surface \end{definition} 
\begin{definition} Greenhouse Gases: any gas that absorbs infrared; have two different kinds of elements (CO$_2$, H$_2$O, CH$_4$); molecules with one or two atoms of the same element are not greenhouse gases (O$_2$, N$_2$)  \end{definition} 
Due to the greenhouse effect, certain molecules let sunlight through but trap escaping infrared photons. \\~\\
The atmosphere scatters blue light from the Sun, making it appear to come from different directions. Sunsets are red because less of the red light from the Sun is scattered. 

\subsection{The Moon and Mercury: Geologically Dead} 
\begin{definition} Lunar Maria: regions of flooded lunar craters \end{definition}
The Moon's dark, smooth maria were made by floods of molten lava billions of years ago, when the Moon's interior was heated by radioactive decay. \\~\\
Mercury has a mixture of heavily cratered and smooth regions like the Moon. The smooth regions are likely ancient lava flows. The planet appears to have shrunk long ago, leaving behind long steep cliffs. The existence of water near the poles of Mercury in regions that stay constantly in shadows. 

\subsection{Mars: A Victim of Planetary Freeze-Drying} 
The main difference between Mars and Earth is that Mars is smaller. Seasonal winds on Mars can drive huge dust storms. Mars has had extremely active volcanism in the past (as recent as 180 million years) and its surface is dotted with numerous large volcanoes. It has been observed that in the past, there was an abundance of liquid water on Mars. Dried up riverbeds and other signs of erosion show that water flowed on Mars in the past. \\~\\
Climate Changes on Mars \begin{itemize} 
\item Mars has not had widespread surface water for  3 billion years 
\item The greenhouse effect probably kept the surface warmer before that 
\item Somehow Mars lost more of its atmosphere 
\item Magnetic fields may have preserved early Martian atmosphere
\item Solar wind may have stripped atmosphere after field decreased because of interior cooling
\end{itemize}

\subsection{Venus: A Hothouse World} 
Venus was the first planet of attempted exploration and first to have a probe reach its surface and return information as well. \\~\\
Venus shows features of volcanism and tectonics, just as expected for a planet of similar size to Earth. There are impact craters on Venus but fewer than Moon, Mercury and Mars. However, there are many volcanoes on Venus. Fractured and contorted surface indicates tectonic stresses. Most of Earth's major geological features can be attributed to plate tectonics, which gradually remakes Earth's surface. Venus does not appear to have plate tectonics, but its entire surface seems to have been ``repaved" 750 million years ago. Venus's lack of Earth-like plate tectonics poses a scientific mystery, but may have arise because Venus has a thicker and stronger lithosphere than Earth. \\~\\
Venus is very hot due to the vert extreme greenhouse effect it has, called runaway greenhouse effect 
The greenhouse effect on Venus keeps its surface temperature at 470 degrees C. Venus has a very thick carbon dioxide atmosphere with a surface pressure 90 times that of Earth. Reflective clouds contain droplets of sulfuric acid. The upper atmosphere has fast winds that remain unexplained. \\~\\
Thick carbon dioxide atmosphere produces an extremely strong greenhouse effect. Earth escapes this fate because most of its carbon and water are in rocks and oceans.Venus retains carbon dioxide in its atmosphere because it lacks oceans to dissolve the carbon dioxide and lock it away in rocks. 

\subsection{Earth as a Living Planet} 
Unique Features of Earth Important for Life \begin{itemize} 
\item Surface liquid water: Earth's distance from the Sun and moderate greenhouse effect make liquid water possible
\item Atmospheric oxygen: photosynthesis (plant life) is required to make high concentrations of oxygen, which produces the protective layer of ozone gas 
\item Plate Tectonics: an important step in the carbon dioxide cycle \begin{itemize} 
\item Acts like a giant conveyer belt on Earth's lithosphere, continually recycling seafloor crust and building up the continents 
\item Continental motion is caused by mantle material eruption when seafloor spread 
\item Movement of plate motions tell us past and future layout of continents
\item The mechanism by Earth self-regulates its temperature is called the carbon dioxide cycle \begin{itemize} 
\item Atmospheric carbon dioxide dissolves in rainwater, creating a mild acid 
\item The mildly acidic rainfall erodes rocks on Earth's continents and rivers carry the broken-down minerals to the ocean
\item In the oceans, the eroded minerals combine with dissolved carbon dioxide and fall to the ocean floor, making carbonate rocks such as limestone 
\item Over million of years, the conveyor belt of plate tectonics carries the carbonate rocks to subduction zones, where they are carried downward 
\item As they are pushed deeper into the mantle, some of the subducted carbonate rock melts and releases its carbon dioxide, which then outgasses back into the atmosphere through volcanoes \end{itemize} \end{itemize} 
\item Climate Stability: the carbon dioxide cycle acts like a thermostat for Earth's surface \end{itemize} \newpage
Human Activity Affecting the Planet \begin{itemize} 
\item Human-made CFCs in the atmosphere destroy ozone, reducing protection from UV radiation
\item Human activity is driving many other species to extinction
\item Human use of fossil fuels produces greenhouse gases that can cause global warming
\item Earth's average temperature has increased by 0.5 degrees C in the past 50 years
\item The concentration of CO$_2$ is rising rapidly
\item An unchecked rise in greenhouse gases will eventually lead to global warming
\item Antarctic air bubbles indicate the current carbon dioxide concentration is at its highest level in at least 500,000 years where most of the increase has been in the past 50 years
\item Human activity is rapidly increasing the atmospheric concentrations of greenhouse gases, causing the global average temperature to increase \end{itemize} 
A planet is habitable if it is located at an optimal distance from the Sun for liquid water to exist, and it's large enough for geological activity to release and retain water and atmosphere. Earth is habitable because it is large enough to remain geologically active and located at a distance from the Sun where oceans were able to form. 

\section{Jovian Planet Systems} 
\subsection{A Different Kind of Planet} 
Jupiter and Saturn are mostly made of hydrogen and helium gases while Uranus and Neptune are mostly made of hydrogen compounds and some hydrogen, helium and rocks. This makes them very different in composition from terrestrial worlds. The jovian planets nearer to the Sun captured more hydrogen and helium gas, making them larger and leaving them with smaller proportions of hydrogen compounds, rock and metal. The jovian cores are very similar and have a mass of 10 Earths. \\~\\
Difference in Jovian Planet Formation \begin{itemize} 
\item Timing: The planets that forms earliest captures the most hydrogen and helium gas; captures ceases after the first solar wind blows the leftover gas away 
\item Location: The planets that forms in a denser part of the nebula forms its core first \end{itemize} \newpage
Uranus and Neptune are denser than Saturn because they have less hydrogen and helium, but not Jupiter. Adding mass to a jovian planet compresses the underlying gas layers. Greater compression is why Jupiter is not much larger than Saturn, even though it is three times more massive. Jovian planets with even more mass can be smaller than Jupiter. All the jovian planets have cloudy skies but clouds of different kinds form at different altitudes in each planet's atmosphere. The rapid rotation of the jovian planets helps drive strong winds, creating their banded appearances and sometimes give rise to huge storms. \\~\\
Interiors of Jovian Planets \begin{itemize} 
\item No solid surface 
\item Layers under high pressure and temperatures 
\item Cores are made of hydrogen compounds, metals and rocks 
\item Layers are different for each planet \end{itemize} 
Inside Jupiter, high pressure causes the phase of hydrogen to change with depth. The core is thought to be made of rock, metals and hydrogen compounds and is about the same size as Earth but 10 times more massive. \\~\\ 
Models suggest that cores of jovian planets have similar compositions. Lower pressure inside Uranus and Neptune mean no metallic hydrogen. \\~\\
Jupiter's strong magnetic field gives it a huge magnetosphere. Hydrogen compounds in Jupiter form clouds. Different cloud layers correspond to freezing points of different hydrogen compounds. \\~\\
Colors \begin{itemize} 
\item Jupiter: ammonium sulfide clouds reflect red/brown, ammonia (the highest, coldest layer), reflects white 
\item Saturn: layers are similar to Jupiter but are deeper in and farther from the Sun  more subdued 
\item Uranus and Neptune: methane gas absorbs red light but absorbs blue light; blue lights reflect off methane clouds, making those planets look blue \end{itemize} 
\begin{definition} Great Red Spot: a storm on Jupiter that is twice as wide as Earth and has existed for the past three centuries \end{definition} 
All the jovian planets have strong winds and storms. 

\subsection{A Wealth of Worlds: Satellites of Ice and Rock} 
A few jovian moons rival the smallest planets in size and geological interest, while vast numbers of smaller moons are captured asteroids and comets. \\~\\
Size of Moons \begin{itemize} 
\item Small moons ($<$ 300 km): no geological activity 
\item Medium-sized moons (300 - 1500 km): geological activity in the past 
\item Large moons ($>$ 1500 km): ongoing geological activity \end{itemize} 
Characteristics of medium-sized and large moons: enough self-gravity to be spherical, have substantial amounts of ice, formed in orbits around jovian planets, circular orbits is in the same direction as planet rotation \newline
Characteristics of small moons: far more numerous than medium and large moons, not enough gravity to be spherical and thus are "potato-shaped" \\~\\ 
Io is the most volcanically active body in the solar system due to tidal heating. 
\begin{definition} Tidal Heating: arises from effects of tidal forces exerted by Jupiter \end{definition} 
Io is squished and stretched as it orbits Jupiter and thus causes volcanic eruptions. 
\begin{definition} Orbital Resonance: a series of small pushes that makes their orbits elliptical due to small gravitational tugs that repeat at each alignment \end{definition} 
Orbital resonances among the Galilean moons make Io's orbit slightly elliptical, leading to the tidal heating that makes Io the most volcanically active place in the solar system. Another effect of tidal stresses is the formation of cracks in Europa's surface ice. Its interior is also warmed by tidal heating which have created a deep ocean of liquid water beneath Europa's icy crust. Jupiter's other two large moons also have some intriguing features. Ganymede is the largest moon in the solar system and has clear evidence of geological activity. Callisto is a ``classic" cratered ice ball. It has no tidal heating and no orbital resonance whereas Ganymede has tidal heating. Tidal heating is weak on Ganymede and absent on Callisto, yet both moons show some evidence of subsurface oceans. \\~\\
Titan is one of Saturn's moons. It is the only moon in the solar system to have a thick atmosphere, leading to methane rain and surprisingly erosional geology. It consists mostly of nitrogen with some argon, methane and ethane. Titan's surface has liquid methane, ``rocks" made of ice. The medium moons of Saturn almost all show evidence of past volcanism and/or tectonics. Enceladus is the smallest moon in the solar system known to be geologically active due to fountains of ice particles and water vapor on the surface. \\~\\
For the medium moons of Uranus, there are varying amounts of geological activity occurring. The moon Miranda has large tectonic features and few craters which may have implications of tidal heating in the past. \\~\\
Triton is one of Neptune's moon. It is similar to Pluto but larger. There is evidence for past geological activity. It orbits Neptune ``backwards." \\~\\
It is clear by now that jovian moons are more geologically active than small rocky planets. For terrestrial planets, internal heat, primarily from radioactive decay, can cause volcanic and tectonic activity. Only large planets retain enough internal heat to stay geologically active today. As per jovian moons, tidal heating can cause tremendous geological activity on moons with elliptical orbits around massive planets. Even without tidal heating, icy materials can melt and deform at lower temperature than rock, increasing the likelihood of geological activity. Together, these effects explain why icy moons are much more like to have ongoing geological activity than rocky terrestrial worlds of the same size. 

\subsection{Jovian Planet Rings} 
Saturn's rings are made up of numerous, tiny individual particles. They orbit over Earth's equator and are very thin. Ring particles cannot last for billions of years, so the rings we see today must be made of particles created recently. New ring particles are released by impacts on small moons within the rings. 
\begin{definition} Gap Moon: creates ripples as its gravity nudges particles that orbit faster than the moon (inside the gap) or slower (outside) \end{definition} 
All four jovian planets have ring systems. Others have ring particles that are smaller and darker than Saturn's. The rings are formed from dust created in impacts on moons orbiting these planets. Rings aren't leftovers from planet formation because the particles are too small to have survived this long. Thus there must be a continuous replacement of tiny particles which is probably impacts with the jovian moons. Jovian planets all have rings because they possess many small moons close-in. Impacts on these moons are random. Saturn's incredible rings may be an ``accident" of our time.

\section{Asteroids, Comets and Dwarf Planets} 
\subsection{Asteroids and Meteorites} 
Asteroids are rocky leftovers of planet formation. The largest asteroid is Ceres, having a diameter of about 1000 km. There are 150,000 listed asteroids and probably over a million whose diameter is grater than 1 km. Small asteroids are more common than large asteroids. \\~\\
Note: All the asteroids in the solar system wouldn't up to even a single terrestrial planet. Some large asteroids have their own moons. \\~\\
Most asteroids orbit in a between Mars and Jupiter, called the Asteroid Belt. 
\begin{definition} Trojan Asteroids: asteroids that follow Jupiter's orbit \end{definition} 
\begin{definition} Near-Earth Asteroids: asteroids whose orbit cross Earth's orbit \end{definition} 
Asteroids in orbital resonance with Jupiter experience periodical nudges. Eventually those nudges move asteroids out of resonance orbits, leaving gaps in the belt. Rocky planetesimals survived in the asteroid belt between Mars and Jupiter because they did not accrete into a planet. Jupiter's gravity, through the influence of orbital resonances, stirred up asteroid orbits and thereby prevented their accretion into a planet. Asteroids leave trails in long-exposure images because of their orbital motion around the Sun. 
\begin{definition} Meteor: a flash of light caused by a particle entering our atmosphere at high speed, not the particle itself \end{definition}
\begin{definition} Meteorite: a rock from space that survives the plunge through the atmosphere; unusually bright meteors called fireballs \end{definition}
Types of Meteorites \begin{itemize} 
\item Primitive meteorites: unchanged in composition since they first formed 4.6 billion years ago
\item Processed meteorites: younger, have experienced processes such as volcanism or differentiation \end{itemize} 
Most meteorites are pieces of asteroids and they teach us much about the early history of our solar system. \\~\\
Meteorites from Mars and the Moon do appear from time to time. Its composition differs from the asteroid fragments. It is a very cheap and slow way to acquire moon rocks and Mars rocks.

\subsection{Comets}
Comets are formed beyond the frost line and are icy counterparts to asteroids. The nucleus of a comet is like a ``dirty snowball." Most comets do not have tails and remain perpetually frozen in the outer solar system. Only comets that enter the inner solar system grow tails. \\~\\
The nucleus of the comet is the source of material for a comet's tail. When a comet nears the Sun, its ices can sublimate into gas and carry off dust, creating a coma and long tails. 
\begin{definition} Coma: the atmosphere that comes from the heated nucleus \end{definition}
\begin{definition} Plasma Tail: the gas escaping from the coma, pushed by solar wind \end{definition}
\begin{definition} Dust Tail: made of dust-size particles escaping from the coma; not affected by the solar wind and so are pushed away from the Sun by the photons \end{definition}
A comet ejects small particles that follow it around in its orbit and cause meteor showers when Earth crosses the comet's orbit. \newpage
There are two major reservoirs of comets in the outer solar system. \begin{itemize} 
\item Oort cloud: comets on random orbits extending to about 50,000 AU 
\item Kuiper belt: comets on orderly orbits at 30-100 AU in disk of solar system \end{itemize} 
Kuiper belt comets orbit in the region in which they formed, just beyond Neptune's orbit. The more distant Oort cloud contains comets that once orbited among the jovian planets. 

\subsection{Pluto: Lone Dog No More}
Pluto's composition and orbit indicate that it is essentially a large comet of the Kuiper belt. Its orbit is tilted and significantly elliptical. In fact, Neptune orbits three times during the time Pluto orbits twice. A collision is prevented by resonance. Pluto is much smaller than the eight major planets. It is not a gas giant like the planets nearby. It has an icy component like a comet. \\~\\ 
Eris, discovered in 2005, is an iceball that is about 5\% larger than Pluto, making it the largest known member of the Kuiper belt. It also has its own moon, Dysnomia. In terms of composition, Pluto, Eris and all the other large objects of the Kuiper belt are essentially large comets. These large icy objects have orbits similar to the smaller objects in the Kuiper belt that became short period comets. \\~\\
In 2006, the IAU decided to call Pluto and objects like it ``dwarf planets." Pluto's largest moon, Charon, is nearly as large as Pluto itself. Pluto is very cold (40 K). It has a thin nitrogen atmosphere that refreezes onto the surface as Pluto's orbit takes it farther from the Sun. \\~\\
Pluto and other larger Kuiper belt objects are smaller, icier and more distant than any of the planets. They can have moons, atmospheres and possibly geological activity. However most have been discovered after 2000 and thus little is known about them. 

\subsection{Cosmic Collisions: Small Bodies versus the Planets} 
Comet SL9 caused a string of violent impacts on Jupiter in 1994, reminding us that catastrophic collisions still happen. Tidal forces tore it apart during a previous encounter with Jupiter. Crater chains on Callisto probably came from another comet that tidal forces tore to pieces. 
\begin{definition} Mass Extinction: the rapid extinction of a large fraction of all living species \end{definition} 
The most recent mass extinction was 65 million years ago, ending the reign of dinosaurs. \\~\\
Iridium is very rare in Earth surface rocks but is often found in meteorites. Luis and Walter Alvarez found a worldwide layer containing iridium, laid down 65 million years ago, probably by a meteorite impact. Below this layer lies dinosaur fossils. \newpage
Consequences of an Impact \begin{enumerate} 
\item A meteorite 10 km in size would send large amounts of debris into the atmosphere
\item Debris would reduce the amount of sunlight reaching Earth's surface 
\item The resulting climate change may have caused mass extinction \end{enumerate} 
Geologists found a large subsurface crater about 65 million years old in Mexico. The size of the crater suggests impacting object was about 10 km in diameter. The impact of such a large object would have ejected debris high into Earth's atmosphere. \\~\\
Facts about Impacts \begin{itemize} 
\item Asteroids and comets have hit Earth
\item A major impact is only a matter of time: not IF but WHEN
\item Major impacts are very rare \begin{itemize}
\item Extinction level events - millions of years
\item Major damage - tens to hundreds of years \end{itemize} \end{itemize} 
Small impacts happen almost daily. Impacts large enough to cause mass extinctions are many millions of years apart. Nearly every asteroid or comet that has ever struck Earth was in some sense sent our way by the influence of the jovian planets. The gravity of a jovian planet (especially Jupiter) can redirect a comet. 


\section{Our Star}
\subsection{A Closer Look at the Sun}
The Sun is not on fire nor contracting; it is powered by nuclear energy. 
\begin{definition} Gravitational Equilibrium: gravity pulling in balances pressure pushing out \end{definition} 
\begin{definition} Energy Balance: thermal energy released by fusion in core balances radiative energy lost from surface \end{definition} 
Gravitational contraction released the energy that made the Sun's core hot enough for fusion. Everywhere inside the Sun, the outward push of pressure balances the inward pull of gravity. Contraction stopped when fusion started replacing the energy radiated into space. 
\begin{definition} Solar Wind: the stream of charged particles that are continually blown outward in all directions from the surface of the Sun \end{definition} \newpage
\begin{definition} Corona: the outermost layer of the solar atmosphere, about 1 million K \end{definition} 
\begin{definition} Chromosphere: middle layer of the solar atmosphere, about $10^4$ to $10^5$ K \end{definition}
\begin{definition} Photosphere: the visible surface of the Sun, about 6000 K \end{definition} 
Note: The Sun's upper atmosphere is much hotter than the visible surface, or photosphere, but its density is much lower. 
\begin{definition} Convection Zone: the zone where energy generated in the solar core travels upward, transported by the rising of hot gas and falling of cold gas called convection \end{definition} 
\begin{definition} Radiation Zone: the zone where energy moves outward primarily in the form of photons of light \end{definition} 
\begin{definition} Core: the zone where energy is generated by nuclear fusion \end{definition} 
Note: Inside the Sun, temperature rises with depth, reaching 15 million K in the core. 

\subsection{Nuclear Fusion in the Sun}
\begin{definition} Fission: big nucleus splits into smaller pieces; done in nuclear power plants \end{definition} 
\begin{definition} Fusion: small nuclei stick together to form a big one; done in the Sun, stars \end{definition}
Positively charged nuclei fuse together if they pass close enough for the strong force to overpower electromagnetic repulsion. The Sun releases energy by fusing four hydrogen nuclei into one helium nucleus. High temperatures enable nuclear fusion to happen in the core. 
\begin{definition} Proton-Proton Chain: the sequence of steps that occurs in the Sun for the formation of helium from hydrogen \begin{enumerate} 
\item Step 1: Two protons fuse to make a deuterium nucleus (1 proton and 1 neutron). This step occurs twice in the overall reaction. 
\item Step 2: The deuterium nucleus and a proton fuse to make a nucleus of helium-3 (2 protons, 1 neutron). This step occurs twice in the overall reaction. Gamma rays are released. 
\item Step 3: Two helium-3 nuclei fuse to form helium-4 (2 protons, 2 neutrons), releasing two excess protons in the process. Gamma rays and subatomic particles known as neutrinos and positrons carry off the energy released in the reaction. \end{enumerate} 
Summary: IN: 4 protons, OUT: $^4$He nucleus, 2 gamma rays, 2 positrons, 2 neutrinos; total mass is 0.7\% lower \end{definition}
Gravitational equilibrium and energy balance together act as a thermostat to keep the Sun's core temperature and fusion rate steady. A slight rise in core temperature leads to a large increase in the fusion rate that raises the core pressure causing the core to expand and cool down. A slight drop in core temperature leads to a large decrease in the fusion rate that lowers the core pressure causing the core to contract and heat up. \\~\\
Randomly bouncing photons carry energy through the deepest layers of the Sun and convection carries energy through the upper layers to the surface. We learn about the inside of the Sun by making mathematical models, observing solar vibrations and observing solar neutrinos. Patterns of vibration on the surface tell us about what the Sun is like inside. Neutrinos created during fusion fly directly through the Sun. Observations of these solar neutrinos can tell us what's happening in the core. Neutrinos provide a direct way to measure nuclear fusion in the Sun and recent results indicate the fusion occurs as our models predict. Early searches for solar neutrinos failed to find the predicted number. More recent observations find the right number of neutrinos, but some have changed form.

\subsection{The Sun-Earth Connection}
Solar activity can be caused by \begin{itemize} 
\item Sunspots \item Solar flares \item Solar prominences \end{itemize} All these phenomena are related to magnetic fields. 
\begin{definition} Sunspots: regions of cooler temperature than the other parts of the Sun's surface (4000 K); regions with strong magnetic fields \end{definition} 
Charged particles spiral along magnetic field lines. Loops of bright gas often connect sunspot pairs. 
\begin{definition} Solar Flares: storms that send bursts of X rays and fast moving charged particles shooting into space, can be caused by the energy released when magnetic field lines snap \end{definition} 
\begin{definition} Solar Prominences: giant loops of trapped gas in the Sun's chromosphere and corona and can erupt high above the Sun's surface \end{definition} 
The corona appears bright in X-ray photos in places where magnetic fields trap hot gas. 
\begin{definition} Coronal Holes: regions of the corona that barely show up in X-ray images; nearly devoid of hot coronal gas; magnetic field lines project out into space like broken rubber bands, allowing particles spiraling along them to escape the Sun altogether \end{definition}
\begin{definition} Coronal Mass Ejections: sends bursts of energetic charged particles out through the solar system \end{definition} 
Note: Particles ejected from the Sun during periods of high activity can hamper radio communications, disrupt power delivery and damage orbiting satellites. 
\begin{definition} Sunspot Cycle: a cycle in which the average number of sunspots on the Sun gradually rises and falls; the average number of sunspots on the Sun rises and falls in an approximately 11 year cycle \end{definition} 
The sunspot cycle has something to do with the winding and twisting of the Sun's magnetic field. The Sun rotates more quickly at its equator than it does near its poles. Because gas circles the Sun faster at the equator, it drags the Sun's north-south magnetic field lines into a more twisted configuration. The field lines become more and more twisted with time and sunspots form when the twisted lines loop above the Sun's surface. 

\section{Surveying the Stars} 
\subsection{Properties of Stars} 
\begin{definition} Apparent Brightness: the amount of power (energy per second) reaching us per unit area \end{definition}
\begin{definition} Luminosity: the total amount of power that a star emits into space \end{definition}
Note: A star's apparent brightness in the sky depends on both its true light output, or luminosity, and its distance from us. \\~\\
Luminosity passing through each sphere is the same. Dividing luminosity by area gets the brightness. 
\begin{definition} Inverse Square Law for Light: the apparent brightness of a star or any other light source obeys an inverse square law with distance $$\text{apparent brightness } = \frac{\text{luminosity}}{4\pi \times (\text{distance})^2} $$ \end{definition}
The most direct way to measure a star's distance is with stellar parallax, the small annual shifts in a star's apparent position caused by Earth's motion around the Sun. Astronomers measure stellar parallax by comparing observations of a nearby star made 6 months apart. The parallax angle depends on distance. The nearby star appears to shift against the background of more distant stars because it observed from two opposite points of Earth's orbit. Parallax was the first reliable technique for measuring distances to stars and it remains the only technique that tells us stellar distances without any assumptions about the nature of stars. To relate parallax angle, $p$, and distance, $d$, $$d = \frac{3.26}{p} $$ where $p$ is in arcseconds and $d$ is in light years.  \newpage
Studies of the Luminosities of Many Stars Shown That \begin{itemize} 
\item Stars come in a wide range of luminosities, with our Sun somewhere in the middle; the dimmest stars have luminosities $\frac{1}{10,000}$ times that of the Sun while the brightest stars are about 1 million times as luminous as the Sun 
\item Dim stars are far more common than bright stars \end{itemize} 
\begin{definition} Apparent Magnitude: another term for apparent brightness \end{definition}
\begin{definition} Absolute Magnitude: another term for luminosity \end{definition}
Every object emits thermal radiation with a spectrum that depends on its temperature. An object of fixed size grows more luminous as its temperature rises. \\~\\
Properties of Thermal Radiation \begin{itemize} 
\item Hotter objects emit more light per unit area at all frequencies
\item Hotter objects emit more photons with a higher average energy \end{itemize} 
The temperature of stars can range from 3000 K to 50,000 K while the Sun's temperature is 5,800 K. \\~\\
A star's temperature can also be revealed by the level of ionization. Absorption lines in a star's spectrum tells the star's ionization level. 
\begin{definition} Spectral Type: determined from the spectral lines present in a star's spectrum; from hottest to coldest: O, B, A, F, G, K, M  \end{definition}
\begin{definition} Binary Star Systems: systems in which two stars continually orbit one another \end{definition} 
Types of Binary Star Systems \begin{itemize} 
\item Visual Binary: a pair of stars that we can see distinctly as the stars orbit each other 
\item Eclipsing Binary: a pair of stars that orbit in the plane of our line of sight; measure periodic eclipses
\item Spectroscopic Binary: a pair of stars that can be observed by its Doppler shifts in its spectral lines \end{itemize} 
Note: About half of all stars are in binary star systems. \\~\\
The masses of stars in binary systems can be determined if both their orbital period and the separation between them is known. $$p^2 = \frac{4\pi^2}{G(M_1 + M_2)}a^3 $$ where $p$ is the period, $a$ is the average separation, $M_1$ is the mass of the first star and $M_2$ is the mass of the second star. Thus to measure mass, only two of the following is needed: (1) orbital period ($p$), (2) orbital separation ($a$ or $r$ = radius) or (3) orbital velocity ($v$). Note: For circular orbits, $v = \frac{2\pi r}{p} $. \\~\\ 
The mass of stars can vary from 0.08 times the mass of the Sun to at least 100 times the mass of the Sun. 

\subsection{Patterns Among Stars} 
\begin{definition} Hertzsprung-Russell (H-R) Diagrams: plots the luminosities and surface temperature of stars; depicts temperature, color, spectral type, luminosity and radius \end{definition}
Patterns in the H-R Diagram \begin{itemize} 
\item Most stars fall somewhere along the main sequence, the prominent streak running from the upper left to the lower right on the H-R diagram
\item The stars in the upper right are called supergiants because they are very large in addition to being very bright
\item Just below the supergiants are the giants, which are somewhat smaller in radius and lower in luminosity (but still much larger and bright than main-sequence stars of the same spectral type)
\item The stars near the lower left, the white dwarfs, are small in radius and appear white in color because of their high temperatures \end{itemize} 
\begin{definition} Luminosity Class: a class system that describes the region of the H-R diagram in which the star falls in; is more related to a star's size than to its luminosity \end{definition}
A star's full classification includes spectral lines (line identities) and luminosity class (line shapes, related to the size of the star): \begin{itemize} 
\item Class I - supergiant 
\item Class II - brightgiant 
\item Class III - giant
\item Class IV - subgiant 
\item Class V - main sequence \end{itemize} 
Main sequence stars are fusing hydrogen into helium in their cores, like the Sun. Luminous main-sequence stars are hot (blue). Less luminous ones are cooler (yellow or red). Mass measurements of main-sequence stars show that the hot, blue stars are much more massive than the cool, red ones. The core temperature of a higher mass star needs to be higher in order to balance gravity. A higher core temperature boosts the fusion rate, leading to greater luminosity. 
\begin{definition} Main Sequence Lifetime: the lifetime of a main sequence star; a star is born with a limited supply of core hydrogen and therefore can remain as a hydrogen-fusing main sequence star for only a limited time \end{definition}
More massive stars live much shorter lives because they fuse hydrogen at a much greater rate. Giants and supergiants are stars that are nearing the end of their lives. The Sun's life expectancy is 10 billion years. The life expectancy of a star that is 10 times more massive than the Sun is $\frac{10}{10^4} \times 1,000,000,000$, or 10 million years, because it has 10 times as much fuel but it uses it $10^4$ times as fast. Thus for a star that is 0.01 times more massive than the Sun, its life expectancy is $\frac{0.1}{0.01} \times 1,000,000,000$, or 100 billion years. \\~\\
To summarize, high mass stars: high luminosity, short-lived, large radius, blue; low mass stars: low luminosity, long-lived, small radius, red. \\~\\
Stars that have finished fusing hydrogen into helium in their cores are no longer on the main sequence. 
\begin{definition} Giants and Supergiants: stars that are becoming larger and redder after exhausting their core hydrogen \end{definition}
\begin{definition} White Dwarfs: stars that are white and small after fusion has ceased \end{definition}

\subsection{Star Clusters} 
\begin{definition} Star Clusters: groups of stars \begin{itemize} 
\item All the stars in a cluster lie at about the same distance from Earth
\item All the stars in a cluster formed at about the same time (within a few million years of one another) \end{itemize} \end{definition}
Types of Star Clusters \begin{itemize} 
\item Open Clusters: a few thousand loosely packed stars 
\item Globular Clusters: up to a million or more stars in a dense ball bound together by gravity \end{itemize}
Massive blue stars die first, followed by white, yellow, orange and red stars. 
\begin{definition} Pleiades: the most famous open cluster \end{definition}
\begin{definition} Main Sequence Turnoff: the precise point on the H-R diagram at which the Pleiades' main sequence diverges from the standard main sequence \end{definition}
Note: The age of the cluster is equal to the lifetime of stars at its main sequence turnoff point. The main sequence turnoff point tells us its age. 
Note: Pleiades now has no stars with a life expectancy less than around 100 million years. \\~\\
Detailed modeling of the oldest globular clusters reveal that they are about 13 billion years old. 

\section{Star Stuff}
\subsection{Star Birth} 
Stars are born in cold, dense clouds of gas whose pressure cannot resist gravitational contraction. Gravity within a contracting gas cloud becomes stronger as the gas becomes denser. 
\begin{definition} Interstellar Medium: the gas between the stars \end{definition}
\begin{definition} Molecular Cloud: star-forming cloud \end{definition}
Gravity can create stars only if it can overcome the force of thermal pressure in a cloud. Gravity within a contracting gas cloud becomes stronger as the gas becomes denser. 
A typical molecular cloud (T around 30 K, n about 300 particles per cm$^3$) must contain at least a few hundred solar masses for gravity to overcome pressure. The cloud can prevent a pressure buildup by converting thermal energy into infrared and radio photons that escape the cloud. The random motions of different sections of the cloud cause it to become lumpy. Each lump of the cloud in which gravity can overcome pressure can go on to become a star. A large cloud can make a whole cluster of stars. 
\begin{definition} Protostar: the dense, clump of gas, center of the cloud fragment that while become a new star \end{definition}
As stars begin to form, dust grains that absorb visible light heat up and emit infrared light. Long wavelength infrared light is brightest from regions where many stars are currently forming. The cloud heats up as gravity causes it to contract due to conservation of energy. Contraction can continue if thermal energy is radiated away. As gravity forces a cloud to become smaller, it begins to spin faster and faster, due to conservation of angular momentum. Gas settles into a spinning disk because spin hampers collapse perpendicular to the spin axis. Rotation of a contracting cloud speeds up. Collisions between particles in the cloud causes it to flatten into a disk and also reduce up and down motions. The spinning cloud flattens as it shrinks. Rotation also causes jets of matter to shoot out along the rotation axis. Jets are observed coming from the center of disks around protostars. \\~\\
A protostar contracts and heats until the core temperature is sufficient enough for hydrogen fusion. Contraction ends when energy released by hydrogen fusion balances energy radiated from the surface and this is when the protostar becomes a main sequence star. \\~\\
Summary of Star Birth \begin{enumerate} 
\item Gravity causes gas cloud to shrink and fragment 
\item Core of shrinking cloud heats up
\item When core gets hot enough, fusion begins and stops the shrinking
\item New star achieves long-lasting state of balance \end{enumerate} 
Note: Neighboring protostars sometimes end up orbiting each other in binary star systems. \\~\\
A cluster of many stars can form out of a single cloud. Very massive stars are rare while low mass stars are common. Photons exert a slight amount of pressure when they struck matter. Very massive stars are so luminous that the collective pressure of photons drives their matter into space. Models of stars suggest that radiation pressure limits how massive a star can be without blowing itself apart. Observations have not found stars more massive than 300 times the mass of the Sun.
\begin{definition} Thermal Pressure: ordinary gas pressure; closely linked to temperature; the main form of pressure in most stars \end{definition}
Fusion will not begin in a contracting cloud if some sort of force stops contraction before the core temperature rises above $10^7$ K. Thermal pressure cannot stop contraction because the star is constantly losing thermal energy from its surface through radiation. 
\begin{definition} Degeneracy Pressure: a type of pressure that does not depend on temperature; depends instead on the laws of quantum mechanics that also give rise to distinct energy levels in atoms; particles cannot be in same state in same place \end{definition}
Degeneracy pressure halts the contraction of objects with 0.08 times the mass of the Sun before the core temperature becomes hot enough for fusion. 
\begin{definition} Brown Dwarfs: starlike objects not massive enough to start fusion; emits infrared light because of heat left over from contraction; luminosity gradually declines with time as it loses thermal energy \end{definition}
Note: Infrared observations can reveal recently formed brown dwarfs because they are still relatively warm and luminous. Stars more massive than 300 times the mass of the Sun would blow apart. Stars less massive that 0.08 times the mass of the Sun cannot sustain fusion. 

\subsection{Life as a Low-Mass Star}
\begin{definition} Low Mass Star: a star with mass of less than two solar mass \end{definition}
A star on the main sequence continues to remain there until it cannot fuse hydrogen into helium in its core. \newpage
Life Stages of a Low-Mass Star: \begin{enumerate} 
\item Protostar 
\item Main Sequence Star 
\item Red Giant 
\item Helium Burning Star 
\item Double-shell Burning Red Giant 
\item Planetary Nebula 
\item White Dwarf \end{enumerate} 
\begin{definition} Red Giant: the phase of the Sun when it has grown in size and luminosity over a period of about a billion years or 10\% of its main sequence lifetime \end{definition}
Observations of star clusters show that star becomes larger, redder and more luminous after its time on the main sequence is over. As the core contracts, hydrogen becomes fusing into helium in a shell around the core. Luminosity increases because the core thermostat is broken; the increasing fusion rate in the shell does not stop the core from contracting. Helium fusion does not start right away because it requires higher temperatures than hydrogen fusion. The fusion of two helium nuclei does not work, so helium fusion must come three three helium nuclei to form carbon. 
\begin{definition} Helium Flash: the moment when helium fusion rate spikes dramatically due to rising temperature; stops core shrinkage and the Sun will become smaller and less luminous than it was as a red giant \end{definition}
The thermostat is broken in a low mass red giant because degeneracy pressure supports the core. The core temperature rises rapidly when helium fusion begins. It skyrockets until thermal pressure takes over and expand the core again. Helium core-fusion stars neither shrink nor grow because the thermostat is temporarily fixed. Models show that a red giant should shrink and become less luminous after helium fusion begins in the core. \\~\\
Note: Helium core-fusion stars are found in a horizontal branch on the H-R diagram. \\~\\
After core helium-fusion stops, helium fuses into carbon in a shell around the carbon core and hydrogen fuses into helium in a shell around the helium layer. This double shell-fusion stage never reaches equilibrium because the fusion rate periodically spikes upward in a series of thermal pulses. With each spike, convection dredges carbon up from the core and transports it to the surface. 
\begin{definition} Planetary Nebula: the Sun's outer layer of hydrogen and helium that gets ejected into space after double shell-fusion ends\end{definition}
The core left behind becomes a white dwarf. Fusion progresses no further in a low mass star because the core temperature never grows hot enough for fusion of heavier elements. Degeneracy pressure supports the white dwarf against gravity. \\~\\
Life Track of a Sun-like Star \begin{enumerate} 
\item Sun becomes a subgiant: inert helium, hydrogen fusion
\item Subgiant becomes a red giant
\item Red giant becomes a helium core-fusion star: helium fusion, hydrogen fusion
\item Helium core-fusion star becomes a double shell-fusion red giant: inert carbon, helium fusion, hydrogen fusion
\item Double shell-fusion red giant becomes a planetary nebula and thus a white dwarf \end{enumerate} 

\subsection{Life as a High-Mass Star} 
Life Stages of a High-Mass Star \begin{enumerate}
\item Protostar
\item Blue Main Sequence Star
\item Red Supergiant 
\item Helium-burning Supergiant
\item Multiple-shell Burning Supergiant
\item Supernova
\item Neutron Star or Black Hole \end{enumerate}

High mass main sequence stars fuse hydrogen to helium at a higher rate using carbon, nitrogen and oxygen as catalysts. A greater core temperature enables hydrogen nuclei to overcome greater repulsion. Late life stages of high mass stars are similar to those of low mass stars: hydrogen core fusion, hydrogen shell fusion, helium core fusion. 
\begin{definition} CNO Cycle: a chain of reactions in which hydrogen fusion in high mass stars form helium, carbon, nitrogen and oxygen \end{definition}
Note: The CNO cycle can change C into N into O. \\~\\
High core temperatures allow helium to fuse with heavier elements. 
\begin{definition} Helium Capture Reactions: reactions in which a helium nucleus fuses into some other nucleus; helium capture builds C into O, Ne, Mg, and so forth \end{definition}
Core temperatures in stars of mass eight times the mass of the Sun allow fusion of elements as heavy as iron. Fusion in stars makes Si, S, Ca and Fe (iron). Advanced nuclear burning proceeds in a series of nested shells. Iron is a dead end for for fusion because nuclear reactions involving iron do not release energy.(iron has the lowest mass per nuclear particle). \\~\\
Iron builds up in the core until degeneracy pressure can no longer resist gravity. The core then suddenly collapses, creating a supernova explosion. Core degeneracy pressure goes away because electrons combine with protons, making neutrons and neutrinos. 
\begin{definition} Neutron Star: the ball of neutrons left behind \end{definition}
Energy and neutrons released in a supernova explosion enables elements heavier than iron to form. 
\begin{definition} Supernova Remnant: energy released by the collapse of the core driving outer layers into space \end{definition}

\subsection{Summary of Stellar Lives}
A star's mass determines its entire life story because it determines its core temperature. High mass stars have short lives, eventually becoming hot enough to make iron and end in supernova explosions. Low mass stars have long lives, never becoming hot enough to fuse carbon nuclei, and end as white dwarfs. \\~\\
Life Stages of Low Mass Star \begin{enumerate} 
\item Main Sequence: hydrogen fuses to helium in core
\item Red Giant: hydrogen fuses to helium in shell around helium core 
\item Helium Core Fusion: helium fuses to carbon in core while hydrogen fuses to helium in shell
\item Double Shell Fusion: hydrogen and helium both fuse in shells
\item Planetary Nebula: leaves white dwarf behind \end{enumerate} 
Reasons for Life Stages \begin{itemize} 
\item Core shrinks and heats until it's hot enough for fusion
\item Nuclei with larger charge require higher temperature for fusion
\item Core thermostat is broken while core is not hot enough for fusion (shell burning)
\item Core fusion can't happen if degeneracy pressure keeps core from shrinking \end{itemize} \newpage
Life Stages of High Mass Star \begin{enumerate} 
\item Main Sequence: hydrogen fuses to helium in core
\item Red Supergiant: hydrogen fuses to helium in shell around helium core 
\item Helium Core Fusion: helium fuses to carbon in core while hydrogen fuses to helium in shell
\item Multiple Shell Fusion: many elements fuse in shells
\item Supernova: leaves neutron star behind \end{enumerate} 
\begin{definition} Mass Exchange: occurs when the giant grows so large that its tidally distorted outer layers succumb to the gravitational attraction of the smaller companion star; the companion then begins to gain mass at the expense of the giant \end{definition}
Stars in close binary systems can exchange mass with each other, altering their life histories. 

\section{The Bizarre Stellar Graveyard}
\subsection{White Dwarfs}
\begin{definition} White Dwarf: the corpse of a low mass star, supported against the crush of gravity by electron degeneracy pressure \end{definition}
White dwarfs cool off and grow dimmer over time. 
White dwarfs with the same mass as the Sun are about the same size as the Earth. Higher mass white dwarfs are smaller. 
\begin{definition} White Dwarf Limit: a white dwarf's mass cannot reach more than 1.4 times the mass of the Sun; this is due to electrons moving faster as they are squeezed into a tight space and thus moving towards the speed of light, but nothing can overcome that speed \end{definition}
In a close binary system, gas from a companion star can spill toward a white dwarf, forming a swirling accretion disk around it. 
\begin{definition} Accretion Disk: matter orbiting a white dwarf that fell from a close binary system with some angular momentum \end{definition}
Friction between orbiting rings of matter in the disk transfers angular momentum outward and causes the  disk to heat up and glow. The temperature of accreted matter eventually becomes hot enough for hydrogen fusion. 
\begin{definition} Nova: caused by hydrogen fusion on the surface of a white dwarf in a binary star system; can last for a few weeks \end{definition}
The nova star system temporarily appears much brighter. The explosion drives drives accreted matter out into space. If a white dwarf gains enough matter to exceed the 1.4-solar mass white dwarf limit, it will explode completely into a white dwarf supernova. \\~\\
Types of Supernovas \begin{itemize} 
\item Massive Star Supernova: iron core of massive star reaches white dwarf limit and collapses into a neutron star, causing an explosion 
\item White Dwarf Supernova: carbon fusion suddenly begins as white dwarf in close binary system reaches white dwarf limit, causing a total explosion \end{itemize}
One way to tell supernova types apart is with a light curve showing how luminosity changes with time. \\~\\
Supernovas are much more luminous than novas (about 10 million times). Their light curves differ as well as spectra. Exploding white dwarfs don't have hydrogen absorption lines. 

\subsection{Neutron Stars}
\begin{definition} Neutron Star: a ball of neutrons created by the collapse of the iron core in a massive star supernova; about the size of a small city \end{definition}
Neutron degeneracy pressure supports a neutron star against gravity. \\~\\
Using a radio telescope in 1967, Jocelyn Bell noticed very regular pulses of radio emission coming from a single part of the sky. 
\begin{definition} Pulsar: a spinning neutron star that emits rapidly pulsing radio radiation along a magnetic axis that is not aligned with the rotation axis \end{definition} 
Neutron stars can spin rapidly and emit beams of radiation along their magnetic poles, which we detect as pulses of radiation if the beams sweep by Earth. The radiation beams sweep through space like lighthouse beams as the neutron star rotates. \\~\\
Note: All pulsars are neutron stars but not all neutron stars are pulsars. \\~\\
Matter falling toward a neutron star forms an accretion disk, just as in white dwarf binary. Accreting matter adds angular momentum to a neutron star, increasing its spin. 
\begin{definition} X-Ray Bursts: occurs when matter accreting onto a neutron star becomes hot enough for helium to fuse; episodes of fusion on the neutron star; produces a burst of X rays; typically flare every few hours to every few days; each burst lasts only a few seconds \end{definition} 
Quantum mechanics says that neutrons in the same place cannot be in the same state. Neutron degeneracy pressure can no longer support a neutron star against gravity if its mass is greater than three times the mass of the Sun. 

\subsection{Black Holes: Gravity's Ultimate Victory}
\begin{definition} Black Hole: an object whose gravity is so powerful that not even light can escape it \end{definition} 
Some massive star supernovas can make a black hole if enough mass falls onto the core. \\~\\
Light would not escape Earth's surface if it could be shrunken down to less than 1 cm. \\~\\
The ``surface" of a black hole is the radius at which the escape velocity equals the speed of light. 
\begin{definition} Event Horizon: the boundary between the inside of the black hole and the universe outside \end{definition} 
The event horizon of a black hole three times the mass of the Sun is as big as a small city. Event horizon is larger for black holes of larger mass. 
\begin{definition} Schwarzschild Radius: the radius of the event horizon \end{definition} 
A black hole's mass strongly warps space and time in the vicinity of the event horizon. Nothing can escape from the within the event horizon because nothing can go faster than light. No escape means there is no more contact with something that falls in. It increases the hole's mass, changes its spin or charge, but otherwise loses its identity. 
\begin{definition} Singularity: a point in the black hole's center where gravity crushes all matter into a single point \end{definition} 
If the Sun shrank into a black hole, its gravity would be different only near the event horizon. Note: Black holes don't suck. If you fell toward a black hole, you would rapidly accelerate and soon cross the event horizon. But to someone watching from afar, your fall would appear to take forever. 
\begin{definition} Gravitational Redshift: caused by light waves taking extra time to climb out of a deep hole in spacetime \end{definition} 
Note: Time passes more slowly near the event horizon. \\~\\
Tidal forces near the event horizon of a black hole three times the mass of the Sun would be lethal to humans. Tidal forces would be gentler near a supermassive black hole because its radius is much bigger. \newpage
To Verify a Black Hole \begin{itemize} 
\item Need to measure mass \begin{itemize} 
\item Use orbital properties of companion 
\item Measure velocity and distance of orbiting gas \end{itemize} 
\item It's a black hole if it's not a star and its mass exceeds the neutron star limit (three times the mass of the Sun) \end{itemize} 
Some X-ray binaries probably contain compact objects of mass exceeding three times the mass of the Sun that are likely to be accreting black holes rather than accreting neutron stars. 

\subsection{The Origin of Gamma-Ray Bursts}
\begin{definition} Gamma-Ray Bursts: brief bursts of gamma rays coming from space; first detected in the 1960s \end{definition} 
Observations in the 1990s showed that many gamma-rays bursts were coming from very distant galaxies. They must be among the most powerful explosions in the universe - could be the formation of a black hole. Observations show that at least some gamma-ray bursts are produced by supernova explosions. Some others may come from collisions between neutron stars.  

\section{Our Galaxy}
\subsection{The Milky Way Revealed} 
\begin{definition} Spiral Galaxy: composed of many spiral arms that contains billions of stars; the Milky Way Galaxy is one example of a spiral galaxy, holding over 100 billion stars \end{definition}
The spiral arms are part of a fairly flat disk of stars surrounding a bright central bulge. The entire disk is surrounded by a dimmer, rounder halo. Most of the galaxy's bright stars reside in its disk. The most prominent stars in the halo are found in about 200 globular clusters of stars. 
\begin{definition} Interstellar Medium: clouds of interstellar gas and dust; absorbs visible light; obscures our view \end{definition}
Note: The Milky Way is a large galaxy and several smaller galaxies orbit it. \\~\\
Disk stars orbit the galaxy's center in orderly circles that all go in the same direction, bobbling slightly up and down as they orbit. Halo stars travel high above and far below the disk on orbits with random orientations. Bulge stars also have orbits with random orientations. \\~\\
We can calculate the mass of the galaxy within the Sun's orbit using the Sun's orbital properties and Newton's version of Kepler's third law. \\~\\
The orbits of the Milky Way's stars reveal that most of the galaxy's mass consists of invisible dark matter in the halo. 

\subsection{Galactic Recycling}
\begin{definition} Star-Gas-Star Cycle: the process of recycling gas from old stars into new star systems
\end{definition}
High mass stars have strong stellar winds that blow bubbles of hot gas. Lower mass stars return gas to stellar space through stellar wind and planetary nebula. X rays from hot gas in supernova remnants reveal newly made heavy elements. A supernova remnant cools down and begins to emit visible light as it expands. New elements made by the supernova mix into the interstellar medium. Multiple supernovas create huge hot bubbles that can blow out of the disk. This spreads their contents over a large region of the galaxy. Gas clouds cooling in the halo can rain back down on the disk. 
\begin{definition} Atomic Hydrogen Gas: forms when hot gas cools \end{definition}
\begin{definition} Molecular Cloud: forms after the formation of atomic hydrogen gas; gas cools down enough to allow atoms to combine into molecules \end{definition}
Note: Gas heated by supernova first cools into atomic hydrogen clouds and then cools further into molecular clouds. \\~\\
Gravity forms stars out of the gas in molecular clouds, completing the star-gas-star cycle. Ultraviolet radiation from newly forming stars can erode the molecular clouds that gave birth to them. \\~\\
Summary of Galactic Recycling \begin{enumerate} 
\item Stars make new elements by fusion 
\item Dying stars expel gas and new elements, producing hot bubbles 
\item Hot gas cools, allowing atomic hydrogen clouds to form
\item Further cooling permits molecules to form, making molecular clouds 
\item Gravity forms new stars (and planets) in molecular clouds \end{enumerate} 
\begin{definition} Dust Grain: tiny, solid flecks of carbon and silicon minerals that resemble particles of smoke and form in the winds of red giant stars; remains in the interstellar medium until they are heated and destroyed by a passing shock wave or incorporated into a protostar \end{definition}
We observe the star-gas-star cycle operating in the Milky Way's disk using many different wavelengths of light. Light at the optical (visible) part of the electromagnetic spectrum coming from the Milky Way is blocked by gas clouds. Infrared light reveals stars whose visible light is blocked by gas clouds. X rays are observed from hot gas above and below the Milky Way's disk. Radio waves emitted by atomic hydrogen show where gas has cooled and settled into the disk. Radio waves from carbon monoxide show locations of molecular clouds. Long wavelength infrared emission shows where young stars are heating dust grains. Gamma rays show where cosmic rays from supernova collide with atomic nuclei in gas clouds. 
\begin{definition} Ionization Nebulae: colorful wispy blobs of glowing gas near short lived high mass stars signifying active star formation \end{definition}
Hot, massive stars and ionization nebulae are found only near clouds that are actively forming stars. 
\begin{definition} Reflection Nebulae: scatters light through space; looks bluer than nearby stars because interstellar dust grains scatter blue light more easily than red light \end{definition}
Note: At the halo, no ionization nebulae, thus no blue stars, thus no star formation. \\ At the disk, there are ionization nebulae, thus there are blue stars thus there is star formation. 
\begin{definition} Spiral Density Waves: disturbances that produces spiral arms \end{definition}
\begin{definition} Spiral Arms: waves of star formation \end{definition}
Much of star formation in disk happens in the spiral arms. 
\begin{itemize} 
\item Gas clouds get squeezed as they move into spiral arms 
\item The squeezing of clouds triggers star formation 
\item Young stars flow out of spiral arms \end{itemize} 

\subsection{The History of the Milky Way}
\begin{definition} Halo Stars: 0.02-0.2\% heavy elements (O, Fe, ...), only old stars \end{definition} 
\begin{definition} Disk Stars: 2\% heavy elements, stars of all ages \end{definition}
Halo stars formed first, then stopped. Disk stars formed later and kept forming. \\~\\
Our galaxy probably formed from a giant gas cloud. Halo stars formed first as gravity caused the cloud to contract. The remaining gas settled into a spinning disk. Stars continuously form in the disk as the galaxy grows older. 
\begin{definition} Disk Population: contains both young stars and old stars, all of which have heavy-element proportions of about 2\%, like the Sun \end{definition}
\begin{definition} Spheroidal Population: consists of stars in the halo and the bulge, both of which are roughly spherical in shape; stars are old and low in mass and can have heavy-element proportions as low as 0.02\% \end{definition}
Halo stars are all old with a very low proportion of heavy elements, while disk stars come in all ages and contain a higher proportion of heavy elements. 
\begin{definition} Protogalactic Cloud: a cloud containing all the hydrogen and helium gas that eventually turned into stars; one of the earliest model of the formation of the galaxy \end{definition}
Halo stars formed first when our galaxy's protogalactic cloud was still still large and blobby as gravity caused the cloud to contract. The remaining gas settled into a spinning disk. Disk stars formed after the gas had settled into a spinning disk. \\~\\
Our galaxy's halo stars may have formed in several small protogalactic clouds that later merged to form one large protogalactic cloud. 

\subsection{The Mysterious Galactic Center}
Infrared light and radio emission comes from the center of the galaxy. Zoomed in even more shows there is swirling gas in the center of the galaxy. Zoomed in even more shows orbiting stars near the center. 
Stars quite close to our galaxy's center orbit a nearly invisible and tiny object about 4 million times as massive as our Sun - probably a huge black hole. X ray flares from the galactic center suggest that tidal forces of suspected black hole occasionally tear apart chunks of matter about to fall in. 

\section{Galaxies and the Foundation of Modern Cosmology}
\subsection{Islands of Stars}
Our deepest images of the universe show a great variety of galaxies, some of them billions of light years away. A galaxy's age, its distance and the age of the universe are all closely related. 
\begin{definition} Cosmology: the study of the overall structure and evolution of the universe \end{definition}
Types of Galaxies \begin{itemize} 
\item Spiral Galaxy: looks like flat white disks with yellowish bulges at their centers; disks are filled with cool gas and dust, interspersed with hotter ionized gas and has spiral arms (the Milky Way is one of them) \begin{itemize}
\item Barred Spiral Galaxy: has a bar of stars across the bulge, two spiral arms
\item Lenticular Galaxy: has a disk like a spiral galaxy but much less dusty gas (intermediate between spiral and elliptical) \end{itemize}
\item Elliptical Galaxy: redder, rounder and often longer in one direction than in the other; contain very little cool gas and dust though they often contain very hot ionized gas; all spheroidal component, virtually no disk component - red, yellow color indicates older star population
\item Irregular Galaxy: appear neither disklike nor rounded - blue-white color indicate ongoing star formation \end{itemize} 
Spiral and irregular galaxies look white because they contain stars of all different colors and ages, while elliptical galaxies look redder because old, reddish stars produce most of their light. 
\begin{definition} Disk Component: the flat disk in which stars follow orderly, nearly circular orbits around the galactic center; always contains an interstellar medium but amount and proportions of molecular, atomic and ionized gases in this medium differ from one spiral galaxy to the next; stars of all age, many gas clouds \end{definition}
\begin{definition} Spheroidal Component: the bulge and halo together; stars in her have orbits with many different inclinations and contains little cool gas and dust; old stars, few gas clouds \end{definition}
Spiral galaxies tend to congregate in small groups of galaxies (up to a few dozen per group), while elliptical galaxies are primarily found in large clusters (hundreds to thousands). 

\subsection{Distances of Galaxies} 
Brightness alone does not provide enough information to measure distance. 
\begin{definition} Radar Ranging: a technique to measure the AU in which radio waves are transmitted from Earth and bounded off Venus \end{definition} 
\begin{definition} Standard Candle: an object whose luminosity can be determined without measuring its distance \end{definition} 
\begin{definition} Cepheid Variable Stars (Cepheid): useful for measuring distances due to it being very luminous \end{definition} 
Steps of Main Sequence Fitting \begin{enumerate} 
\item Determine size of solar system using radar 
\item Determine distances of stars out to a few hundred light years using parallax \begin{itemize}
\item To get brightness, $$\text{Brightness} = \frac{\text{Luminosity}}{4\pi(\text{distance})^2} $$ 
\item To get star distance, $$\text{distance} = \frac{\text{Luminosity}}{\sqrt{4\pi \times \text{brightness}}} $$ \end{itemize} \newpage
\item Apparent brightness of star cluster's main sequence tells its distance \begin{itemize}
\item Knowing a star cluster's distance, the luminosity of each type of star within it can be determined 
\item Cepheid variable stars are useful for measuring distances because a Cepheid's luminosity can be determined from the period between its peaks of brightness
\item The light curve of the Cepheid shows that its brightness alternately rises and falls over a 50 day period
\item Cepheid with longer periods have greater luminosities \end{itemize}
\item Because the period of a cepheid variable star tells its luminosity, the star can be used as a standard candle; white dwarf supernova can also be used as standard candles 
\item Apparent brightness of a white dwarf supernova tells the distance to its galaxy (up to 10 billion light years) \end{enumerate} 
Before Hubble, some scientists argued that ``spiral nebulae" were entire galaxies like the Milky Way, whereas other scientists maintained that they were smaller collection of stars within the Milky Way. The debate remained unsettled until someone finally measured the distances of spiral nebulae. Hubble settled the debate by measuring the distance to the Andromeda Galaxy using cepheid variables as standard candles. Hubble also knew that the spectral features of virtually all galaxies are redshifted and thus they're all moving away from us. By measuring distances to galaxies, Hubble found that redshift and distance are related in a special way. \\~\\
A galaxy's redshift tells us how fast it is moving away from us, and the relationship between redshift and distance shows that the universe is expanding. Redshift of a galaxy tells us its distance. 
\begin{definition} Hubble's Law: expresses a relationship between galaxy speeds and distances and hence allows to determine a galaxy's distance from its speed $$v = H_0 \times d $$ where $v$ is the galaxy's velocity away from us, $d$ is is the distance and $H_0$ is Hubble's constant \end{definition} 
Hubble's constant tells us the age of the universe because it relates velocities and distances of all galaxies. 
$$ \text{Age} = \frac{\text{Distance}}{\text{Velocity}} = \frac{1}{H_0} $$ 
The expansion rate appears to be the same everywhere in space. The universe has no center nor edge (as far as we can tell). \newpage
Distance Chain \begin{enumerate} 
\item Radar ranging \item Parallax \item Main-sequence fitting \item Cepheid variables \item Distant standards \item Hubble's law \end{enumerate} 
Note: The expansion of the universe implies that the universe came into being at a single moment in time. 
\begin{definition} Cosmological Principle: The universe looks about the same no matter where you are within it \begin{itemize} 
\item Matter is evenly distributed on very large scales in the universe 
\item No center and no edges 
\item Not proved but consistent with all observations to date \end{itemize} \end{definition} 
The rate at which the universe expands tells us how old it is - about 14 billion years. 
\begin{definition} Lookback Time: the time it took for the object's light to reach us \end{definition} 
\begin{definition} Cosmological Redshift: the shifting to longer redder wavelengths due to the expansion of the universe stretching out all the photons within it \end{definition} 
\begin{definition} Cosmological Horizon: a boundary in time, not in space, that marks the limits of the observable universe \end{definition} 
Note: The size of the observable universe is determined by the age of the universe. 

\subsection{Galaxy Evolution}
\begin{definition} Galaxy Evolution: the formation and development of galaxies \end{definition} 
Deep observations show very distant galaxies as they were much earlier in time (old light from young galaxies). Images of the deep universe allow us to study galaxies at many different distances and therefore many different ages. \newpage
Models of galaxy formation assume the following: \begin{itemize} 
\item Hydrogen and helium gas filled all of space more or less uniformly when the universe was very young 
\item The distribution of matter in the universe was not perfectly uniform - certain regions of the universe started out slightly more denser than others \end{itemize} 
Early in time, the gas in a cubic region of the universe was almost uniformly distributed. Gravity drew gas into the denser regions of space as time passed by. Protogalactic clouds formed in the densest regions and went on to become galaxies. \\~\\ 
Conditions in the Protogalactic Cloud \begin{enumerate} 
\item Protogalactic Spin: initial angular momentum of the protogalactic cloud could determine the size of the resulting disk 
\item Density: elliptical galaxies could come from dense protogalactic clouds that were able to cool and form stars before gas settled into a disk \end{enumerate} 
Observations of some distant red elliptical galaxies support the idea that most of their stars formed very early in the history of the universe. Elliptical galaxies may have formed from protogalactic clouds that were spinning more slowly or were denser than those that formed spiral galaxies. \\~\\
Galactic collisions were much more likely to occur early in time because galaxies were closer together. Many of the galaxies we see at great distances (and early times) do indeed look very violently disturbed. The collisions we observe nearby trigger bursts of star formation. \\~\\ 
Collision of Galaxies \begin{enumerate} 
\item Two simulated spiral galaxies approach each other on a collision course 
\item The first encounter begins to disrupt the two galaxies and sends them into orbit around each other 
\item As the collision continues, much of the gas in the disk of each galaxy collapses toward the center 
\item Gravitational forces between the two galaxies tear out long streamers of stars called tidal tails 
\item The centers of the two galaxies approach each other and begin to merge 
\item The single galaxy resulting from the collision and merger is an elliptical galaxy surrounded by debris \end{enumerate} 
Computer models show that a collision between two spiral galaxies can form an elliptical galaxy. Collisions may explain why elliptical galaxies tend to be found where galaxies are closer together. Giant elliptical galaxies at the centers of clusters seem to have consumed a number of smaller galaxies. 

\subsection{Quasars and Other Active Galactic Nuclei} 
\begin{definition} Supermassive Black Holes: black holes whose gigantic accretion disks supple energy to quasars (active galactic nuclei) \end{definition}
\begin{definition} Active Galactic Nuclei: the center of an usually bright galaxy \end{definition} 
\begin{definition} Quasar: the most luminous active galactic nuclei; in some cases produce more light than 1000 galaxies the size of the Milky Way \end{definition} 
The highly redshifted spectra of quasars indicate large distances. From brightness and distance, we find that luminosities of some quasars are $10^{12}$ times as luminous as the Sun. Variability shows that all this energy comes from a region smaller than the solar system. The incredible luminosities of active galactic nuclei and quasars are apparently being generated in a volume of space not much bigger than the solar system. Galaxies around quasars sometimes appear disturbed by collisions. Quasars powerfully radiate energy over a very wide range of wavelengths, indicating that they contain matter with a wide range of temperatures. 
\begin{definition} Radio Galaxies: galaxies that contain active nuclei shooting out vast jets of plasma, which emit radio waves coming from electrons moving at near the speed of light \end{definition} 
The lobes of radio galaxies can extend over hundreds of thousands of light years. The speed of ejection suggests that a black hole is present. Thus, quasars are probably powered by matter falling into supermassive black holes. \\~\\
Characteristics of Active Galaxies \begin{itemize} 
\item Luminosity can be enormous (up to at least $10^{12}$ times that of the Sun)
\item Luminosity can vary rapidly (comes from a space smaller than the solar system)
\item They emit energy over a wide range of wavelengths (contain matter with wide temperature range)
\item Some drive jets of plasma at near light speed \end{itemize} 
The accretion of gas onto a supermassive black hole appears to be the only way to explain all the properties of quasars. The gravitational potential energy of matter falling into a black hole turns into kinetic energy. Friction in the accretion disk turns kinetic energy into thermal energy (heat). Heat produces thermal radiation (photons). This process can convert 10-40\% of $E = mc^2$ into radiation. \\~\\ 
Jets are thought to come from the twisting of a magnetic field in the inner part of the accretion disk. \\~\\
Orbits of stars at center of Milky Way indicate a black hole with mass of 4 million times that of the Sun. Orbital speed and distance of gas orbiting center of M87 indicate a black hole with mass of at least 3 billion times that of the Sun. \\~\\ 
Many nearby galaxies - perhaps all of them - have supermassive black holes at their centers. These black holes seem to be dormant active galactic nuclei. All galaxies may have passed through a
quasar-like stage earlier in time. The mass of a galaxy's central black hole is closely related to the mass of its bulge. The development of a central black hole must somehow be related to galaxy evolution. 





\end{document}










