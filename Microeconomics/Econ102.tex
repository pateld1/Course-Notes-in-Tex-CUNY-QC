\documentclass[12pt]{article}
\usepackage[letterpaper, portrait, margin=1in]{geometry}
\usepackage{amsmath, amsthm, amssymb, mathrsfs}

\usepackage{fancyhdr}
\pagestyle{fancy}
\fancyhf{}
\lhead{Darshan Patel}
\rhead{ECON 102: Microeconomics}
\renewcommand{\footrulewidth}{0.4pt}
\cfoot{\thepage}

\begin{document}

\theoremstyle{definition}
\newtheorem{definition}{Definition}[section]
\newtheorem{formula}{Formula}[section]

\title{ECON 102: Microeconomics}
\author{Darshan Patel}
\date{Spring 2017}
\maketitle

\tableofcontents





\section{The Big Ideas}
\subsection{Introduction} 
\begin{definition} Microeconomics: deals with the individual (consumer, producer), one firm/company, price of goods \end{definition}
\begin{definition} Macroeconomics: deals with the whole economy - country, unemployment, inflation rate, trade \end{definition}
Why do we study economy? We have limited resources but also unlimited wants. Thus we have to make choices and decisions, either on an individual level or government level, micro level or macro level respectively. 
\begin{definition} Resources: labor, land, capital, and entrepreneurship; also called inputs, factors of production \end{definition} 
Note: Money comes under financial capital, not physical capital which is what the above refers to. Capital is all the goods that you use in a firm to allow it to function. 

\subsection{Big Idea One: Incentives Matter} 
Incentives can change depending on the situation at hand. It is everywhere. People respond to incentives of all kinds: fame, power, reputation, love. Prices are lowered during a sale and thus people buy more. 

\subsection{Big Idea Two: Good Institutions Align Self-Interest with the Social Interest} 
When self-interest aligns with the broader public interest, good outcomes come out. But when self-interest and the social interest are at odds, bad outcomes come out. Adam Smith said that when markets work well, those who pursue their own interest end up promoting the social interest, as if led to do so by an ``invisible hand." Markets, however, do not always align self-interest with the social interest. Sometimes, the invisible hand is absent, not just invisible. Market incentives can be too strong. When markets don't properly align self-interest with the social interest, government can sometime improve the situation by changing incentives with taxes, subsidies or other regulations. Private property is one example of good institutions. 

\subsection{Big Idea Three: Trade-offs Are Everywhere} 
Trade-offs are everywhere. Resources are scarce therefore something must be traded/give up to receive resources. 
\begin{definition} Opportunity Cost: the value of the opportunities lost of a choice; the things you give up \end{definition}
Opportunity cost affects our decisions and behavior. Recognizing trade-offs is the first step to making wise choices. Most of the time, people do respond to changes in opportunity costs - even when money costs have not changed. 

\subsection{Big Idea Four: Thinking on the Margin} 
\begin{definition} Marginal: extra, additional \end{definition}
Thinking on the margin is just making choices by thinking in terms of marginal benefits and marginal costs, the benefits and costs of a little bit more (or a little bit less). Most of our decisions in life involve a little bit more of something or a little bit less, and it turns out that thinking on the margin is also useful for understanding how consumers and producers make decisions. 

\subsection{Big Idea Five: The Power of Trade}
Trade is good because everyone benefits out of it, Both countries benefit when they trade with each other. The real power of trade is the power to increase production through specialization which brings down prices. Few of us could survive if we had to produce our own food, clothing and shelter. We survive and prosper only because specialization increases productivity. Trade also allows us to take advantage of economics of scale, the reduction in costs when goods are mass-produced. Everyone can benefit from trade, even those who are not especially productive because productive people can't do everything. The theory of comparative advantage says that when people or nations specialize in goods in which they have a low opportunity cost, they can trade to mutual advantage. The greater the productivity of American business in producing goods, the greater will be the demand to trade for its manufacturing goods. If trade is not beneficial, it is inefficient and there is no use of doing so. 

\subsection{Big Idea Six: The Importance of Wealth and Economic Growth} 
Wealth can bring solutions to problems such as health issues. Wealthier economics lead to richer and more fulfilled human lives. Economic growth brings up standards of living. 

\subsection{Big Idea Seven: Institutions Matter}
Incentives causes some countries to have more physical and human capital and to be organized well using the latest technological knowledge. Entrepreneurs, investors, and saves need incentives to save and invest in physical capital, human capital, innovation, and efficient organization. Property rights, political stability, honest government, a dependent legal system and competitive and open markets are all powerful institutions for supporting good incentives. New ideas require incentives. Ideas aren't used up when they are used and that has tremendous implications for understanding the benefits of trade, the future of economic growth and many other topics. Good financial institutions help with economic growth. People bring their money to America because it is secure here and dependable. This is due to the legal system and rules and regulations. 

\subsection{Big Idea Eight: Economic Booms and Busts Cannot Be Avoided but Can Be Moderated} 
No economy grows at a constant pace. Economies advance and recede, rise and fall, boom and bust. Recessions are not all avoidable. Booms and busts are part of the normal response of an economy to changing economic conditions. Not all booms and busts are normal though, such as the Great Depression. The tools of monetary and fiscal policy can reduce swings in unemployment and GDP during a recession. Eventually, over time, we get better off through the business cycle (booms and bursts). Government try to control it using fiscal and monetary policy. 
\begin{definition} Fiscal Policy: changes in taxes and federal money; comes from President and Congress \end{definition}
\begin{definition} Monetary Policy: changes in money supply and interest rates; controlled by the Federal Reserve (central banking); fights off recession \end{definition}

\subsection{Big Idea Nine: Prices Rise When the Government Prints Too Much Money} 
\begin{definition} Inflation: an increase in the general level of prices by increasing money supply; done by the central bank \end{definition}

\begin{definition} Hyper-inflation: too much inflation to the point where there is no control \end{definition} 
Inflation comes about when there is a sustained increase in the supply of money. When people have more money, they spend it, and without an increase in the supply of goods, prices must rise. 

\subsection{Big Idea Ten: Central Banking Is a Hard Job} 
The US central bank, the Federal Reserve Bank, is often called on to combat recessions. But this is not easy to do because there is a lag - often of many months - between when the bank makes a decision and when the effects of that decision on the economy are known. No one can foresee the future perfectly and so the bank's decisions are not always the right one. 
Printing too much money can create inflation and printing too little can create a recession. Having an inflation rate is 2\% is better than 0\% because 0\% is too dangerous and can cause deflation if prices fall. 

\subsection{The Biggest Idea of All: Economics Is Fun} 
Economics teaches us how to make the world a better place. It's about the difference between wealth and poverty, work and unemployment, happiness and squalor. Economics increases your understanding of the distant past, present events and future possibilities. The basic principles of economics hold everywhere. Economics is also linked to everyday life. 


\section{The Power of Trade and Comparative Advantage} 
\subsection{Trade and Preferences} 
Trade makes people with different preferences better off. 

\subsection{Specialization, Productivity and the Division of Knowledge} 
in a world without trade, no one can afford to specialize. People will specialize in the production of a single good only when they are confident that they will be able to trade that good for the many other goods that they need. Thus, as trade develops, so does specialization and specialization turns out to vastly increase productivity. The extent of specialization in a modern economy explains why no one knows the full details of how even the simplest product is produced. The extent of specialization in modern society is immense. The division of knowledge increases with specialization and trade. Economic growth in the modern era is primarily due to the creation of new knowledge. Every increase in world trade is an opportunity to increase the division of knowledge and extend the power of the human mind. 

\begin{definition} Production Possibilities Frontier: plots how many goods can be produced based on how many of another good is produced; shows the difference combinations of 2 goods that can be produced \end{definition}
Note: Any combination shown as a point on the curve is possible. Two different possibilities cannot occur at the same time. \\~\\
The absolute value of the slope of the PPF tells how much of another good is given up for an additional good, or opportunity cost. 

\subsection{Comparative Advantage} 
Trades allow people and countries to take advantage of differences. 
\begin{definition} Absolute Advantage: the ability to produce the same good using fewer inputs than another producer \end{definition} 
A producer has an absolute advantage over another producer if it can produce more output from the same input. To benefit from trade, a country need not have an absolute advantage in production, such as if there's a bigger advantage in making one product over another. \\ A country has an absolute advantage in two products if it can produce more of it compared to another country. 
\begin{definition} Production Possibilities Frontier: shows all the combinations of goods that a country can produce given its productivity and supply of inputs \end{definition} 
A PPF illustrates trade-offs. 
\begin{definition} Comparative Advantage: the ability to produce goods for which it has the lowest opportunity cost \end{definition}
When each country produces according to its comparative advantage and then trades, total production and consumption increases. A country (or a person) will always be the low-cost seller of some good because the greater the advantage a country/person has in producing A, the greater the cost to it of producing B. A producer has an absolute advantage over another producer if it can produce more output from the same input. High-productivity countries have high wages, low-productivity countries have low wages. Trade means that workers in both countries can raise their wages to the highest levels allowed for by their respective productivities. \\~\\ Comparative advantage is not possible to have in both items. \newpage
To find comparative advantage, find the absolute value of the slope of both PPFs. For example: the slope of a shirt/boot PPF in US is -1 and -3/4 in Mexico. Then: 1 shirt costs 1 boot in US and 1 shirt costs 3/4 boots in Mexico. On the other hand, 1 boot costs 1 shirt in US and 1 boot costs 4/3 shirt in Mexico. Boots costs more in Mexico and so, the US has a comparative advantage in boots. Shirts costs more in the US and so, Mexico has a comparative advantage in it. Boots are produced in the US and shirts are produced in Mexico. \\
\noindent Before trade: GDP for Mexico = $100 \times 3 + 100 \times 6 = \$900$ in production \\ Wage = $900/24 \text{ workers} = \$37.50$ 
Before trade: GDP for US: $300 \times 12 + 200 \times 100 = 4800$ \\ Wage = $4800 /24 = \$200$ 
After trade: Mexico: $1 \times 300 + 9 \times 100 = \$1200$ \\ Wage = $1200/24 = \$50$
After trade: US: $ 13 \times 300 + 13 \times 100 =  5200$ \\ Wage = $ 5200 / 24 = \$216 $ 
\begin{definition} Term of Trade: determined from between the magnitudes of slopes of two PPFs \end{definition}
If the term of trade is closer to one country than the other, the other country is better off. If the term of trade is equal to one of the country's stope, that country does not benefit at all. 

\subsection{Trade and Globalization}
Globalization has been a theme in human history since at least the Roman Empire, which knit together different parts of the world in a common economic and political area. When these trade networks later fell apart, the subsequent era was named ``The Dark Ages." Later, the European Renaissance arose from revitalized trade routes, the rebirth of commercially based cities, and also the spread of science from China, India and the Middle East. Periods of increased trade and the spread of ideas have been among the best for human progress. 


\section{Supply and Demand}
\subsection{The Demand Curve} 
\begin{definition} Market: the interaction between buyers and sellers \end{definition}
Conditions for Competitive Markets: \begin{itemize} 
\item large number of buyers and sellers, (a monopoly has one seller) 
\item no differentiation in product (product is the same whether it's from one seller or another)
\item no barriers for entry/exit
\end{itemize} 
\begin{definition} Demand Curve: a function that shows the quantity demanded at different prices; represented by a downward sloping curve \end{definition}
\begin{definition} Quantity Demanded: the quantity that buyers are willing and able to buy at a particular price \end{definition}
Note: The lower the price, the greater the quantity demanded
\begin{definition} Consumer Surplus: the consumer's gain from exchange, or the difference between the maximum price a consumer is willing to pay for a certain quantity (height of the demand) and the market price (actual price) - represents how much the buyer gained \end{definition}
\begin{definition} Total Consumer Surplus: triangular region measured by the area beneath the demand curve and above the price \end{definition}
An increase in demand shifts the demand curve outward, up and to the right because there is a greater willingness to pay for the same quantity. A decrease in demand shifts the demand curve inward, down and to the left because there is less willingness to pay for the same quantity. \\~\\
Important Demand Shifters \begin{itemize} 
\item Income: if people have more income, they can afford more goods and thus increase or decrease demand depending on type of good 
\item Population: if there's an increase in population, there's more demand for goods
\item Price of substitutes: prices of a substitute good changes (ex: American cars and Japanese cars prices)
\item Price of complements: price of a related good changes (ex: cars and gas prices)
\item Expectations: expected future prices; weather effects on crops will increase demand for the crop before the crop becomes scarce and price is increased 
\item Tastes/Preferences: a change in people's taste can increase or decrease demand, such as from the effect of advertisements 
\end{itemize} 
\begin{definition} Normal Good: a good for which demand increases when income increases \end{definition}
\begin{definition} Inferior Good: a good for which demand decreases when income increases \end{definition}
Note: If two goods are substitutes, a decrease in the price of one good leads to a decrease in demand for the other good. On the other hand, if two goods are complements, a decreases in the price of one good leads to an increase in the demand for the other good. 

\subsection{The Supply Curve} 
\begin{definition} Supply Curve: a function that shows the quantity supplied at different prices; represented by a upward sloping curve  \end{definition}
\begin{definition} Quantity Supplied: the amount of good that sellers are willing and able to sell at a particular price\end{definition}
Note: The higher the price, the greater the quantity supplied. 
\begin{definition} Producer Surplus: the producer's gain from exchange, or the difference between the market price and the minimum price at which a producer would be willing to sell a particular quantity \end{definition}
\begin{definition} Total Producer Surplus: measured by the area above the supply curve and below the price \end{definition}
A decrease in costs means that the supply curve shifts down and to the right, increasing supply, because there is willingness to sell the same quantity at lower prices. Higher costs means that the supply curve shifts up and to the left, decreasing supply, because higher prices are required to sell the same quantity. \\~\\
Important Supply Shifters \begin{itemize} 
\item Technological innovations and changes in the price of inputs: a change in an input good affects the supply of the produced good; technology decreases costs and increases supply 
\item Taxes and subsidies: putting on a tax increases cost thus supply shifts to the left and decreases; putting on a subsidy decreases cost thus supply shifts to the right and increases 
\item Expectations: if price of a good is going to dramatically go up in the future, supplier will decrease supply for a better profit in the future 
\item Entry or exit of producers: if more producers come into the market, supply will increase; if producers leave the market, supply will decrease 
\item Changes in opportunity costs: looking at the market to see what product costs more in the market and shifting production towards that 
 \end{itemize} 
\begin{definition} Marginal Cost: the cost of an additional good to be produced; the height of the supply curve at a specific quantity \end{definition}
Note: For demand AND supply, going to the left is decrease is demand/supply while going to the right is increase in demand/supply. DO NOT ASSOCIATE UP AND DOWN WITH INCREASE AND DECREASE! \\~\\
If demand and supply both increase, separate them. Concentrate on demand first, keeping supply constant and analyze. Prices go up and quantity goes up. Then concentrate on supply, keeping demand constant and analyze. Prices go down and quantity increase. Thus quantity increases but price is uncertain. \\ If demand increases and supply decreases, first prices increase and quantity increases as well. Then prices go up and quantity goes down. Therefore, if demand increases and supply decreases, prices go up and quantity is uncertain. Strength of both changes is needed to examine the uncertainty. The stronger change will determine the effect on the uncertain change. 

\section{Equilibrium: How Supply and Demand Determine Prices} 
\subsection{Equilibrium and the Adjustment Process}
\begin{definition} Surplus: a situation in which the quantity supplied is greater than the quantity demanded; above the equilibrium \end{definition}
Note: Competition will push prices down whenever there is a surplus. 
\begin{definition} Shortage: a situation in which the quantity demanded is greater than the quantity supplied; below the equilibrium \end{definition}
Note: Competition will push prices up whenever there is a shortage. 
\begin{definition} Equilibrium Price: the price at which the quantity demanded is equal to the quantity supplied \end{definition}
Note: The equilibrium price is stable because the quantity demanded is exactly equal to the quantity supplied. \\~\\
Sellers compete with other sellers and buyers compete with other buyers.
\begin{definition} Invisible Hand: what the market does by itself without the aid of government; coined by Adam Smith in \textit{Wealth of Nations} \end{definition} If an individual follows his/her interest, society will benefit out of it. 

\subsection{Gains from Trade Are Maximized at the Equilibrium Price and Quantity} 
\begin{definition} Equilibrium Quantity: the quantity at which the quantity demanded is equal to the quantity supplied \end{definition}
Below the equilibrium quantity, there are unexploited gains from trade which don't last for long. \\~\\
A free market maximizes the gains from trade, meaning, a free market maximizes producer plus consumer surplus. \begin{itemize} 
\item The supply of goods is bought by the buyers with the highest willingness to pay
\item The supply of goods is sold by the sellers with the lowest costs 
\item Between buyers and sellers, there are no unexploited gains from trade and no wasteful trades \end{itemize}

\subsection{Terminology: Demand Compared with Quantity Demanded and Supply Compared with Quantity Supplied}
\begin{itemize} 
\item An increase in quantity demanded is a movement along a fixed demand curve caused by a shift in the supply curve 
\item An increase in demand is a shift in the demand curve up and to the right
\item An increase in supply is a shift in the supply curve down and to the right 
\item An increase in quantity supplied is a movement along a fixed supply curve caused by a shift in the demand curve \end{itemize}

\section{Elasticity and its Applications}
\subsection{The Elasticity of Demand}
\begin{definition} Price Elasticity of Demand: measures how responsive the quantity demanded is to a change in price; more responsive (sensitivity) means more elastic \end{definition}
If demand is inelastic, consumer's response is less responsive. If consumer's response is more responsive, demand is more elastic. \\~\\
Elasticity Rule: If two linear demand or supply curves run through a common point, then at any given quantity that is flatter is more elastic. \\~\\
Some Factors Determining the Price Elasticity of Demand $$\begin{tabular}{|c|c|} \hline
Less Elastic & More Elastic \\ \hline
Fewer substitutes & More substitutes \\ \hline
Short run (less time) & Long run (more time) \\ \hline
Categories of product & Specific brands \\ \hline 
Necessities & Luxuries \\ \hline 
Small part of budget & Large part of budget \\ \hline \end{tabular} $$ 
\begin{formula} Price Elasticity of Demand: 
$$E_d = \frac{\text{Percentage change in quantity demanded}}{\text{Percentage change in price}} = \frac{\%\Delta Q_{\text{Demanded}}}{\%\Delta\text{Price}} $$ \end{formula} \newpage
Degrees of Elasticity: \begin{itemize} 
\item If $|E_d| > 1$, then demand is elastic
\item If $|E_d| < 1$, then demand is inelastic 
\item If $|E_d| = 1$, then demand is unit elastic \end{itemize}

\begin{formula} Alternative Formula for Elasticity of Demand using Midpoint Method: 
$$\begin{aligned} E_d &= \frac{\%\Delta Q_{\text{Demanded}}}{\%\Delta\text{Price}} \\ &= \frac{\frac{\text{Change in quantity demanded}}{\text{Average Quantity}}}{\frac{\text{Change in price}}{\text{Average price}}} \\ &= \frac{\frac{Q_{\text{After}} - Q_{\text{Before}}}{(Q_{\text{After}} + Q_{\text{Before}})/2}}{\frac{P_{\text{After}} - P_{\text{Before}}}{(P_{\text{After}} + P_{\text{Before}})/2}} \\ &= \frac{\frac{Q_2 - Q_1}{0.5(Q_1 + Q_2)}}{\frac{P_2 - P_1}{0.5(P_1 + P_2)}} \\ &= \frac{\frac{Q_2 - Q_1}{Q_1 + Q_2}}{\frac{P_2 - P_1}{P_1 + P_2}} \end{aligned} $$ \end{formula}
Note: It does not matter which of the two prices/quantities is $P_1/P_2$ or $Q_1/Q_2$ but be consistent. 
\begin{formula} Revenue: $$\text{Revenue} = \text{Price} \times \text{Quantity} $$ or $$R = P \times Q $$ \end{formula}
Note: If the demand curve is inelastic, then revenues go up when the price goes up because quantity is not very responsive to price. If the demand curve is elastic, then revenues go down when the price goes up because quantity is very responsive to price. \\~\\
Elasticity and Revenue $$\begin{tabular}{|c|c|c|} \hline
Absolute Value of Elasticity & Name & How Revenue Changes with Price \\ \hline 
$|E_d| < 1$ & Inelastic & Revenue and price move together \\ \hline
$|E_d| > 1$ & Elastic & Revenue and price move in opposite directions \\ \hline
$|E_d| = 1$ & Unit Elastic & Revenue stays the same when price changes \\ \hline \end{tabular} $$
Elasticity is higher (elastic) at the top of the demand curve and lower (inelastic) at the bottom of the demand curve. \\~\\
A more elastic demand curve is a horizontal line. A more inelastic demand curve is a vertical line. \\~\\

\subsection{The Elasticity of Supply}
\begin{definition} Elasticity of Supply: measures how responsive the quantity supplied is to a change in price \end{definition}
Primary Factors Determining the Elasticity of Supply $$\begin{tabular}{|c|c|} \hline
Less Elastic & More Elastic \\ \hline
Difficult to increase production & Easy to increase production \\ 
at constant unit cost & at constant unit cost \\ \hline
Large share of market for inputs & Small share of market for inputs \\ \hline
Global supply & Local supply \\ \hline 
Short run & Long run \\ \hline \end{tabular} $$
\begin{formula} Elasticity of Supply: 
$$E_s = \frac{\text{Percentage change in quantity supplied}}{\text{Percentage change in price}} = \frac{\%\Delta Q_{\text{Supplied}}}{\%\Delta\text{Price}} $$ \end{formula}
\begin{formula} Alternative Formula for Elasticity of Supply using Midpoint Method: 
$$ \begin{aligned} E_s &= \frac{\%\Delta Q_{\text{Supplied}}}{\%\Delta\text{Price}} \\ &= \frac{\frac{\text{Change in quantity supplied}}{\text{Average Quantity}}}{\frac{\text{Change in price}}{\text{Average price}}} \\ &= \frac{\frac{Q_{\text{After}} - Q_{\text{Before}}}{(Q_{\text{After}} + Q_{\text{Before}})/2}}{\frac{P_{\text{After}} - P_{\text{Before}}}{(P_{\text{After}} + P_{\text{Before}})/2}} \end{aligned} $$ \end{formula}


\subsection{Using Elasticities for Quick Predictions}
\begin{formula} Price Changes using Elasticities: 
$$\text{Percent Change in price from a shift in demand} = \frac{\text{Percent change in demand}}{|E_d| + E_s} $$ 
$$\text{Percent change in price from a shift in supply} = \frac{\text{Percent change in supply}}{|E_d| + E_s} $$ \end{formula}

\subsection{Other Types of Elasticities} 
\begin{definition} Cross-Price Elasticity of Demand: measures how responsive the quantity demanded of good A is to the price of good B \end{definition} 
\begin{formula} Cross-Price Elasticity of Demand: 
$$ = \frac{\text{Percentage change in quantity demanded of good A}}{\text{Percentage change in price of good B}} = \frac{\%\Delta Q_{\text{Demanded, A}}}{\%\Delta P_{\text{Price, B}}} $$ \end{formula} 
\begin{formula} Alternative Formula for Cross-Price Elasticity of Demand using Midpoint Method: 
$$\begin{aligned} \frac{\frac{\text{Change in quantity demanded A}}{\text{Average Quantity A}}}{\frac{\text{Change in price B}}{\text{Average price B}}} = \frac{\frac{Q_{\text{After, A}} - Q_{\text{Before, A}}}{(Q_{\text{After, A}} + Q_{\text{Before, A}})/2}}{\frac{P_{\text{After, B}} - P_{\text{Before, B}}}{(P_{\text{After, B}} + P_{\text{Before, B}})/2}} \end{aligned} $$ \end{formula}
Note: If the cross-price elasticity is positive, an increase in the price of good B increases the quantity of good A demanded so the two goods are substitutes. If the cross-price elasticity is negative, an increase in the price of good B decreases the quantity of good A demanded so the two goods are complements. If the cross-price elasticity is zero, then there is no relation. 
\begin{definition} Income Elasticity of Demand: measures how responsive the quantity demanded of a good is with respect to changes in income \end{definition}
\begin{formula} Income Elasticity of Demand: 
$$\begin{aligned} \text{Income elasticity of demand} &= \frac{\text{Percentage change in quantity demanded}}{\text{Percentage change in income}} \\ &= \frac{\%\Delta Q_{\text{Demanded}}}{\%\Delta\text{Income}} \end{aligned} $$ \end{formula}
\begin{formula} Alternative Formula for Income Elasticity of Demand using Midpoint Method: 
$$\begin{aligned} \frac{\frac{\text{Change in quantity demanded}}{\text{Average Quantity}}}{\frac{\text{Change in income}}{\text{Average income}}} = \frac{\frac{Q_{\text{After}} - Q_{\text{Before}}}{(Q_{\text{After}} + Q_{\text{Before}})/2}}{\frac{I_{\text{After}} - I_{\text{Before}}}{(I_{\text{After}} + I_{\text{Before}})/2}} \end{aligned} $$ \end{formula}
Note: If the income elasticity of demand is positive, then the good is a normal good. If the income elasticity of demand is negative, then the good is an inferior good. If the income elasticity demand is between 0 and 1, then the good is a necessity. If the income elasticity of demand is greater than 1, then the good is a luxury good. 

\section{Taxes and Subsidies} 
\subsection{Commodity Taxes}
\begin{definition} Commodity Taxes: taxes on goods \end{definition} 
Truths about Commodity Taxation: \begin{itemize}
\item Who ultimately pays the tax does not depend on who writes the check to the government 
\item Who ultimately pays the tax does depend on the relative elasticities of demand and supply
\item Commodity taxation raises revenue and reduces the gains from trade (creates deadweight loss) \end{itemize}
\begin{formula} Tax: $$\text{The tax} = \text{Price paid by buyers} - \text{Price received by sellers} $$ \end{formula}
Tax shifts supply curve to the left, causing equilibrium price to increase. 
\begin{definition} Tax Incidence: when the tax is paid by both the producer and the consumer and not entirely by one group \end{definition} 
A good costs \$2.00. Government imposes a \$2 per unit tax on sellers, shifting the supply curve up \$2. The supply curve meets the demand curve at \$3.50 now. Therefore consumer pays \$1.50 of the tax and supplier pays \$0.50. Per unit tax multiplied by new quantity sold is the amount of total tax paid to the government. \\~\\ 
A tax wedge is the difference from the price paid by the buyer (top point of the tax wedge) and the price received by the producer (bottom point of the tax wedge). The tax the producer pays is the rectangular area from the tax wedge to the edges. \\
If demand is very elastic, the tax cannot be shifted to the consumer. If the demand is inelastic, the tax can be shifted to the consumer.  \\~\\ 
If tax is imposed on consumers, the demand curve will shift to the left. But the result is the same as if the tax was imposed on the producer.
To determine how much producer and consumer contribute to tax, move the curve until the tax is formed. The difference between the price paid by the buyer and the original equilibrium price is the amount consumer contribute to tax. The difference between the original equilibrium price and the price received by buyer is the amount producer contribute to tax.
The most important effect of a tax is to drive a tax wedge between the price paid by buyers and the price received by sellers. \\~\\ Note:  When demand is more elastic than supply, demanders/buyers pay less of the tax than sellers and producer will pay more of the tax. When supply is more elastic than demand, suppliers pay less of the tax than demanders/buyers and consumer pays more of the tax. \\~\\
When government imposes taxes, the prices become artificially too high. 
When government imposes tax on a market, it interferes with market efficiency making the market inefficient in the form of deadweight loss. If per unit tax is large, deadweight loss is greater and thus the market is less efficient. 
Tax decreases consumer and producer surplus and creates deadweight loss which no one gets. If the demand curve is relatively elastic, then the tax deters a lot of trades, $Q_{\text{tax}}$ is much less than $Q_{\text{no tax}}$, so the lost gains from trade are large. If the demand is relatively inelastic, then the tax does not deter many trades and $Q_{\text{tax}}$ is only slightly smaller than $Q_{\text{no tax}}$. The deadweight loss from taxation is also lower the less elastic the supply curve. If the supply curve is elastic, then the tax deters many trades. If the supply curve is inelastic, then there is little deterrence and thus, few lost gains from trade. If demand is more elastic and government imposes a tax, the deadweight loss will be large. If the demand is less elastic and government imposes a tax, the deadweight loss is small. 

\subsection{Subsidies}
\begin{definition} Subsidy: a reverse tax \end{definition}
Facts about Commodity Subsidies: \begin{itemize} 
\item Who gets the subsidy does not depend on who gets the check from the government 
\item Who benefits from a subsidy does depend on the relative elasticities of demand and supply
\item Subsidies must be paid for by taxpayers and they crease inefficient increases in trade (deadweight loss) \end{itemize} 
\begin{formula} Subsidy: $$\text{The subsidy} = \text{Price received by sellers} - \text{Price paid by buyers} $$ \end{formula}
With the subsidy, some non-beneficial trades do occur. A subsidy drives a wedge between the price received by the sellers and the price paid by the buyers, creating a deadweight loss, on the opposite side of the equilibrium point where the tax is. \\~\\
Subsidy also creates inefficiencies in the market by creating a deadweight loss. When government imposes subsidies, the prices become artificially too low. 
\\~\\
Note: Whoever bears the burden of a tax receives the benefit of a subsidy. When demand is more elastic than supply, suppliers bear more of the burden of a tax and thus receive more of the benefit of a subsidy. When supply is more elastic than demand, buyers bear more of the burden of a tax and thus receive more of the benefit of a subsidy. \\~\\ The more elastic either the demand or the supply curve is, the more a tax deters trade and the more trades that are deterred, the greater the deadweight loss. 

\section{Price Ceilings and Price Floors} 
\subsection{Price Ceilings}
\begin{definition} Price Ceiling: a maximum price allowed by law \end{definition}
Price ceilings create five important effects: \begin{itemize} 
\item Shortages (excess demand) - the lower the price ceiling, the more shortage there is because producers do not want to sell goods at a lower price 
\item Reductions in product quality - quality drops as price drops and there is shortage 
\item Wasteful lines and other search costs - gas stations during a gas shortage - total value of wasted time is the rectangular region between willingness to pay and price ceiling and axis to deadweight loss - illegal methods such as bribes prevalent 
\item A loss of gains from trade - deadweight loss is created when government steps in 
\item A misallocation of resources - if price cannot go up, resources cannot go to where it is demanded more \end{itemize}
\begin{definition} Deadweight Loss: the total of lost consumer and producer surplus when not all mutually profitable gains from trade are exploited; created by price ceilings \end{definition}

Some consumers pay the highest value(50\%); some pay the price ceiling(50\%). Thus the average price is the average of the lowest value and the highest value. On average, consumers valued the good at this price. Therefore the area from average price to control price is total consumer surplus under random allocation. The area from average price to highest value is loss due to random allocation. 

\subsection{Rent Controls}
\begin{definition} Rent Control: a price ceiling on rental housings, such as apartments \end{definition}
Rent controls create shortages, reduce quality, create wasteful lines and increase the costs of search, cause a lost of gains from trade and misallocate resources.

\subsection{Arguments for Price Controls} 
Without rent controls, some people may not be able to afford appropriate housing. Price controls, however, are never the only way to help the poor and they are rarely the best way. Price controls can also discipline monopolies. Another reason for price control may be that the public does not see the consequences of price controls. 

\subsection{Universal Price Controls}
Price controls in the US have caused shortages, lineups, delays, quality reductions, misallocations, bureaucracy and corruption.  An economy with permanent, universal price controls is in essence a ``command economy.''

\subsection{Price Floors}
\begin{definition} Price Floor: a minimum price allowed by law \end{definition}
Price floors create four important effects: \begin{itemize} 
\item Surpluses (excess supply) - the higher the price floor, the more surplus there is because consumers don't want to buy as many at a higher price
\item Lost gains from trade (deadweight loss) - quantity that can be sold by the producer is less than equilibrium quantity 
\item Wasteful increases in quality - excess supply because price cannot be lowered, must provide incentives (higher quality) to reason higher price (paying for quality at a high price) 
\item A misallocation of resources \end{itemize}


\section{Costs and Profit Maximization under Competition} 
\subsection{What Price to Set?}
\begin{definition} Long Run: the time after all exit or entry has occurred \end{definition}
\begin{definition} Short Run: the period before exit or entry can occur \end{definition} 
An industry is competitive when firms don't have much influence over the price of their product, assuming the following conditions \begin{itemize}
\item The product being sold is similar across sellers
\item There are many buyers and sellers, each small relative to the total market 
\item There are many potential sellers 
\item The price is known by everyone 
\item No barriers for entry/exit for producers 
\item The price is set by the market - no firm has control over it 
\end{itemize}
In a competitive market, the supply and demand curves intersect, caused by the interaction of producers and consumers respectively. On the other hand, a firm has a constant price for the demand of a good, hence a horizontal line, given by the equilibrium price in the competitive market. \\~\\
In reality, a perfectly competitive market is impossible to achieve. 


\subsection{What Quantity to Produce?}
\begin{formula} Profit: $$\text{Profit} = \pi = \text{Total Revenue} - \text{Total Cost} $$ \end{formula}
\begin{definition} Total Revenue: price times quantity sold \end{definition}
\begin{formula} Total Revenue: $$TR = P \times Q $$ \end{formula}
\begin{formula} Average Revenue: $$ \text{Average Revenue} = \frac{\text{Total Revenue}}{\text{Quantity}} $$ \end{formula}
\begin{definition} Total Cost: the cost of producing a given quantity of output \end{definition}
\begin{definition} Explicit Cost: a cost that requires a money outlay - out of pocket \end{definition}
\begin{definition} Implicit Cost: a cost that does not require an outlay of money - includes lost opportunity cost, included in economic profit, never in accounting profit \end{definition}
\begin{definition} Economic Profit: total revenue minus total costs including implicit costs (opportunity costs) \end{definition}
\begin{definition} Accounting Profit: total revenue minus explicit costs (out of pocket documented costs) \end{definition}
Note: Economic profits are typically less than accounting profits. Firms want to maximize economic profit, not accounting profit. 
\begin{definition} Fixed Costs: costs that do not vary with output \end{definition}
\begin{definition} Variable Costs: costs that do vary with output \end{definition}
\begin{formula} Total Cost: $$\text{Total Cost} (TC) = \text{Fixed Cost} (FC) + \text{Variable Costs} (VC) $$ \end{formula} 
The total cost of producing 0 goods is the fixed cost. \\~\\
To find the maximum profit, look for the quantity that maximizes TR - TC. 
\begin{definition} Marginal Revenue: the change in total revenue from selling an additional unit \end{definition} 
\begin{formula} Marginal Revenue: $$ MR = \frac{\Delta TR}{\Delta Q} $$ \end{formula}
Note: For a firm in a competitive industry, $MR$ = Price. 
\begin{definition} Marginal Cost: the change in total cost from producing an additional unit; a U-shaped curve \end{definition}
To maximize profit, a firm compares the revenue from selling an additional unit, marginal revenue (for a firm in a competitive industry, this is equal to the price) to the costs of selling an additional unit, marginal cost. Profit increases from an additional sale whenever $MR > MC$ so profit is maximized by producing up until the point where $MR = MC$, or for a firm in a competitive industry, $P = MC$. If $MC > MR$, there's a loss of profit. If $MR > MC$, goods can be produced at this quantity because there is profit but it is not maximized. 

\subsection{Profits and the Average Total Cost Curve}
\begin{definition} Average Total Cost of Production: the cost per good, that is, the total cost of producing $Q$ goods divided by $Q$ \end{definition}
\begin{formula} Average Total Cost of Production: $$ATC = \frac{TC}{Q} = AVC + AFC $$ \end{formula}
\begin{formula} Average Variable Cost of Production: $$AVC = \frac{\text{total } VC}{Q} $$ \end{formula}
\begin{formula} Average Fixed Cost of Production: $$AFC = \frac{\text{total } FC}{Q} $$ \end{formula} 
\begin{formula} Profit: $$\begin{aligned} \text{Profit} &= TR - TC \\ &= (\frac{TR}{Q} - \frac{TC}{Q}) \times Q \\ &= (P - ATC) \times Q \end{aligned} $$ \end{formula}
\noindent Average total cost and average variable cost curve are both U-shaped, as well as marginal cost which goes through the minimum of the average total cost curve. \\ A price below the minimum of the average variable cost curve is the shutdown price. 
\\ The difference between total revenue and total cost is economic profit (abnormal profit). 

\subsection{Entry, Exit and Shutdown Decisions}
A firm is profitable when $P > ATC$ and unprofitable when $P < ATC$. In the long run, firms will enter profitable industries ($P > AC$) and exit unprofitable industries ($P < AC$). At the intermediate point, when $P = AC$, profits are zero and there is neither entry nor exit. 
\begin{definition} Zero Profits (Normal Profits): occurs when $P = AC$; at this price, the firm is covering all of its costs, including enough to pay labor and capital their ordinary opportunity costs \end{definition}
In the long run, a firm will exit an industry if price falls below average variable cost, but exit typically takes some time. A firm may not want to shut down even when $P < AVC$ because shutdown does not immediately eliminate all costs. \\~\\
A firm should exit when $P < AVC$ only if it expects $P$ to remain below $AVC$ for a substantial period of time, and it should enter only if $P > AVC$ and it expects $P$ to stay above $AVC$ for a substantial period. 
\begin{definition} Sunk Cost: a cost that once incurred can never be recovered \end{definition}
It does not always make sense to exit an industry immediately when $P < AC$, or even when $TR < $ Variable Costs. Evaluate fixed/variable costs and revenue/profit between staying and leaving to determine if leaving would be better off. Where do you lose less? 

\subsection{Entry, Exit and Industry Supply Curves}
\begin{definition} Increasing Cost Industry: an industry in which industry costs increase with greater output; shown with an upward sloped supply curve \end{definition}
\begin{definition} Constant Cost Industry: an industry in which industry costs do not change with greater output (increase in supply); shown with a flat supply curve \end{definition}
\begin{definition} Decreasing Cost Industry: an industry in which industry costs decrease with an increase in output; shown with a downward sloped supply curve \end{definition}
If the industry is small relative to its input markets so the industry can expand without pushing up its costs, the supply curve will be flat; this is a constant cost industry. In an increasing cost industry, costs increase with industry output and the supply curve slopes upward. Industry supply curves can even slope downward; this is a decreasing cost industry. But this is rare and temporary. 


\section{Competition and the Invisible Hand} 
\subsection{Invisible Hand Property 1: The Minimization of Total Industry Costs of Production}
\begin{definition} Productive Efficiency: producing goods and services at a minimal cost \end{definition} 
\begin{definition} Allocative Efficiency: producing goods and services that have higher value \end{definition} 
Economic efficiency is the sum of both productive and allocative efficiency. 
In a competitive market with $N$ firms, the following will be true: $$ P = MC_1 = MC_2 = \dots = MC_N $$ where $MC_1$ is the marginal cost of firm 1, $MC_2$ is the marginal cost of firm 2, and so forth. As a result, the total industry costs of production are minimized by the power of the invisible hand. 

If economic profit is positive, firms will enter and more resources will go towards what people value and thus is allocatively efficient. If economic profit is negative, firms will exit and there will be less resources to go towards what people value. 

\subsection{Invisible Hand Property 2: The Balance of Industries}
Entry and exit decisions not only work to eliminate profits and losses, they work to ensure that labor and capital move across industries to optimally balance production so that the greatest use is made of our limited resources.

\subsection{Creative Destruction}
\begin{definition} Elimination Principle: above-normal profits are eliminated by entry and below-normal profits are eliminated by exit \end{definition} 
Above-normal prices are temporary. Great ideas are soon adopted by others; they diffuse throughout the economy and become commonplace and no one profits from the commonplace. Since no one profits from the commonplace to earn above-normal profits, an entrepreneur must innovate. 

\subsection{The Invisible Hand Works with Competitive Markets}
The invisible hand will not work if \begin{itemize} 
\item Prices do not accurately signal costs and benefits - no optimal balance between industries 
\item Markets are not competitive - monopolists and oligopolists produce less than the ideal amount, firms make above normal profits and entry is limited 
\item Commodities are public goods - self interest does not align with social interest \end{itemize}


\section{Monopoly}
\subsection{Market Power}
\begin{definition} Market Power: the power to raise price above marginal cost without fear that other firms will enter the market \end{definition}
\begin{definition} Monopoly: a firm with market power, has no close substitute for its good \end{definition} 
A monopoly can charge at a price greater than $MC$. \\ 
A monopoly has much more power than another if its demand curve is more inelastic (when price is much bigger than $MC$) because if people are not responsive to prices, the monopolist can charge as much as it wants. 

\subsection{How a Firm Uses Market Power to Maximize Profit}
A monopolist decides the price of goods. They face downward facing demand curves. 
A monopolist uses its market power to earn positive or above-normal profits. When the demand curve is a negative sloping line, the marginal revenue curve begins at the same point on the vertical axis as the demand curve and has twice the slope in a downward sloping line. To maximize profit, the monopolist produces when $MR = MC$ and the cost to consumers is the height of the demand curve at the quantity. Total cost is $AC \times Q$ and revenue is $P \times Q$. Therefore economic (abnormal) profit is $(P - AC) \times Q$. Unlike a competitive market, since the monopoly is producing abnormal profits, more firms cannot enter the monopoly. 

\subsection{The Costs of Monopoly: Deadweight Loss}
Monopolies are bad because it reduces total surplus, the total gains from trade (consumer surplus plus producer surplus), creating a deadweight loss. Monopolies decrease consumer surplus (gains of the consumer) and increases gains for the monopoly which creates a deadweight loss. 

\subsection{The Costs of Monopoly: Corruption and Inefficiency} 
\begin{definition} Anti-Trust Laws: laws against monopolies \end{definition} 
Due to higher prices of monopolies, consumers consume less than in a competitive market. \\ 
Monopolies are especially harmful when the goods that are monopolized are used to produce other goods. Each firm wants to raise the price of a good and the resulting cost increases are spread throughout the economy, resulting in various problems. 

\subsection{The Benefits of Monopoly: Incentives for Research and Development}
Patents are one way of rewarding research and development. It's precisely the expectation (and hope) of enjoying the monopoly profit that encourages firms to research and develop new goods. \\~\\
Monopoly may increase economic growth when it increases innovation. 

\subsection{Economies of Scale and the Regulation of Monopoly}
\begin{definition} Economies of Scale: the advantages of large-scale production that reduce average cost as quantity increases \end{definition}
\begin{definition} Natural Monopoly: said to exist when a single firm can supply the entire market at a lower cost than two or more firms \end{definition}
Utility companies are government-regulated (natural) monopolies. \\
When there is no regulation, monopolies charge at the demand price at the quantity $MR = MC$. When the government steps in and regulates the monopoly, $P = MC$, acting as if the monopolist has no monopoly power. Therefore the monopoly charges where $MC = P = D$, which is less than $AC$. In this case, the monopolist should be subsidized by the government.  \newpage

\subsection{Other Sources of Market Power}
Some Sources of Market Power \begin{itemize} 
\item Patents 
\item Laws preventing entry of competitors 
\item Economies of scale 
\item Hard to duplicate inputs 
\item Innovation \end{itemize}
\begin{definition} Barriers to Entry: factors that increase the cost to new firms of entering an industry \end{definition}


\section{Price Discrimination and Pricing Strategy}
\subsection{Price Discrimination}
\begin{definition} Price Discrimination: selling the same product at different prices to different customers \end{definition}
To price discriminate, a market must have monopoly/market power (has price control) and have different demands for different markets.\\
The best examples of price discrimination are: airline tickets, coupons at grocery stores, movie tickets. For airline tickets, a market with inelastic demand will mostly be for business people. A market with elastic demand will be for people who are traveling for leisure. 

\begin{definition} Arbitrage: taking advantage of price differences for the same good in different markets by buying low in one market and selling high in another market \end{definition}
Due to arbitrage, demand will increase in one market and supply will increase in the other market, therefore equating price in some time. Arbitrage makes price discrimination difficult. Those who brought goods at lower prices cannot sell at higher prices. Monopolists should have difference markets with difference demand curves, therefore charging different prices (charge more where demand is inelastic and less when demand is elastic).  \\~\\
Principles of Price Discrimination \begin{itemize} 
\item If the demand curves are different, it is more profitable to set different prices in different markets than a single price that covers all markets 
\item To maximize profits ($MR = MC$), the firm should set a higher price in markets with more inelastic demand 
\item Arbitrage makes it difficult for a firm to set different prices in different markets, thereby reducing the profit from price discrimination \end{itemize} 
To succeed at price discrimination, the monopolist must prevent arbitrage. 


\subsection{Price Discrimination Is Common}
\begin{definition} Perfect Price Discrimination (PPD): each customer is charged his or her maximum willingness to pay (height of demand curve) \end{definition}
The height of the demand curve represents the price as well as marginal revenue. Thus $D = MR$. This is different from a regular monopolist whose marginal revenue is twice the slope of the demand. A consumer pays the maximum he would pays which is greater than the price. Therefore the ``consumer surplus'' is actually profit for the monopolist. There is zero actual consumer surplus since consumers are paying their maximum price. 
A perfectly price-discriminating monopolist produces more output than a single price monopolist. 

\subsection{Is Price Discrimination Bad?} 
Price discrimination is bad if the total output with price discrimination falls or stays the same, but if output increases under price discrimination, then total surplus will usually increase. 

\subsection{Tying and Bundling}
\begin{definition} Tying: occurs when to use one good, the consumer must use a second good that is sold (only) by the same firm; a firm can price discriminate by tying two goods and carefully setting their prices; ex: printer and ink cartridges \end{definition}
\begin{definition} Bundling: requiring that products be bought together in a bundle or package; ex: Microsoft Office Suite \end{definition}
Tying and bundling both make more profit than by selling goods individually. 


\section{Labor Markets}
\subsection{The Demand for Labor and the Marginal Product of Labor}
\begin{definition} Competitive Labor Market: quantity of labor on horizontal axis vs price of labor on vertical axis that shows demand and supply for labor \end{definition} 
Equilibrium of the market determines the wage and how many people employers want to hire. 
Wage is determined from the market and is constant in a firm. \\ 
 If wage is high, demand for labor is high; if wage is low, demand for labor is low. 
The equilibrium wage is the price at the intersection of supply of jobs in the market and demand of jobs in the market. A price floor above the equilibrium is minimum wage. There are more people who want to work than number of jobs available. 
\begin{definition} Marginal Product of Labor (MPL): the increase in a firm's revenues created by hiring an additional laborer; if revenue is prices times quantity, MPL is the change of revenue from hiring one more laborer \end{definition}
Note: The marginal product of labor generally declines as more labor is hired. \\
The MPL curve is an upside down U-shaped curve. It increases at first due to specialization but there can't be too much specialization (law of diminishing returns). The downward sloping piece of the MPL curve is the demand curve for the labor. The point where the wage line and MPL/demand curve meet ($W = MP_L$) is the equilibrium quantity of labor. The upward sloping part of the MPL curve is ignored since the firm will want to hire more laborer at the same wage (goes against maximization of profit). 
If demand increases, then more people will be hired and wage will be higher and productivity increases.

\subsection{Supply of Labor}
There are two different supply of labor, one for the individual and one for the market. The individual supply curve is a backward C curve. As wage increases, a laborer would want to work more and more. At some point, a laborer will want to work less. There is a limit on the number of hours a laborer can work. In general, the slope cannot be forever upward slopping. \\~\\
Substitution effect will dominate income effect when an individual continues to work more and more as wage goes up. Income effect will dominate when wage goes up and the individual decides to work less and have more leisure. 

The supply of labor for the market, in totality, is an upward sloping line because workers in general will want to work more and new workers will enter the labor market. \\

High wages encourage a greater supply of labor. But an individual's labor supply curve need not slope upward throughout its range. \\~\\ The wage and the marginal product of labor will always be very close together. This is because a firm will keep hiring workers so long as the MPL is greater than the wage. 

\subsection{Labor Market Issues}
An advancement in technology increases productivity of American firms and offices which raise the marginal product of labor and therefore the wages of American workers thereby raising the living condition of the worker. Workers in firms in other countries are less productive and have lower wages because they work in a less productive (and less technologically advanced) economy. The American worker gets the benefit of productivity in many other sectors of the American economy. Any factor that increases productivity increases wages. \\
Technology advances will always increase productivity. 
\begin{definition} Human Capital: tools of the mind, the stuff in people's heads that makes them productive \end{definition}
Some workers have higher wages than others because they have more human capital. Human capital is not something we are born with - it is produced by investing time and other resources in education, training and experience.
\begin{definition} Compensating Differential: a difference in wages that offsets differences in working conditions \end{definition}
Similar jobs must have similar compensation packages. Every job has a different combination of wages, benefits, fun, risk and other conditions. Some workers will choose jobs with less risk but lower wages while others will prefer jobs with more risk but higher wages.  \\~\\ 
Reasons for Differences in Wages \begin{itemize} 
\item More productive environment, working for someone more productive - higher wages 
\item Supply of employees in a market - less people willing to work, higher wages 
\item Human Capital - skills and knowledge - if you have more human capital, you have a higher wage 
\item Compensating Differential - a risky/dirty job will pay more than a safe/clean job 
\item Unions - reduces supply of labor by letting only union members work thus raising wage \end{itemize} 


\subsection{How Bad Is Labor Market Discrimination}
\begin{definition} Statistical Discrimination: using information about group averages to make conclusions about individuals; ex: hiring a young male rather a young female thinking the woman will drop out in a few years to take care of a newborn, hiring someone who is well dressed rather than someone who has tattoos, a gun, etc. \end{definition}
\begin{definition} Preference-Based Discrimination: based on a plain, flat-out dislike of some group of people, such as a race, religion or gender \begin{itemize} 
\item Discrimination by employers \item Discrimination by customers \item Discrimination by employees \end{itemize} \end{definition}
Note: Government can pass laws to disallow discrimination but people alway find a way to subtlety bypass it. 

\section{Public Goods and the Tragedy of the Commons}
\subsection{Four Types of Goods}
\begin{definition} Excludable Goods: goods that can be prevented from usage by people who don't pay for the goods, ex: food, clothing \end{definition} 
\begin{definition} Nonexcludable Goods: goods that cannot be prevented from usage by people who don't pay for the goods, ex: listening to an AM radio station, air purifier \end{definition}
\begin{definition} Rival Goods: goods whose usage for a person is reduced by another person who also uses the goods, ex: chair \end{definition} 
\begin{definition} Nonrival Goods: goods whose usage for a person is not reduced by another person who also uses the goods, ex: ceiling light \end{definition}
Four Types of Goods $$ \begin{tabular}{|c|c|c|} \hline
& Excludable & Nonexcludable \\ \hline 
Rival & Private Goods & Common Resources \\ \hline
& Jeans & Tuna in the ocean \\ \hline
& Hamburgers & The environment \\ \hline
& Contact lenses & Public roads \\ \hline
Nonrival & Nonrival Private Goods & Public Goods \\ \hline
& Cable TV & Asteroid deflection \\ \hline
& Wi-Fi- & National defense \\ \hline
& Digital music & Mosquito control \\ \hline \end{tabular} $$ 

\subsection{Private Goods and Public Goods}
\begin{definition} Private Goods: goods that are excludable and rival; ex: ice cream \end{definition}
\begin{definition} Public Goods: goods that are nonexcludable and nonrival; ex: National Defense \end{definition}
\begin{definition} Common Resources: goods that are nonexcludable and rival; ex: tuna \end{definition} 
\begin{definition} Club Goods: goods that are excludable and nonrival; ex: cable television \end{definition} 
\begin{definition} Free Rider: enjoys the benefits of a public good without paying a share of the costs; has to be nonexcludable good, ex: someone enjoying the purifier in a classroom \end{definition}
Market cannot provide for public nonexcludable goods since some people will not pay for it, thus the government steps in and provides us with it. It is paid for by taxes. Government does not provide the good, only provides it using the private sector. \\
An additional person enjoying the benefit of a club good is a free rider. 
\begin{definition} Forced Rider: someone who pays a share of the costs of a public good but who does not enjoy the benefits; ex: being forced to pay for a purifier but not enjoying it \end{definition}


\subsection{Nonrival Private Goods}
\begin{definition} Nonrival Private Goods: goods that are excludable but nonrival \end{definition}

\subsection{Common Resources and the Tragedy of the Commons}
\begin{definition} Common Resources: goods that are nonexcludable but rival \end{definition}
\begin{definition} Tragedy of the Commons: the tendency of any resource that is unowned and hence nonexcludable (also rival) to be overused and under maintained \end{definition}
Saying something belongs to everyone is saying something belongs to nobody because it becomes overused. 
\begin{definition} Command and Controls: regulations by the government to control natural resources such as controlling air quality and population (EPA); causes supply to decrease \end{definition}
If the government makes goods privatized, belong to someone, the tragedy of the commons can be fixed (such as private housing for homeless). The quality of the good will increase. 


\section{Stock Markets and Personal Finance} 
\subsection{Passive vs. Active Investing}
\begin{definition} Passive (Mutual) Fund: pools money from many customers and invests the money in many firms, in return for a management fee \end{definition} 
\begin{definition} Active Funds: mutual funds run by managers who try to pick stocks when and where; are charged higher than average fees \end{definition} 
\begin{definition} Passive Funds: mutual funds that attempt to mimic a broad stock market index such as Dow Jones \end{definition}
Most of the time, passive funds are better than active funds. \\ 
Note: Only very few mutual fund managers can consistently beat the market averages. 
The market cannot be timed; therefore one cannot know when for sure the market will go up or down. One can get lucky from time to time but not all the time. The difficulty of beating the stock market is a tribute to the power of markets and the ability of market prices to reflect information. 
\begin{definition} Efficient Markets Hypothesis: the prices of traded assets, such as stocks and bonds, reflect all publicly available information; unless an investor is trading on inside information, he/she will not systematically outperform the market as a whole over time \end{definition}

\subsection{How to Really Pick Stocks, Seriously}

\begin{definition} Bond: a company owes you some money which is given as interest \end{definition} 
\begin{definition} Stock: a part of a company owned  \end{definition} 
\begin{definition} Dividend: the profit/money the company gives back if you own stock \end{definition} 
\begin{definition} Diversification: reduces risks (dramatic fluctuation in stock) by putting together different stocks so that one compensates for the other; portfolio has no risk \end{definition}
Pick stocks that are moving in the opposite direction to diversify your portfolio, not perfectly done in the market though.  \\
Well-known Stock Indexes: \begin{itemize} \item Dow Jones Industrial Average: auction market (stock goes to person who buys it for the highest price); follows average of 30 companies  \item Standard and Poor's 500 (S\&P 500): auction market; follows average of 500 companies \item NASDAQ Composite Index: dealer market (stocks go to a dealer who then trade to other people); follows average of technology companies \end{itemize} 
\begin{definition} Buy and Hold: buy stocks and then hold them for the long run, regardless of what prices do in the short run \end{definition}
\begin{definition} Compounding: interest on interest, making money grow faster because money accumulates in s shorter range of time; ex: bank gives 5\% at the end of 6 months and then 5\% again at the end of 12 months rather than 10\% at the end of 12 months \end{definition} 
Avoid investments and mutual funds that have high fees or ``loads.'' Index (passive) funds have lower fees because it is not actively managed (thus passive). 
\begin{definition} Rule of 70: when compounding involved, if the rate of return (annual percent increase in value including dividends) of an investment in $x\%$, then the doubling time is $70/x$ years \end{definition} 
The rule of 70 is just a mathematical approximation but it shows that when compounded, small differences in investment returns can have a large effect. In the long run, stocks offer higher returns than bonds. 
\begin{definition} Risk-Return Trade-Off: higher returns come at the price of higher risk \end{definition}
It is better to invest in stocks at a young age than at old age; at old age, it is best to invest into bonds. 

\subsection{Other Benefits and Costs of Stock Markets}
Stocks can give information about companies. Stock market can convert stocks into cash (liquidity). Cash is considered as a liquid asset. Stock markets allow companies to raise funds. If managers are not running the company well, it can give an opportunity for people to buy more stocks from a company so that management can be replaced using investment bankers. 


\section{Consumer Choice}
\subsection{How to Compare Two Goods}
\begin{definition} Utility: economical definition of satisfaction \end{definition} 
The goal of the consumer is to maximize utility. Utility is not measurable but you can say you prefer one good over another good. One good can give more utility than another good. Thus utility is ordinal. \\
Assumption: Utility can be measured. It is cardinal. 
\begin{definition} Marginal Utility ($MU$): the change in utility from consuming an additional unit \end{definition} 
\begin{definition} Law of Diminishing Marginal Utility: each additional unit of a good adds less to utility than the previous unit \end{definition} 
The marginal utility curve is a decreasing curve and can even be zero and then become negative. 
Even if marginal utility is decreasing, Total utility is always increasing. When marginal utility is zero, total utility is maximized. If a good is free, you consume it until marginal utility is zero and thus total utility is maximum. But in reality, not everything is free. 
\begin{definition} Optimal Consumption Rule: to maximize utility, a consumer should allocate spending so that the marginal utility per dollar is equal for all purchases $$\frac{MU_A}{P_A} = \frac{MU_B}{P_B} = \dots = \frac{MU_Z}{P_Z} $$ Alternative formula: $$\frac{MU_A}{MU_B} = \frac{P_A}{P_B} $$ \end{definition} 
If $\frac{MU_A}{P_A} \neq \frac{MU_B}{P_B}$, then utility is not maximized. \\~\\
If the price of an apple is \$2 per pound and price of grape is \$5 per pound in a market. The individual received 600 utility from apple and 900 utility from grape the last time he purchased from the market. Then this individual is not maximizing utility. $$\frac{600}{2} < \frac{900}{5}  \to 300 < 180 $$ The individual received 300 utility from the apple and 180 from grapes. Thus apple gave more utility than grapes. The individual should thus buy more apple since it gives more utility and less grapes. 

\subsection{The Demand Curve}
Suppose that the consumer is currently maximizing utility, so the two-goods version of the optimal consumption rule says $$\frac{MU_A}{P_A} = \frac{MU_B}{P_B} $$ If $P_A$ increases, then consumers will purchase fewer $A$ and more $B$ because $$\frac{MU_A}{P_A} < \frac{MU_B}{P_B} $$ $A$ gives less utility than $B$. By buying more $B$, $MU_B$ becomes smaller. Then $MU_A$ becomes bigger and so the ratio will eventually become equal. 

\subsection{The Budget Constraint} 
\begin{definition} Budget Constraint: shows all the consumption bundles that a consumer can afford given their income and prices \end{definition} 
\begin{definition} Budget Line: a line connecting the maximum amount of two goods one can buy with a certain fixed amount of money \end{definition} 
Note: Each point on the budget line can be chosen as a combination of two goods that a consumer can buy. A point outside the line is out of the consumer's budget. A point inside the line is not maximizing amount of money spent. In fact, there are infinite number of combinations of goods a consumer can buy on the budget line. 
\begin{definition} Relative Price (of $A$): the absolute value of the slope of the budget constraint, $\frac{P_A}{P_B}$ or $\frac{P_{\text{vertical}}}{P_{\text{horizontal}}} $; to get 1 unit of $A$, you give up $\frac{P_A}{P_B}$ worth of $B$  \end{definition} 
Note: $P_A$ and $P_B$ are absolute prices. \\
If the relative price of $A$ increases, it becomes expensive and thus, $B$ becomes cheaper. \\
An increase in income shifts the budget constraint to the right; a decrease in income shifts the budget constraint to the left. \\ A fall in the price of one good rotates the budget constraint inward towards the other axis (increases the relative price) while a rise in the price of one good rotates the budget constraint outward away from the other axis (decreases the relative price). 

\subsection{Preferences and Indifference Curves}
\begin{definition} Indifference Curve: an inward, convex, curve that connects all consumption bundles that give the consumer the same utility, explained by the Law of Diminishing Marginal Utility \end{definition}
Two points on the indifferent curve give the same utility and so a consumer is indifferent to both combinations of two goods. 
\begin{definition} Marginal Rate of Substitution ($MRS$): the rate at which the consumer is willing to trade for another and remain indifferent; is equal to the slope of the indifference curve at that point \end{definition} 
Since indifference curves toward the north-east of the diagram, giving the consumer more utility, the consumer will want to be as far to the north-east as possible. Of two indifference curves on a graph, a consumer would want to be at the point that is most indifferent, or top right. But choosing between two points on a single indifference curve, a consumer is indifferent to either. \\
Indifference curves can never intersect for a single person. \\ 
When the (highest) indifference curve and budget line intersect, the point of tangency will dictate which combination of two goods will maximize utility. \\

Maximize Utility \begin{itemize} 
\item At point of tangency of indifference curve and budget line 
\item Ratio of prices and marginal utilities are equal for two goods
\item Marginal utility to price ratio should be equal for all goods

\end{itemize} 

\subsection{Optimization and Consumer Choice}
A consumer must be on (or inside) the budget constraint and on the indifference curve that is the farthest to the north-east. To find the optimal consumption bundle, look for the consumption bundle that is on the highest indifference curve but still on the budget constraint. \\
At the optimal bundle, the slope of the indifference curve is equal to the curve of the budget constraint which is equal to $MRS$. 

\subsection{The Income and Substitution Effects} 
\begin{definition} Income Effect: the change in consumption caused by the change in purchasing power from a price change \end{definition} 
Income effect says you want to work less when wages go up because you have more money for leisure and so you work less. 
\begin{definition} Substitution Effect: the change in consumption caused by a change in relative prices holding the consumer's utility level constant \end{definition} 
Substitution effect says work more if wages go up because of a greater loss of money if staying home. \\~\\ 
If the price of $A$ drops, the budget line on the axis of $A$ will become farther from the origin (rotate outward). This budget line is now tangent to a new indifference curve which has a new point of tangency, called the final point of tangency where more $A$ is brought. A consumer will buy more $A$ because it's cheaper (substitution effect) but also he will feel he has more purchasing power (income effect). (You have more money and so you think you can buy more because you still have excess money in your budget). \\
To see how much is from each of the effects: put the new budget line tangent to the old indifference curve. The amount of change from the old point of tangency to the new new point of tangency is due to the substitution effect (because it's cheaper). The amount of change from the new new point to the new point of tangency is due to the income effect (because of more purchasing power). The sum of income effect and substitution effect is the total effect. 

\subsection{Applications of Income and Substitution Effects}
Supply of Labor for an Individual: Budget line shows combination of hours committed to leisure and work. The point of tangency to the indifference curve will show how many hours go to leisure and how many to work. If wage increases, the budget line will rotate so that it will be more steeper. There will be a new point of tangency showing there's less hours of leisure and more hours of work. Thus, when wage went up, the supply of labor went up. But eventually the person will work less due to their supply of labor curve. \\
If wage of an individual increases, the individual will think it is more expensive to stay home (higher opportunity cost) and so less leisure and thus work more (this is from substitution effect). But from income effect, the individual is making more money anyways and so wants to stay home and not work. Thus buy more leisure and less work. Therefore one tells an individual to work more and one tells to work less. Income effect dominates when an individual is on the top half of the individual's supply curve because ``he will not want to work too many hours.'' If individual is on the bottom half of his individual's supply curve, he will want to work more and so substitution effect dominates. Therefore different people react differently to wage changes. \\~\\
Welfare Programs and incentive to Work: Budget line shows combination of hours of leisure and hours of work. if government gives money if you don't work (income guarantee) and then cuts part of it if you do work, a part of the budget line will become flat at the end of the line. The point of the tangency with the indifference curve will say that the government will provide all the money and you won't want to work. Thus the welfare program gives an incentive to people to not work. 



\end{document}

























