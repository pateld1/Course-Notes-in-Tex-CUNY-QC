\documentclass[12pt]{article}
\usepackage[letterpaper, portrait, margin=1in]{geometry}
\usepackage{amsmath, amsthm, amssymb, mathrsfs}

\usepackage{fancyhdr}
\pagestyle{fancy}
\fancyhf{}
\lhead{Darshan Patel}
\rhead{Math 223: Differential Equations}
\renewcommand{\footrulewidth}{0.4pt}
\cfoot{\thepage}

\begin{document}

\theoremstyle{definition}
\newtheorem{theorem}{Theorem}[section]
\newtheorem{definition}{Definition}[section]
\newtheorem{example}{Example}[section]

\newcommand{\dy}{\frac{dy}{dt}}
\newcommand{\ddy}{\frac{d^2y}{dt^2}} 
\newcommand{\sumzinf}{\sum_{n = 0}^{\infty}}

\title{Math 223: Differential Equations}
\author{Darshan Patel}
\date{Spring 2017}
\maketitle

\tableofcontents

\section{First Order Differential Equations} 
\subsection{First Order Linear Differential Equations} 
\begin{definition} Differential Equation: an equation that has a function and one or more of its derivative $$ \ddy + y^2\dy + 3y = \cos(t) $$ \end{definition}
Note: The order of a differential equation is the order of the highest derivative of the function $y$ that appears in the equation. 
\begin{definition} First Order Linear Differential Equation: an equation in the form of 
$$\dy = f(t, y)$$ \end{definition}
\begin{definition} First Order Linear Homogeneous Equation: an equation in the form of 
$$\dy + a(t)y = 0$$ \end{definition}
\begin{definition} First Order Linear Nonhomogeneous Equation: an equation in the form of $$\dy + a(t)y = b(t) $$ \end{definition} 
Steps to solving First Order Linear Homogeneous Equations: $$\begin{aligned} 
\dy + a(t)y &= 0 \\ \frac{1}{y}\dy &= -a(t) \\ \frac{d}{dt} \ln|y(t)| &= -a(t) \\ \ln y(t) &= -\int a(t)dt + C_1 \\ |y(t)| &= \exp(-\int a(t)dt + C_1) = C\exp(-\int a(t) dt) \\ |y(t)\exp(\int a(t)dt)| &= C \\ y(t) &= C\exp(-\int a(t)dt) \end{aligned} $$ 
\begin{example} Solve: $\dy + 2ty = 0$. $$a(t) = 2t$$ $$y(t) = C\exp(-\int 2tdt) = C\exp(-t^2) $$ \end{example}
\begin{definition} General Solution: $$y(t) = C\exp(-\int a(t) dt) $$ \end{definition} 
\begin{definition} Initial Condition: $y(t_0) = y_0$ or some other value for a certain differential equation \end{definition} 
To solve an initial condition problem: $$\begin{aligned} \int_{t_0}^t \frac{d}{ds}\ln|y(s)|ds &= -\int_{t_0}^t a(s)ds \\ \ln|y(t)| - \ln|y(t_0)| = \ln|\frac{y(t)}{y(t_0)} &= -\int_{t_0}^t a(s)ds \\ |\frac{y(t)}{y(t_0)}| &= \exp(-\int_{t_0}^t a(s)ds) \\ y(t) &= y_0\exp(-\int_{t_0}^t a(s)ds) \end{aligned} $$ 
\begin{example} Solve: $\dy + (\sin(t))y = 0$, given $y(0) = \frac{3}{2} $. 
$$a(t) = \sin(t) $$ $$y(t) = \frac{3}{2}\exp(-\int_0^t \sin(s)ds) = \frac{3}{2}e^{\cos(t) - 1}$$ \end{example}
\begin{example} Given initial condition $y(0) = 2$ for the equation $\dy + ty = 0$, solve for $y$. $$\begin{aligned} y &= Ce^{-\int tdt} \\ &= Ce^{-\frac{t^2}{2}} \\ &= 2e^{-\frac{t^2}{2}} \end{aligned} $$ \end{example} 
\begin{example} Solve the equation: $\dy + a(t)y = 0$ given $y(0) = y_0$. 
$$ \begin{aligned} \dy &= -a(t)y \\ \frac{1}{y}\dy &= -a(t) \\ \frac{d}{dx}(\ln(y(t))) &= -a(t) \\ \int_{t_0}^t \frac{d}{dt}(\ln(y(t))) dt &= -\int a(t) dt \\ \ln(y)\Big|_{y_0}^y &= -\int_{t_0}^t a(t) dt \\ \ln(y(t)) - \ln(y_0) &= -\int a(t) dt \\ 
\ln(\frac{y}{y_0}) &= -\int_{t_0}^t a(t)dt \\ \frac{y}{y_0} &= e^{-\int_{t_0}^t a(t) dt} \\ y &= y_0e^{-\int_{t_0}^t a(t) dt} \end{aligned} $$ \end{example}
\begin{example} Solve $\dy + \sin(t)y = 0$ where $y(0) = 3$. 
$$y(t) = 3e^{-\int_0^t \sin(t) dt} = 3e^{\cos(0) - \cos(t)} = 3e^{1 - \cos(t)} $$ \end{example} 
\begin{example} Solve $\dy + e^{t^2}y = 0$ where $y(0) = 4$. 
$$y(t) = 4e^{-\int_0^t e^{t^2} dt} $$ \end{example} 


\begin{definition} Integrating Factor: $\mu(t)$, a factor used to solve a nonhomogeneous equation \end{definition} 
Note: For a first order linear nonhomogeneous equation: $$\mu(t) = \exp(\int a(t)dt) $$ 
Thus: $$\begin{aligned} \dy + a(t) &= b(t) \\ \mu(t)\dy + a(t)\mu(t)y &= \mu(t)b(t) \\ \frac{d}{dt} \mu(t)y &= \mu(t)\dy + \frac{d\mu}{dt}y \\ \mu &= \exp(\int a(t)dt) \\ \frac{d}{dt} \mu(t)y &= \mu(t)b(t) \\ \mu(t)y &= \int \mu(t)b(t)dt + C \\ y &= \frac{1}{\mu(t)}(\int \mu(t)b(t)dt + C) \\ &= \exp(-\int a(t)dt)(\int \mu(t)b(t)dt + C) \end{aligned} $$ 
For satisfying an initial condition $y(t_0) = y_0$: $$ \begin{aligned} 
\int_{t_0}^t \frac{d}{dt} \mu(t)ydt &= \int_{t_0}^t \mu(t)b(t)dt \\ \mu(t)y - \mu(t_0)y_0 &= \int_{t_0}^t \mu(s)b(s)ds \\ y &= \frac{1}{\mu(t)}(\mu(t_0)y_0 + \int_{t_0}^t \mu(s)b(s)ds) \end{aligned} $$ 

\begin{example} Solve: $\dy - 2ty = t$. $$ \begin{aligned} \mu(t) &= \exp(\int -2tdt) = e^{-t^2} \\ e^{-t^2}(\dy - 2ty) &= e^{-t^2}(t) \\ \frac{d}{dt} e^{-t^2}y &= te^{-t^2} \\ e^{-t^2}y &= \int te^{-t^2}dt = -\frac{1}{2}\int -2te^{-t^2}dt \\ &= -\frac{1}{2}e^{-t^2} + C \\ y(t) &= -\frac{1}{2} + Ce^{-t^2} \end{aligned} $$ \end{example} 
\begin{example} Solve: $\dy + 2ty = t$, given $y(1) = 2$. $$\begin{aligned} 
\mu(t) &= \exp(\int a(t)dt) = \exp(\int 2tdt) = e^{t^2} \\ \int_1^t \frac{d}{ds}e^{s^2}y(s)ds &= \int_1^t se^{s^2}ds \\ e^{s^2}y(s)\Big|_1^t &= \frac{e^{s^2}}{2}\Big|_{1}^t \\ e^{t^2}y - 2e &= \frac{e^{t^2}}{2} - \frac{e}{2} \\ y &= \frac{1}{2} + \frac{3e}{2}e^{-t^2} = \frac{1}{2} + \frac{3}{2}e^{1 - t^2} \end{aligned} $$ \end{example} 
\begin{example} Solve: $\dy + y = \frac{1}{1 + t^2}$ given $y(2) = 3$. $$ \begin{aligned} \mu(t) &= \exp(\int a(t)dt) = \exp(\int 1dt) = e^t \\ e^t(\dy + y) &= \frac{e^t}{1 + t^2} \\ \frac{d}{dt}e^ty &= \frac{e^t}{1 + t^2} \\ \int_2^t \frac{d}{ds}e^sy(s)ds &= \int_2^t \frac{e^s}{1 + s^2}ds \\ e^ty - 3e^2 &= \int_2^t \frac{e^s}{1 + s^2}ds \\ y &= e^{-t}\Big[3e^2 + \int_2^t \frac{e^s}{1 + s^2}ds\Big] \end{aligned} $$ \end{example}


\subsection{Separation of Variables} 
\begin{definition} Separable Equation: an equation that can be separated $$\dy = \frac{g(t)}{f(y)} $$ \end{definition} 
If $\dy = \frac{g(t)}{f(y)}$, then $$\begin{aligned} f(y)\dy &= g(t) \\ f(y(t))\dy &= g(t) \\ \frac{d}{dt}F(y(t)) &= f(y(t))\frac{d}{dt} \\ \frac{d}{dt}F(y(t)) &= g(t) \\ F(y(t)) &= \int g(t)dt + C \end{aligned} $$ 
\begin{example} Solve: $\dy = \frac{t^2}{y^2}$. $$\begin{aligned} y^2\dy &= t^2 \\ \frac{d}{dt}\frac{y^3(t)}{3} &= t^2 \\ y^3(t) &= t^3 + C \\ y(t) &= (t^3 + C)^{\frac{1}{3}} \end{aligned} $$ \end{example} 
\begin{example} Solve: $e^y\dy - t - t^3 = 0$. $$\begin{aligned} \frac{d}{dt}e^{y(t)} &= t + t^3 \\ e^{y(t)} &= \frac{t^2}{2} + \frac{t^4}{4} + C \\ y(t) &= \ln(\frac{t^2}{2} + \frac{t^4}{4} + C) \end{aligned} $$ \end{example}

If initial condition $y(0) = y_0$ is given for $\dy = \frac{g(t)}{f(y)}$, then $$ \begin{aligned} F(y(t)) - F(y_0) &= \int_{t_0}^t g(s)ds \\ F(y) - F(y_0) &= \int_{y_0}^y f(r)dr \\ \int_{y_0}^y f(r)dr &= \int_{t_0}^t g(s)ds \end{aligned} $$
\begin{example} Solve: $e^y\dy - (t + t^3) = 0$ where $y(1) = 1$. $$ \begin{aligned} 
y &= \ln(\frac{t^2}{2} + \frac{t^4}{4} + C) \\ 1 &= \ln(\frac{3}{4} + C) \\ C &= e - \frac{3}{4} \\ 
y(t) &= \ln(e - \frac{3}{4} + \frac{t^2}{2} + \frac{t^4}{4}) \end{aligned} $$ Using the new method: $$\begin{aligned} e^y\dy &= t + t^3 \\ \int_1^y e^rdr &= \int_1^t (s + s^3)ds \\ e^y - e &= \frac{t^2}{2} + \frac{t^4}{4} - \frac{1}{2} - \frac{1}{4} \\ y(t) &= \ln(e - \frac{3}{4} + \frac{t^2}{2} + \frac{t^4}{4}) \end{aligned} $$ \end{example} 
\begin{example} Solve: $\dy = \frac{1 + t^2}{y^2}$. 
$$\begin{aligned} y^2dy &= (1 + t^2)dt \\ \frac{y^3}{3} &= t + \frac{t^3}{3} + C \end{aligned} $$ \end{example} 
\begin{example} Solve: $\dy = 1 + y^2$ where $y(0) = 0$. 
$$\begin{aligned} \frac{1}{1 + y^2}dy &= dt \\ \int_0^y \frac{1}{1 + y^2}dy &= \int_0^t dt \\ \arctan(y)\big|_0^y &= t \\ \arctan(y) - \arctan(0) &= t \\ \arctan(y) &= t \\ y &= \tan(t) \end{aligned} $$ \end{example}
\begin{example} Solve: $\dy = 1 + y^2$ where $y(0) = 1$. $$\begin{aligned} \int_1^y \frac{dr}{1 + r^2} &= \int_0^t ds \\ \arctan(r)\Big|_1^y &= s\Big|_0^t \\ \arctan(y) - \arctan(1) &= t \\ y &= \tan(t + \frac{\pi}{4}) \end{aligned} $$ \end{example} 
\begin{example} Solve: $y\dy + (1 + y^2)\sin(t) = 0$ where $y(0) = 1$. $$\begin{aligned} \frac{y}{1 + y^2}\dy &= -\sin(t) \\ \int_1^y \frac{rdr}{1 + r^2} &= \int_0^t -\sin(s)ds \\ \frac{1}{2}\ln(1 + y^2) - \frac{1}{2}\ln(2) &= \cos(t) - 1 \\ y(t) &= (2e^{-4\sin^2(\frac{t}{2})} - 1)^{\frac{1}{2}} \end{aligned} $$ \end{example} 
\begin{example} Solve: $\dy = (1 + y)t$, where $y(0) = 1$. 
$$\frac{1}{1 + y}\dy = t $$ This is not allowable since $y(0) = -1$, But it can be seen that $y(t) = -1$ is one solution of this initial condition problem. \end{example}
Note: Consider the initial condition problem $\dy = f(y)g(t)$ where $y(t_0) = y_0$. Then, $y(t) = y_0$ is one solution of this initial condition problem. 
\begin{example} Solve: $(1 + e^y)\dy = \cos(t)$, where $y(\frac{\pi}{2}) = 3$. $$\begin{aligned} \int_3^y (1 + e^r)dr &= \int_{\frac{\pi}{2}}^t \cos(s)ds \\ (r + e^r)\Big|_3^y &= \sin(s)\Big|_{\frac{\pi}{2}}^t \\ y + e^y - 3 - e^3 &= \sin(t) - \sin(\frac{\pi}{2}) \\ y + e^y &= \sin(t) + e^3 + 2 \end{aligned} $$ \end{example} 
\begin{example} Solve: $\dy = -\frac{t}{y}$. $$\begin{aligned} \dy &= -\frac{t}{y} \\ ydy &= -tdt \\ \int ydy &= -tdt \\ \frac{y^2}{2} &= -\frac{t^2}{2} \\ y^2 + t^2 &= c^2 \end{aligned} $$ \end{example} 
Note: The circles $t^2 + y^2 = c^2$ are called solution curves of the differential equation $\dy = -\frac{t}{y}$. 
\begin{example} Solve: $\dy - 2ty = 1$, $y(0) = 1$. $$\begin{aligned} 
\mu(t) &= \exp(\int -2t \, dt) = \exp(-t^2) \\ \int_1^y \frac{d}{dt} \exp(-t^2)y \, dy &= \int_0^t \exp(-t^2) \, dt \\ 
\exp(-t^2)y(t)\Big|_0^y &= \int_0^t \exp(-t^2) \, dt \\ \exp(-t^2)y - 1 &= \int_0^t \exp(-t^2)\, dt \\ e^{-t^2}y &= 1 + \int_0^t e^{-t^2} \, dt \\ y &= e^{t^2} + e^{t^2}\int_0^t e^{-t^2} \, dt \end{aligned} $$ \end{example}
\begin{example} Solve: $\dy = 1 - t + y^2 - ty^2$. $$ \begin{aligned} 
\dy &= 1 - t + y^2 - ty^2 = 1 - t + y^2(1 - t) = (1 - t)(1 + y^2) \\ \int \frac{dy}{1 + y^2}dy &= \int (1 + t^2) \, dt \\ 
\arctan(y) &= t + \frac{t^3}{3} + C \\ y &= \tan(t + \frac{t^3}{3} + C) \end{aligned} $$ \end{example} 
\begin{example} Solve: $\dy = (1 + t)(1 + y)$. $$ \begin{aligned}
\dy &= (1 + t)(1 + y) \\ \int \frac{dy}{1 + y} &= \int (1 + t) \, dt \\ \ln(y + 1) &= t + \frac{t^2}{t} + C \\ y + 1 &= \exp(t + \frac{t^2}{2} + C) \\ y &= \exp(t + \frac{t^2}{2} + C) - 1 \\ &= Ce^{t + \frac{t^2}{2}} - 1 \end{aligned} $$ \end{example} 
\begin{example} Solve: $\dy = (1 + y)t$, $y(0) = -1$. $$ \begin{aligned} 
\dy &= (1 + y)t \\ \int_0^y \frac{1}{1 + y} \, dy &= \int_{-1}^t t \, dt \\ \ln(1 + y)\Big|_0^y &= \frac{t^2}{t}\Big|_0^t \\ \ln(1 + y) - \ln(0) &= \frac{t^2}{2} \\ \ln(\frac{1 + y}{0}) &= \frac{t^2}{2} \\ \frac{1 + y}{0} &= e^{\frac{t^2}{2}} \\ 
1 + y &= 0 \\ y &= -1 \end{aligned} $$ \end{example} 
\begin{example} Solve: $\dy = \frac{2t}{y + yt^2}$, $y(2) = 3$. $$ \begin{aligned} 
\dy &= \frac{2t}{y + yt^2} = \frac{2t}{y(1 + t^2)} \\ \int_3^y y \, dy &= \int_2^t \frac{2t}{t^2} \, dt \\ \frac{y^2}{2}\Big|_3^4 &= \ln(t^2)\Big|_2^t \\ \frac{y^2}{2} - \frac{9}{2} &= \ln(t^2) - \ln(4) \\ y^2 &= 2\ln(\frac{t^2}{4}) + 9 \end{aligned} $$ \end{example} 

\subsection{Population Models} 
\begin{example} Solve: $\frac{dp}{dt} = ap$, $p(t_0) = p_0$, where $p$ is the population. 
$$ P(t) = p_0\exp(\int_{t_0}^t at \, dt) = p_0e^{a(t - t_0)}$$ This is too simple to explain population control. \end{example} 
\begin{definition} Population Model: $$\frac{dp}{dt} = ap, p(t_0) = p_0$$ $$ p(t) = p_0\exp(a(t - t_0))$$  \end{definition} 
\begin{definition} Better Population Model: $$\frac{dp}{dt} = ap - bp^2$$ \end{definition}
\begin{example} Solve $\frac{dp}{dt} = ap - bp^2$ where $p(t_0) = p_0$. $$\begin{aligned} 
\frac{dp}{dt} &= ap - bp^2 \\ \frac{1}{ap - bp^2}\, dp &= dt \\ \int_{p_0}^p \frac{1}{ap - bp^2} \, dp &= \int_{t_0}^t \, dt \\ \frac{1}{ap - bp^2} &= \frac{1}{p(a - bp)} \\ &= \frac{A}{p} + \frac{B}{a - bp} \\ &= \frac{A(a - bp) + B(p)}{p(a - bp)} \\ A(a - bp) + Bp &= 1 \\ A &= \frac{1}{a} \\ B &= \frac{b}{a} \\ \int_{p_0}^p \frac{1}{ap - bp^2} \, dp &= \int_{t_0}^t \, dt \\ \int_{p_0}^p \frac{1}{a}\frac{1}{p} + \frac{b}{a}\frac{1}{a - bp} \, dp &= \int_{t_0}^t \, dt \\ \int_{p_0}^p \frac{1}{p} + \frac{b}{a - bp} \, dp &= a\int_{t_0}^t \, dt \\ \ln(p)\Big|_{p_0}^p + \ln(a - bp)\Big|_p^{p_0} &= a(t - t_0) \\ \ln(p) - \ln(p_0) + \ln(a - bp_0) - \ln(a - bp) &= a(t - t_0) \\ \ln\Big[\frac{p(a - bp_0)}{p_0(a - bp)}\Big] &= a(t - t_0) \\ \frac{p(a - bp_0)}{p_0(a - bp)} &= \exp(a(t - t_0)) \\ p(a - bp_0) &= p_0(a - bp)\exp(a(t - t_0)) \\ &= p_0a\exp(a(t - t_0)) - p_0bp\exp(a(t - t_0)) \\ p(a - bp_0) + bp_0p\exp(a(t - t_0)) &= ap_0\exp(a(t - t_0)) \\ p[(a - bp_0) + bp_0\exp(a(t - t_0))] &= ap_0\exp(a(t - t_0)) \\ p(t) &= \frac{ap_0\exp(a(t - t_0))}{a - bp_0 + bp_0\exp(a(t - t_0))} \cdot \frac{\exp(-a(t - t_0))}{\exp(-a(t - t_0))} \\ &= \frac{ap_0}{bp_0 + (a - bp_0)\exp(-a(t - t_0))} \end{aligned} $$ This is the logistic model of population. \end{example} 
Note: $$\lim_{t \to \infty} p(t) = \frac{ap_0}{bp_0} = \frac{a}{b} $$ \\~\\
\begin{example} Find the derivative of $\frac{dp}{dt} = ap - bp^2$. $$\begin{aligned} \frac{dp}{dt} &= ap - bp^2 \\ \frac{d^2p}{dt^2} &= p(-b\frac{dp}{dt}) + (a - bp)\frac{dp}{dt} \\ &= \frac{dp}{dt}(a - bp - bp) \\ &= (a - 2bp)\frac{dp}{dt} \end{aligned} $$ \end{example} 

\subsection{Exact Equations and why we cannot solve very many differential equations} 
Given the function $\phi(t, y(t))$, note that: $$ \frac{d}{dt} \phi(t, y(t)) = \frac{\partial \phi}{\partial t} + \frac{\partial \phi}{\partial y}\frac{dy}{dt} $$ 
\begin{theorem} Let $M(t, y)$ and $N(t, y)$ be continuous and have continuous partial derivatives with respect to $t$ and $y$ in the rectangle $R$ consisting of those points $(t, y)$ with $a < t < b$ and $c < y < d$. There exists a function $\phi(t, y)$ such that $M(t, y) = \frac{\partial \phi}{\partial t}$ and $N(t, y) = \frac{\partial \phi}{\partial y}$ if and only if $$\frac{\partial M}{\partial y} = \frac{\partial N}{\partial t} $$ in $R$\end{theorem} 
\begin{proof} Observe that $M(t, y) = \frac{\partial \phi}{\partial t}$ for some function $\phi(t, y)$ if and only if $$ \phi(t, y) = \int M(t, y) \, dt + h(y)$$ where $h(y)$ is an arbitrary function of $y$. Then $$\frac{\partial \phi}{\partial y} = \int \frac{\partial M(t, y)}{\partial y} \, dt + h'(y)$$ Therefore, $\frac{\partial \phi}{\partial y}$ will be equal to $N(t, y)$ if and only if $$N(t, y) = \int \frac{\partial M(t, y)}{\partial y} \, dt + h'(y)$$ or $$h'(y) = N(t, y) - \int \frac{\partial M(t, y)}{\partial y} \, dt $$ Now $h'(y)$ is a function of $y$ alone, while the RHS is a function of both $t$ and $y$. But a function of $y$ alone cannot be equal to a function of both $t$ and $y$. Thus the above equation can be true only if the RHS is a function of $y$ alone, and that is if and only if $$ \frac{\partial}{\partial t} \Big[ N(t, y) - \int \frac{\partial M(t, y)}{\partial y} \, dt \Big] = \frac{\partial N}{\partial t} - \frac{\partial M}{\partial y} = 0 $$ Hence, if $\frac{\partial N}{\partial t} \neq \frac{\partial M}{\partial y}$, then there is no function $\phi(t, y)$ such that $M = \frac{\partial \phi}{\partial t}$ and $N = \frac{\partial \phi}{\partial y}$. On the other hand, if $\frac{\partial N}{\partial t} = \frac{\partial M}{\partial y}$, then $$h(y) = \int\Big[N(t, y) - \int \frac{\partial M(t, y)}{\partial y}\, dt\Big]\, dy $$ Consequently, $M = \frac{\partial \phi}{\partial t}$ and $N = \frac{\partial \phi}{\partial y}$ with $$\phi(t, y) = \int M(t, y) \, dt + \int \Big[ N(t, y) - \int \frac{\partial M(t, y)}{\partial y}\, dt \Big] \, dy $$ \end{proof} 
\begin{definition} The differential equation $$M(t, y) + N(t, y)\dy = 0$$ is said to be \textit{exact} if $$ \frac{\partial M}{\partial y} = \frac{\partial N}{\partial t} $$ \end{definition}
Methods to Obtain $\phi(t, y)$: \begin{enumerate} 
\item The equation $M(t, y) = \frac{\partial \phi}{\partial t}$ determines $\phi(t, y)$ up to an arbitrary function of $y$ alone, that is $$\phi(t, y) = \int M(t, y) \, dt + h(y)$$ The function $h(y)$ is then determined from the equation $$h'(y) = N(t, y) = \int \frac{\partial M(t, y)}{\partial y} \, dt $$ 
\item If $N(t, y) = \frac{\partial \phi}{\partial y}$, then, of necessity, $$\phi(t, y) = \int N(t, y) \, dy + k(t) $$ where $k(t)$ is an arbitrary function of $t$ alone. Since $$M(t, y) = \frac{\partial \phi}{\partial t} = \int \frac{\partial N(t, y)}{\partial t} \, dy + k'(t) $$ then $k(t)$ is determined from the equation $$k'(t) = M(t, y) - \int \frac{\partial N(t, y)}{\partial t} \, dy $$ 
\item The equations $\frac{\partial \phi}{\partial t} = M(t, y)$ and $\frac{\partial \phi}{\partial y} = N(t, y)$ imply that $$\phi(t, y) = \int M(t, y) \, dt + h(y)$$ and $$\phi(t, y) = \int N(t, y) \, dy + k(t) $$ Usually, $h(y)$ and $k(t)$ can be determined by inspection. \end{enumerate} 
\begin{example} Solve: $3y + e^t + (3t + \cos y) \dy = 0$. \\ Here, $M(t, y) = 3y + e^t$ and $N(t, y) = 3t + \cos(y)$. This equation is exact since $\frac{\partial M}{\partial y} = 3 = \frac{\partial N}{\partial t} $. Hence there exists a function $\phi(t, y)$ such that $3y + e^t = \frac{\partial \phi}{\partial t}$ and $3t + \cos(y) = \frac{\partial \phi}{\partial y}$. \begin{enumerate} 
\item \textit{First Method} $$\begin{aligned} \phi(t, y) &= e^t + 3ty + h(y) \\ \frac{\partial \phi}{\partial y} &= 3t + h'(y) \\ h'(y) + 3t &= 3t + \cos(y) \end{aligned} $$ Thus $h(y) = \sin(y)$ and $\phi(t, y) = e^t + 3ty + \sin(y)$. The general solution of the differential equation must be left in the form $e^t + 3ty + \sin(y) = c$ since $y$ cannot be found explicitly as a function of $t$. 
\item \textit{Second Method} $$\begin{aligned} \phi(t, y) &= 3ty + \sin(y) + k(t) \\ \frac{\partial \phi}{\partial t} &= 3y + k'(t) \\ 3y + k'(t) &= 3y + e^t \end{aligned} $$ Thus $k(t) = e^t$ and $\phi(t, y) = 3ty + \sin(y) + e^t$. 
\item \textit{Third Method} $$\begin{aligned} \phi(t, y) &= e^t + 3ty + h(y) \\ &= 3ty + \sin(y) + k(t) \end{aligned} $$ Comparing these two expressions for the same function $\phi(t, y)$, it is obvious that $h(y) = \sin(y)$ and $k(t) = e^t$. Hence $$\phi(t, y) = e^t + 3ty + \sin(y) $$ \end{enumerate} \end{example} 
\begin{example} Solve: $3t^2t + 8ty^2 + (t^3 + 8t^2y + 12y^2)\dy = 0$, $y(2) = 1$. \\ Here $M(t, y) = 3t^2y + 8ty^2$ and $N(t, y) = t^3 + 8t^2y + 12y^2$. This equation is exact since $$\frac{\partial M}{\partial y} = 3t^2 + 16ty \text{ and } \frac{\partial N}{\partial t} = 3t^2 + 16ty $$ 
Hence there exists a function $\phi(t, y)$ such that $$\begin{aligned} 3t^2y + 8ty^2 &= \frac{\partial \phi}{\partial t} \\ t^3 + 8t^2y + 12y^2 &= \frac{\partial \phi}{\partial y} \end{aligned} $$ \begin{enumerate} 
\item \textit{First Method} $$\begin{aligned} \phi(t, y) &= t^3y + 4t^2y^2 + h(y) \\ \frac{\partial \phi}{\partial y} &= t^3 + 8t^2y + h'(y) \\ t^3 + 8t^2y + h'(y) &= t^3 + 8t^2y + 12y^2 \end{aligned} $$ Hence $h(y) = 4y^3$ and the general solution of the differential equation is $\phi(t, y) = t^3y + 4t^2y^2 + 4y^3 = c$. Setting $t = 2$ and $y = 1$ in this equation, $c = 28$. Thus, the solution of this initial value problem is $$t^3y + 4t^2y^2 + 4y^3 = 28$$ 
\item \textit{Second Method} $$\begin{aligned} \phi(t, y) &= t^3y + 4t^2y^2 + 4y^3 + k(t) \\ \frac{\partial \phi}{\partial t} &= 3t^2y + 8ty^2 + k'(t) \\ 3t^2y + 8ty^2 + k'(t) &= 3t^2y + 8ty^2 \end{aligned} $$ Thus $k(t) = 0$ and $$\phi(t, y) = t^3y + 4t^2y^2 + 4y^3$$ 
\item \textit{Third Method} $$\begin{aligned} \phi(t, y) &= t^3y + 4t^2y^2 + h(y) \\ &= t^3y + 4t^2y^2 + 4y^3 + k(t) \end{aligned} $$ Comparing these two expressions for the same function $\phi(t, y)$, it is clear that $h(y) = 4y^3$ and $k(t) = 0$. Hence $$\phi(t, y) = t^3y + 4t^2y^2 + 4y^3$$ \end{enumerate} \end{example} 
\begin{example} Solve: $4t^3e^{t + y} + t^4e^{t + y} + 2t + (t^4e^{t + y} + 2y)\dy = 0$, $y(0) = 1$. \\ This equation is exact since $$\frac{\partial}{\partial y}(4t^3e^{t + y} + t^4e^{t + y} + 2t) = (t^4 + 4t^3)e^{t + y} = \frac{\partial}{\partial t}(t^4e^{t + y} + 2y) $$ Hence there exists a function $\phi(t, y)$ such that $$\begin{aligned} 4t^3e^{t + y} + t^4e^{t + y} + 2t &= \frac{\partial \phi}{\partial t} \\ t^4e^{t + y} + 2y &= \frac{\partial \phi}{\partial y} \end{aligned} $$ 
The second method will be used here because it is easier to integrate $t^4e^{t + y} + 2y$ with respect to $y$ than it is to integrate $4t^3e^{t + y} + t^4e^{t + y} + 2t$ with respect to $t$. Thus $$\begin{aligned} \phi(t, y) &= t^4e^{t + y} + y^2 + k(t) \\ \frac{\partial \phi}{\partial t} &= (t^4 + 4t^3)e^{t + y} \\ (t^4 + 4t^3)e^{t + y} + k'(t) &= 4t^3e^{t + y} + t^4e^{t + y} + 2t \end{aligned} $$ Thus $k(t) = t^2$ and the general solution of the differential equation is $\phi(t, y) = t^4e^{t + y} + y^2 + t^2 + c$. Setting $t = 0$ and $y = 1$ in this equation yields $c = 1$. Thus the solution of this initial value problem is $$t^4e^{t + y} + t^2 + y^2 = 1$$ \end{example} 
\begin{example} Solve: $2t\sin y + y^3e^t + (t^2\sin y + 2y^2e^t)\dy = 0 $.
$$\begin{aligned} \frac{\partial M}{\partial y} &= 2t\cos y + 3y^2e^t \\ \frac{\partial N}{\partial t} &= 2t\cos y + 3y^2e^t \\ \frac{\partial \phi(t, y)}{\partial t} &= 2t\sin y + y^3e^t \\ \frac{\partial \phi(t, y)}{\partial y} &= t^2\cos y + 3y^2e^t \\ \phi(t, y) &= t^2\sin y + y^3e^t + h(y) \\ &+ t^2\sin y + y^3e^t + k(t) \\ \phi(t, y) &= t^2\sin y + y^3e^t + c \end{aligned} $$  \end{example}
\begin{example} Solve: $1 + (1 + ty)e^{ty} + (1 + t^2e^{ty})\dy = 0$. $$\begin{aligned} \frac{\partial M}{\partial y} &= (1 + ty)te^{ty} + te^{ty} \\ \frac{\partial N}{\partial t} &= t^2ye^{ty} + 2te^{ty} \\ \frac{\partial \phi{t, y}}{\partial t} &= 1 + (1 + ty)e^{ty} \\ \frac{\partial \phi(t, y)}{\partial y} &= 1 + t^2e^{ty} \\ \phi(t, y) &= \int (1 + t^2e^{ty}) \, dy = y + te^{ty} + k(t) \\ \frac{\partial \phi(t, y)}{\partial t} &= tye^{ty} + e^{ty} + k'(t) = 1 + (1 + ty)e^{ty} \\ k'(t) &\to k(t) = t \\ \phi(t, y) &+ y + te^{ty} + t \end{aligned} $$ \end{example} 
\begin{example} Solve: $y\sec^2 t + \sec t\tan t + (2y + \tan t)\dy = 0$. $$\begin{aligned} \frac{\partial M}{\partial y} &= \sec^2 \\ \frac{\partial N}{\partial t} &= \sec^2 t \\ \frac{\partial \phi(t, y)}{\partial t} &+ y\sec^2 y + \sec t \tan t \\ \frac{\partial \phi(t, y)}{\partial y} &= 2y + \tan t \\ \phi(t, y) &= y\tan t + \sec t + h(y) \\ &= y^2 + y\tan t + k(t) \\ \phi(t, y) &= y\tan t + \sec t + y^2 + c \\ ax^2 + bx + c &= (1)y^2 + (\tan t)y + (\sec t) \\ y &= \frac{-\tan t \pm \sqrt{\tan^2 t - 4(\sec t)}}{2} \\ y(0) &= 1 \\ \phi(t, y) &= 1\tan(0) + \sec(0) + 1^1 = 2 \\ \phi(t, y) &= y\tan t + \sec t + y^2 + 2 \end{aligned} $$ \end{example} 

\section{Second Order Linear Differential Equations} 
\subsection{Algebraic Properties of Solutions} 
\begin{definition} Second Order Linear Homogeneous Differential Equation: $$\ddy + p(t)\dy + q(t)y = 0 $$ \end{definition} 
\begin{definition} Second Order Linear Nonhomogeneous Differential Equation: $$\ddy + p(t)\dy + q(t)y = g(t) $$ \end{definition} 
\begin{theorem} Existence-Uniqueness Theorem: Let the functions $p(t)$ and $q(t)$ be continuous in the open interval $\alpha < t < \beta$. Then, there exists one, and only one function $y(t)$ satisfying the differential equation $$ \ddy + p(t)\dy + q(t)y = 0 $$ on the entire interval $\alpha < t < \beta$, and the prescribed initial conditions $y(t_0) = y_0$, $y'(t_0) = y'_0$. In particular, any solutions $y = y(t)$ of this differential equation which satisfies $y(t_0) = 0$ and $y'(t_0) = 0$ at some time $t = t_0$ must be identically zero. \end{theorem} 
\begin{example} Solve: $\ddy + y = 0$. $$\begin{aligned} y_1 &= \cos t \\ y_2 &= \sin t \\ y &= c_1\cos t + c_2\sin t  \end{aligned} $$ \end{example} 
Let $L[y](t) = \ddy(t) + p(t)y'(t) + q(t)y(t)$ be an operator which operates on the function $y$. 
\begin{example} Let $p(t) = 0$ and $q(t) = t$. Then if $y(t) = \cos t$, $$L[y](t) = (\cos t)'' + t\cos t = (t - 1)\cos t $$ If $y(t) = t^3$, then $$L[y](t) = (t^3)'' + t(t^3) = t^4 + 6t $$ The operator $L$ assigns the function $(t - 1)\cos t$ to the function $\cos t$ and the function $6t + t^4$ to the function $t^3$. \end{example} 
Note: \begin{itemize} 
\item $L[cy](t) = cL[y](t)$ for any constant $c$ 
\item $L[y_1 + y_2](t) = L[y_1](t) + L[y_2](t)$ \end{itemize} 
\begin{definition} An operator $L$ which assigns functions to functions and which satisfies both qualities above is called a linear operator. All other operators are nonlinear. \end{definition} 
In the differential equation $$\ddy + y = 0$$, two solutions were $y_1(t) = \cos t$ and $y_2(t) = \sin t$. Thus $$y(t) = c_1\cos t + c_2 \sin t $$ for every choice of constants $c_1$ and $c_2$. Let $y(0)= y_0$ and $y'(0) = y'_0$ and consider $$\phi(t) = y_0\cos t + y'_0 \sin t $$ This function is a solution to the differential equation since it is a linear combination of solutions. Therefore $$y(t) = y_0 \cos t + y'_0 \sin t $$ must be the general solution of the differential equation. \\~\\
The general solution of a second order linear homogeneous differential equation is $$ y(t) = c_1y_1(t) + c_2y_2(t) $$ 
\begin{theorem} Let $y_1(t)$ and $y_2(t)$ be two solutions to $\ddy + p(t)\dy + q(t)y = 0$ on the interval $\alpha < t < \beta$, with $y_1(t)y'_2(t) - y'_1(t)y_2(t) \neq 0$. Then $$y(t) = c_1y_1(t) + c_2y_2(t)$$ is the general solution. \end{theorem} 
\begin{proof} Let $y(t) = c_1y_1(t) + c_2y_2(t)$ and let $y_0$ and $y'_0$ be the values of $y$ and $y_0$ at a time $t = t_0$. Then the following must be true: $$\begin{aligned} c_1y_1(t_0) + c_2y_2(t_0) &= y_0 \\ c_1y'_1(t_0) = c_2y'_2(t_0) &= y'_0 \end{aligned} $$ Multiplying the first equation by $y'_2(t_0)$ and the first equation by $y'_1(t_0)$ gives the following: $$\begin{aligned} c_1\Big[y_1(t_0)y'_2(t_0) - y'_1(t_0)y_2(t_0)\Big] &= y_0y'_2(t_0) - y'_0y_2(t_0) \\ c_2\Big[y'_1(t_0)y_2(t_0) - y_1(t_0)y'_2(t_0)\Big] &= y_0y'_1(t_0) - y'_0y_1(t_0) \end{aligned} $$ 
Thus $$\begin{aligned} c_1 &= \frac{y_0y'_2(t_0) - y'_0y_2(t_0)}{y_1(t_0)y'_2(t_0) - y'_1(t_0)y_2(t_0)} \\ 
c_2 &= \frac{y'_0y_1(t_0) - y_0y'_1(t_0)}{y_1(t_0)y'_2(t_0) - y'_1(t_0)y_2(t_0)} \end{aligned} $$ Thus the denominator $y_1(t_0)y'_2(t_0) - y'_1(t_0)y_2(t_0) \neq 0$. \end{proof} 
\begin{definition} The quantity $y_1(t)y'_2(t) - y'_1(t)y_2(t)$ is called the Wronskian of $y_1$ and $y_2$ and is denoted by $W(t) = W[y_1, y_2](t)$. \end{definition} 
\begin{theorem} Let $p(t)$ and $q(t)$ be continuous in the interval $\alpha < t < \beta$ and let $y_1(t)$ and $y_2(t)$ be two solutions. Then $W[y_1, y_2](t)$ is either identically zero or is never zero on the interval $\alpha < t < \beta$. \end{theorem} 
\begin{proof} Let $W(t) = y_1(t)y'_2(t) - y'_1(t)y_2(t)$. Then $$\begin{aligned} W'(t) &= y_1(t)\ddy_2(t) + y'_1y'_2(t) - y'_2(t)y'_1(t) - \ddy_1(t)y_2(t) \\ &= y_1(t)\Big[-p(t)y'_2 - q(t)y_2\Big] - y_2\Big[-p(t)y'_1 - q(t)y_1\Big] \\ &= -p(t)y_1y'_2 + p(t)y'_1y_2 \\ \frac{dW}{dt} + p(t)y_1y'_2 - p(t)y'_1y_2 &= 0 \\ \frac{dW}{dt} + p(t)\Big[y_1y'_2 - y'_1y_2\Big] &= 0 \\ \frac{dW}{dt} + p(t)W(t) &= 0 \\ W(t) &= ce^{-\int p(t) \, dt} \end{aligned} $$\end{proof} 
\begin{theorem} Let $y_1(t)$ and $y_2(t)$ be two solutions on the interval $\alpha < t < \beta$ and suppose that $W[y_1, y_2](t_0) = 0$ for some $t_0$ in this interval. Then, one of these solutions is constant multiple of the other. \end{theorem} 
\begin{proof} Suppose $y_1y'_2 - y'_1y_2 - 0$. Then $$\begin{aligned} \frac{y'_2}{y_2} &= \frac{y'_1}{y_1} \\ \frac{d}{dt} \ln y_1 &= \frac{d}{dt} \ln y_2 \\ \ln y_2 - \ln y_1 &= c \\ \ln\Big(\frac{y_2}{y_1}\Big) &= 0 \end{aligned} $$ \end{proof} 
\begin{definition} The functions $y_1(t)$ and $y_2(t)$ are said to be linearly dependent on an interval $I$ if one of these functions is a constant multiple of the other on $I$. The functions $y_1(t)$ and $y_2(t)$ are said to be linearly independent on an interval $I$ if they are not linearly dependent on $I$. \end{definition}
\begin{theorem} Two solutions $y_1(t)$ and $y_2(t)$ are linearly independent on the interval $\alpha < t < \beta$ if and only if their Wronskian is unequal to zero on this interval. Thus, two solutions $y_1(t)$ and $y_2(t)$ form a fundamental set of solutions on the interval $\alpha < t < \beta$ if and only if they are linearly independent on this interval. \end{theorem} 

\subsection{Linear Equations with Constant Coefficients} 
Let $$L[y] = a\ddy + b\dy + c = 0$$ For example, $$L[e^{rt}] = a(e^{er})'' + b(e^{er})' + c(e^{rt}) = (ar^2 + br + c)e^{rt} $$ Then $y(t) = e^{rt}$ is a solution if and only if $$ar^2 + br + c = 0$$ 
\begin{definition} Characteristic Equation: an algebraic equation of degree $n$ upon which depends the solution of a differential equation \end{definition} 
Here, the characteristic equation is $$ar^2 + br + c = 0$$ Its solutions are given by $$r = \frac{-b \pm \sqrt{b^2 - 4ac}}{2a} $$ 
\begin{example} Solve: $\ddy + 5\dy + 4y = 0$. $$\begin{aligned} r^2 + 5r + 4 &= 0 \\ (r + 4)(r + 1) &= 0 \\ r &= -4 \\ r &= -1 \\ y_1 &= e^{-4t} \\ y_2 &= e^{-t} \\ y(t) &= c_1e^{-4t} + c_2e^{-t} \end{aligned} $$
If $y_0(0) = 1$ and $y'_0(0) = 0$, then the solution would follow: $$\begin{aligned} y'(t) &= -4c_1e^{-4t} - c_2e^{-t} \\ c_1 + c_2 &= 0 \\ \begin{bmatrix} 1 & 1 & 1 \\ -4 & -1 & 0 \end{bmatrix} &\to \begin{bmatrix} 1 & 0 & \frac{1}{3} \\ 0 & 1 & \frac{4}{3} \end{bmatrix}  \\ c_1 &= \frac{1}{3} \\ c_2 &= \frac{4}{3} \\ y(t) &= \frac{1}{3}e^{-4t} + \frac{4}{3}e^{-t} \end{aligned} $$ \end{example} 
\begin{example} Solve: $\ddy + y = 0$. $$\begin{aligned} r^2 - 1 &= 0 \\ r &= -1. 1\\ y = c_1e^t + c_2e^{-t} \end{aligned} $$ \end{example}
\begin{example} Solve: $6\ddy - 7\dy+ y = 0$. $$\begin{aligned} 6r^2 - 7r + 1 &= 0 \\ (6r - 1)(r - 1) &= 0 \\ r &= \frac{1}{6}, 1 \\ y(t) &= c_1e^{\frac{1}{6}t} + c_2e^t \end{aligned} $$ \end{example} 
\begin{example} Solve: $\ddy - 3\dy- 4y = 0$. $$\begin{aligned} r^2 - 3r - 4 &= 0 \\ (r - 4)(r + 1) &= 0 \\ r &= -1, 4 \\ y(t) &= c_1e^{4t} + c_2e^{-t} \end{aligned} $$ Let $y(0) = 1$ and $y'(0) = 1$. $$y'(t) = 4c_1e^{4t} - c_2e^{-t} $$ To solve for these initial values, plug in the values and solve for the coefficients. $$\begin{aligned} c_1 + c_2 &= 1 \\ 4c_1 - c_2 &= 1 \end{aligned} $$ $$\begin{pmatrix} 1 & 1 & 1 \\ 4 & -2 & 1 \end{pmatrix} = \begin{pmatrix} 1 & 0 & \frac{1}{2} \\ 0 & 1 & \frac{1}{2} \end{pmatrix} $$ Thus the solution is
$$ y(t) = \frac{1}{2}e^{4t} + \frac{1}{2}e^{-t} $$ \end{example} 
\begin{example} Solve: $7\ddy + 5\dy+ y = 0$ where $y(0) = 0$ and $y'(0) = 1$. $$\begin{aligned} 7r^2 + 5r + 1 &= 0 \\ r &= \frac{-b \pm \sqrt{b^2 - 4ac}}{2a} = \frac{-5 \pm \sqrt{5^2 - 4(7)(-1)}}{14} \\ &= \frac{-5 \pm 4 \sqrt{3}}{14} \\ y(t) &= c_1\exp(\frac{-5 - 4\sqrt{3}}{14}t) + c_2\exp(\frac{-5 + 4\sqrt{3}}{14}t) \\ y'(t) &= \frac{-5 - 4\sqrt{3}}{14}c_1\exp(\frac{-5 - 4\sqrt{3}}{14}t) + \frac{-5 + 4\sqrt{3}}{14}c_2\exp(\frac{-5 - 4\sqrt{3}}{14}t) \end{aligned} $$  $$\begin{aligned} c_1 + c_2 &= 0 \\  c_1 &= -c_2 \\ \frac{-5 - 4\sqrt{3}}{14}c_1 + \frac{-5 + 4\sqrt{3}}{14}c_2 &= 1 \\ -\frac{-5 - 4\sqrt{3}}{14}c_2 + \frac{-5 + 4\sqrt{3}}{14}c_2 &= 1 \\ \frac{8\sqrt{3}}{14}c_2 &= 1 \\ c_2 &= \frac{7}{4\sqrt{3}} \\ c_1 &= -c_2 = -\frac{7}{4\sqrt{3}} \\ y(t) &= -\frac{7}{4\sqrt{3}}\exp(\frac{-5 - 4\sqrt{3}}{14}t) + \frac{7}{4\sqrt{3}}\exp(\frac{-5 + 4\sqrt{3}}{14}t)  \end{aligned} $$ \end{example} 
Note: If $b^2 - 4ac$ is negative, then the characteristic equation $ar^2 + br + c = 0$ has complex roots: $$r_1 = \frac{-b + i\sqrt{4ac - b^2}}{2a} \text{ and } r_2 = \frac{-b - i\sqrt{4ac - b^2}}{2a} $$ where $e^{r_1t}$ and $e^{r_2t}$ are solutions of the differential equation $$a\ddy + b\dy + cy = 0$$ 
Let $y(t) = u(t) + iv(t)$ be a complex-valued solution to this differential equation. This means that $$ a[u''(t) + iv''(t)] + b[u'(t) + iv'(t)] + c[u(t) + iv(t)] = 0$$ 
\begin{theorem} Let $y(t) = u(t) + iv(t)$ be a complex-valued solution to $a\ddy + b\dy + cy = 0$, with $a$, $b$ and $c$ being real. Then $y_1(t) = u(t)$ and $y_2(t) = v(t)$ are two real-valued solutions of the differential equation. In other words, both the real and imaginary parts of a complex valued solution are actual solutions to the differential equation. (The imaginary part of the complex number $\alpha + i\beta$ is $\beta$. Similarly, the imaginary part of the function $u(t) + iv(t)$ is $v(t)$.) \end{theorem} 
Let $r = \alpha + i\beta$, then $$e^{rt} = e^{\alpha t}e^{i \beta t} = e^{\alpha t}(\cos \beta t + i\sin \beta t) $$ Therefore $$y(t) = \exp{\frac{[-b + i\sqrt{4ac - b^2}]t}{2a}} = e^{-\frac{bt}{2a}}[\cos \frac{\sqrt{4ac - b^2}}{2a}t] + i[\sin \frac{\sqrt{4ac - b^2}}{2a}t]$$ The two solutions for the differential equation are $$\begin{aligned} y_1(t) &= e^{-\frac{bt}{2a}}\cos \beta t \\ y_2(t) &= e^{-\frac{bt}{2a}}\sin \beta t \end{aligned} $$ where $\beta = \frac{\sqrt{4ac - b^2}}{2a}$. In addition, the solution for if it's a initial value problem is $$y(t) = e^{-\frac{bt}{2a}}[c_1\cos \beta t + c_2\sin \beta t]$$ 
\begin{example} Solve: $4\ddy + 4\dy + 5y = 0$. $$\begin{aligned} 4r^2 + 4r + 5 &= 0 \\ r_1 &= -\frac{1}{2} + i \\ r_2 &= -\frac{1}{2} - i \\ y(t) &= e^{r_1t} = e^{(-\frac{1}{2} + i)t} = e^{-\frac{t}{2}}\cos t + ie^{-\frac{t}{2}}\sin t \end{aligned} $$ 
The two linearly independent real-valued solutions are $$\begin{aligned} \text{Re}\{e^{r_1t}\} &= e^{-\frac{1}{2}}\cos t \\ \text{Im}\{e^{r_1t}\} &= e^{-\frac{t}{2}} \sin t \end{aligned} $$ \end{example} 
\begin{example} Solve: $\ddy + 2\dy + 4y = 0$ where $y(0) = 1$ and $y'(0) = 1$. $$\begin{aligned} r^2 + 2r + 4 &= 0 \\ r_1 &= -1 + \sqrt{3}i \\ r_2 &= -1 - \sqrt{3}i \\ y(t) &= e^{r_1t} = e^{(-1 + \sqrt{3}i)} = e^{-t}\cos \sqrt{3}t + ie^{-t} \sin \sqrt{3}t \\ &= e^{-t}[c_1\cos \sqrt{3}t + c_2\sin \sqrt{3}t] \\ y(0) &= c_1 + 0c_2 = 1 \\ y'(0) &= -c_1 + \sqrt{3}c_2 = 1 \\ c_1 &= 1,c_2 = \frac{2}{\sqrt{3}} \\ y(t) &= e^{-t}[\cos \sqrt{3}t + \frac{2}{\sqrt{3}}\sin \sqrt{3}t] \end{aligned} $$ \end{example} 
\begin{example} Solve: $\ddy + \dy+ y = 0$. $$\begin{aligned} r^2 + r + 1 &= 0 \\ r &= \frac{-1 \pm \sqrt{1 - 4}}{2} = -\frac{1}{2} \pm \frac{\sqrt{3}}{2}i \\ y(t) &= c_1e^{-\frac{t}{2}}\cos \frac{\sqrt{3}}{2}t + c_2e^{-\frac{t}{2}}\sin\frac{\sqrt{3}}{2}t \end{aligned} $$ \end{example} 
\begin{example} Solve: $\ddy + 2\dy+ 3y = 0$. $$\begin{aligned} r^2 + 2r + 3 &= 0 \\ r &= \frac{-2 \pm \sqrt{4 - 12}}{2} = -1 \pm \sqrt{2}i \\ y(t) &= c_1e^{-t}\cos \sqrt{2}t + c_2e^{-t}\sin \sqrt{2}t \end{aligned} $$ \end{example} 
\begin{example} Solve: $\ddy + 4\dy+ 2y = 0$ where $y(0) = 1$ and $y'(0) = 2$. $$\begin{aligned} r^2 + 4r + 2 &= 0 \\ r&= \frac{-1 \pm \sqrt{1 - 8}}{2} = -\frac{1}{2} \pm \frac{\sqrt{7}}{2}i \\ y &= c_1e^{-\frac{t}{2}}\cos \frac{\sqrt{3}}{2}t + c_2e^{-\frac{t}{2}}\sin \frac{\sqrt{3}}{2}t \\ y(0) &= c_1 + 0c_2 = 1 \\ y'(0) &= -\frac{1}{2}c_1 + \frac{\sqrt{3}}{2}c_2 \\ c_1 &= 1 \\ c_2 &= -\frac{3}{\sqrt{2}} \\ y(t) &= e^{-\frac{t}{2}}\cos \frac{\sqrt{3}}{2}t - \frac{3}{\sqrt{2}}e^{-\frac{t}{2}}\sin \frac{\sqrt{3}}{2}t \end{aligned} $$ \end{example} 
\begin{example} Solve: $\ddy + 2\dy+ 5y = 0$ where $y(0) = 0$ and $y'(0) = 2$. $$\begin{aligned} r^2 + 2r + 5 &= 0 \\ r &= \frac{-2 \pm \sqrt{4 - 20}}{2} = -1 \pm 2i \\ y(t) &= c_1e^{-t}\cos 2t + c_2e^{-t}\sin 2t \\ y(0) &= c_1 + 0 = 0 \\ y'(0) &= 0c_1 + 2c_2 = 2 \\ c_1 &= 0 \\ c_2 &= 1 \\ y(t) &= y_2(t) = e^{-t}\sin 2t \end{aligned} $$ \end{example} 

\subsection{Equal Roots; Reduction of Order} 
\begin{definition} Method of Reduction of Order \\ 
If $b^2 = 4ac$, then the characteristic equation $ar^2 + br + c = 0$ has one real roots $r_1 = r_2 = -\frac{b}{2a}$ and so there is only one solution, $y_1(t) = e^{-\frac{bt}{ac}}$ for the differential equation $a\ddy + b\dy + cy = 0$. To find the other solution, let $y(t) = y_1(t)v(t)$. Then $$\dy = v\frac{dy_1}{dt} + y_1\frac{dv}{dt} $$ and $$\ddy = v\frac{d^2y_1}{dt^2} + 2\frac{dv}{dt}\frac{dy_1}{dt} + y_1\frac{d^2v}{dt^2}$$ Consequently, $$\begin{aligned} L[y] &= v\frac{d^2y_1}{dt^2} + 2\frac{dv}{dt}\frac{dy_1}{dt} + y_1\frac{d^2v}{dt^2} + p(t)\Big[v\frac{dy_1}{dt} + y_1\frac{dv}{dt}\Big] + q(t)vy_1 \\ &= y_1\frac{d^2v}{dt^2} + \Big[2\frac{dy_1}{dt} + p(t)y_1\Big]\frac{dv}{dt} + \Big[\frac{d^2y_1}{dt^2} p(t)\frac{dy_1}{dt} + q(t)y_1\Big]v \\ 
&= y_1\frac{d^2v}{dt^2} + \Big[2\frac{dy_1}{dt} + p(t)y_1\Big]\frac{dv}{dt} \end{aligned} $$ 
since $y_1(t)$ is a solution of $L[y] = 0$. Hence, $y(t) = y_1(t)v(t)$ is a solution of the differential equation $a\ddy + b\dy + cy = 0$ if $v$ satisfies $$y_1\frac{d^2v}{dt^2} + \Big[2\frac{dy_1}{dt} + p(t)y_1\Big]\frac{dv}{dt} = 0$$ Note that this is really a first order linear equation for $\frac{dv}{dt}$. Its solution is $$\begin{aligned} \frac{dv}{dt} &= c\exp\Bigg(-\int \Big[ 2\frac{y'_1(t)}{y_1(t)} + p(t)\Big] \, dt\Bigg) \\ &= c\exp\Bigg(-\int p(t) \, dt\Bigg)\exp\Bigg[2\int \frac{y'_1(t)}{y_1(t)} \, dt\Bigg] \\ &= \frac{c\exp(-\int p(t) \, dt)}{y^2_1(t)} \end{aligned} $$ 
Letting $c = 1$ and $v(t) = \int u(t) \, dt$ where $$u(t) = \frac{\exp(-\int p(t) \, dt)}{y^2_1(t)} $$ Therefore $$y_2(t) = v(t)y_1(t) = y_1(t)\int u(t) \, dt $$ \end{definition} 
\begin{example} Solve: $\ddy + 4\dy + 4y = 0$ where $y(0) = 1$ and $y'(0) = 3$. \\ The characteristic equation $r^2 + 4r + 4 = (r + 2)^2 = 0$ has two equal roots $r_1 = r_2 = 2$. Thus $$\begin{aligned} u &= \frac{e^{-\int 4 \, dt}}{e^{-4t}} = \frac{e^{-4t}}{e^{-4t}} = 1 \\ v &= \int u(t) \, dt = \int 1 \, dt = t \\ y(t) &= c_1e^{-2t} + c_2te^{-2t} \end{aligned} $$ 
Furthermore $$\begin{aligned} y(0) &= c_1 + 0 = 1 \\ y'(0) &= -2c_1 + c_2 = 3 \end{aligned} $$
Therefore $c_1 = 1$ and $c_2 = 5$ and hence $$y(t) = e^{-2t} + 5te^{-2t} = (1 + 5t)e^{-2t} $$ \end{example} 
\begin{example} Solve: $(1 - t^2)\ddy + 2t\dy - 2y = 0$ where $y(0) = 3$ and $y'(0) = -4$ in the interval $-1 < t < 1$. \\ $y_1(t) = t$ is one solution of this differential equation. Use method of reduction of order to find the second solution $y_2(t)$. Therefore divide both sides by $1 - t^2$ to obtain the following $$\ddy + \frac{2t}{1 - t^2}\dy - \frac{2}{1 - t^2}y = 0$$ Hence $$\begin{aligned} u(t) &= \frac{\exp\Big(-\int \frac{2t}{1 - t^2} \, dt\Big)}{y^2_1(t)} = \frac{e^{\ln(1 - t^2)}}{t^2} = \frac{1 - t^2}{t^2} \\ y_2(t) &= t\int \frac{1 - t^2}{t^2} \, dt = -t\Big(\frac{1}{t} + t\Big) = -(1 + t^2) \\ y(t) &= c_1t - c_2(1 + t^2) \end{aligned} $$ 
To find the constants $$\begin{aligned} y(0) &= 0c_1 - c^2 = 3 \\ y'(0) &= c_1 + 0c_2 \end{aligned} $$ 
Hence $c_1 = -4$ and $c_2 = 3$ and $$y(t) = -4t + 3(1 + t^2)$$ \end{example} 
\begin{example} Solve: $9\ddy + 6\dy+ y = 0$ with $y(0) = 1$ and $y'(0) = 0$. $$\begin{aligned} 9r^2 + 6r + 1 &= 0 \\ r &= (3t + 1)^2 \\ t &= -\frac{1}{3} \\ y_1 &= c_1e^{-\frac{t}{3}} \\ y_2 &= c_2te^{-\frac{t}{3}} \\ y &= c_1e^{-\frac{t}{3}} + c_2te^{-\frac{t}{3}} \\ y(0) &= c_1 + 0c_2 = 1 \\ y'(0) &= -\frac{1}{3}c_1 + \frac{1}{3}c_2 = 0 \\ c_1 &= 1 \\ c_2 &= \frac{1}{3} \\ y(t) &= e^{-\frac{t}{3}} + \frac{1}{3}te^{-\frac{t}{3}} \end{aligned} $$ \end{example}

\begin{example} Solve: $\ddy - \frac{2(t + 1)}{t^2 + 2t - 1}\dy+ \frac{2}{t^2 + 2t - 1}y = 0$ where $y_1 = t + 1$. 
$$ \begin{aligned} u &= \frac{\exp(\int \frac{2(t + 1)}{t^2 + 2t - 1} \, dt)}{(1 + t)^2} \\ &= \frac{\exp(\ln (t^2 + 2t - 1))}{(t + 1)^2} \\ &= \frac{t^2 + 2t - 1}{(t + 1)^2} \\ &= \frac{t^2 + 2t + 1 - 2}{(t + 1)^2} \\ &= \frac{(t + 1)^2 - 2}{(t + 1)^2} \\ &= 1 - \frac{2}{(t + 1)^2} \\ v &= \int u \, dt = t + \frac{2}{t + 1} \\ y &= t(t + 1) + 2 = t^2 + t + 2 \end{aligned} $$ \end{example}

\begin{example} Solve: $(1 + t^2)\ddy - 2t\dy+ 2y = 0$ where $y_1 = t$. $$\begin{aligned} 
u &= \frac{\exp(\int \frac{2t}{1 + t^2} \, dt)}{t^2} \\ &= \frac{e^{\ln (1 + t^2)}}{t^2} \\ &= \frac{1 + t^2}{t^2} \\ v &= \int u \, dt = t + \frac{1}{t} \\ y(t) &= t^2 + t + 1 \end{aligned} $$ \end{example} 

\begin{example} Solve: $\ddy + y = t$ where $y(0) = 1$ and $y'(0) = 0$. $$\begin{aligned} 
y &= c_1\cos t + c_2\sin t + t \\ y(0) &= c_1 + 0c_2 + 0 = 1 \\ y'(0) &= 0c_1 + c_2 + 1 = 0 \\ c_1 &= 1 \\ c_2 &= -1 \\ y(t) &= \cos t - \sin t + t \end{aligned} $$ \end{example}

\subsection{The Nonhomogeneous Equation} 
Let $L[y] = \ddy + p(t)\dy + q(t)y = g(t)$ where $p(t)$, $q(t)$ and $g(t)$ are continous on an open interval $\alpha < t < \beta$. 
\begin{theorem} Let $y_1(t)$ and $y_2(t)$ be two linearly independent solutions of the homogeneous equation $$L[y] = \ddy + p(t)\dy + q(t)y = 0$$ and let $\psi(t)$ be any particular solution of the nonhomogeneous equation above. Then, every solution $y(t)$ of the nonhomogeneous equation must be of the form $$y(t) = c_1y_1(t) + c_2y_2(t) + \psi(t)$$ for some choice of constants $c_1. c_2$. \end{theorem} 
\begin{theorem} The difference of any two solutions of the nonhomogeneous equation is a solution of the homogeneous equation. \end{theorem} 
\begin{proof} Let $\psi_1(t)$ and $\psi_2(t)$ be two solutions of the nonhomogeneous equation. By the linearity of $L$: 
$$ L[\psi_1 - \psi_2](t) = L[\psi_1](t) - L[\psi_2](t) = g(t) - g(t) = 0 $$ Therefore $\psi_1(t) - \psi_2(t)$ is a solution of the homogeneous equation.  \end{proof} 
\begin{proof} Let $y(t)$ be any solution of the nonhomogeneous equation. By the above theorem, the function $\phi(t) = y(t) - \psi(t)$ is a solution of the homogeneous equation. But every solution $\phi(t)$ of the homogeneous equation is of the form $\phi(t) = c_1y_1(t) + c_2y_2(t)$ for some choice of constants $c_1, c_2$. Therefore $$y(t) = \phi(t) + \psi(t) = c_1y_1(t) + c_2y_2(t) + \psi(t) $$ \end{proof} 
\begin{example} Solve: $\ddy + y = t$. \\ 
The functions $y_1(t) = \cos t$ and $y_2(t) = \sin t$ are two linearly independent solutions of the homogeneous equation $\ddy + y = 0$. Moreover, $\psi(t) = t$ is an obvious solution of this nonhomogeneous equation. Therefore, $$y(t) = c_1\cos t + c_2\sin t + t $$ \end{example} 
\begin{example} Find the general solution of the equation if the three solutions of a certain second-order nonhomogeneous linear equation are $$ \begin{aligned} \psi_1(t) &= t \\ \psi_2(t) &= t + e^t \\ \psi_3(t) &= 1 + t + e^t \end{aligned} $$ 
The functions $$\psi_2(t) - \psi_1(t) = e^t \text{ and } \psi_3(t) - \psi_2(t) = 1$$ are solutions of the corresponding homogeneous equation. Moreover, these functions are linearly independent. Therefore, every solution $y(t)$ of this equation must be of the form $$ y(t) = c_1e^t + e_2 + t $$ \end{example} 

\subsection{The Method of Variation of Parameters} \
\begin{definition} Method of Variation of Parameters: a technique for finding a particular solution $\psi(t)$ of the nonhomogeneous equation $$L[y] = \ddy + p(t)\dy + q(t)y = g(t)$$ once the solutions of the homogeneous equation $$L[y] = \ddy + p(t)\dy + q(t)y = 0$$ are known, using the knowledge of the solutions of the homogeneous equation to find a solution of the nonhomogeneous equation \end{definition} 
Let $y_1(t)$ and $y_2(t)$ be two linearly independent solutions of the homogeneous equation; then $$\psi(t) = u_1(t)y_1(t) + u_2(t)y_2(t)$$ for some functions $u_1(t)$ and $u_2(t)$. Note that $$\frac{d}{dt} \psi(t) = \frac{d}{dt} [u_1y_1 + u_2y_2] = [u_1y'_1 + u_2y'_2] + [u'_1y_1 + u_2'y_2] $$ If it so happens that $$y_1(t)u'_1(t) + y_2(t)u'_2(t) = 0$$ then $\frac{d^2 \psi}{dt^2}$ and $L[\psi]$ will have no second order derivatives of $u_1$ and $u_2$. Therefore $$\begin{aligned} L[\psi] &= [u_1y'_1 + u_2y'_2]' + p(t)[u_1y'_1 + u_2y'_2] + q(t)[u_1y_1 + u_2y_2] \\ &= u'_1y'_1 + u'_2y'_2 + u_1[\ddy_1 + p(t)y'_1 + q(t)y_1] + u_2[\ddy_2 + p(t)y'_2 + q(t)y_2] \\ &= u'_1y'_1 + u'_2y'_2 \end{aligned}$$ since both $y_1(t)$ and $y_2(t)$ are solutions of the homogeneous equation $L[y] = 0$. Consequently, $\psi(t) = u_1y_1 + u_2y_2$ is a solution of the nonhomogeneous equation if $$\begin{aligned} y_1(t)u'_1(t) + y_2(t)u'_2(t) &= 0 \\ y'_1u'_1(t) + y'_2(t)u'_2(t) &= g(t) \end{aligned} $$ By further manipulations, $$\begin{aligned} [y_1(t)y'_2(t) - y'_1(t)y_2(t)]u'_1(t) &= -g(t)y_2(t) \\ [y_1(t)y'_2(t) - y'_1(t)y_2(t)]u'_2(t) &= g(t)y_1(t) \end{aligned} $$ 
Therefore $$ u'_1(t) = -\frac{g(t)y_2(t)}{W[y_1, y_2](t)} \text{ and } u'_2(t) = \frac{g(t)y_1(t)}{W[y_1, y_2](t)} $$ $u_1(t)$ and $u_2(t)$ are obtained by integrating the RHSs. 
\begin{example} Solve: $\ddy + y = \tan t$ on the interval $-\frac{\pi}{2} < t < \frac{\pi}{2}$. Then find the solution $y(t)$ which satisfies $y(0) = 1$ and $y'(0) = 1$. \\ 
The functions $y_1(t) = \cos t$ and $y_2(t) = \sin t$ are two linearly independent solutions of the homogeneous equation $\ddy + y = 0$ with $$W[y_1, y_2](t) = y_1y'_2 - y'_1y_2 = (\cos t)\cos t - (-\sin t)\sin t = 1$$ Thus $$u'_1 = -\tan t\sin t \text{ and } u'_2(t) = \tan t\cos t$$ 
Therefore $$\begin{aligned} u_1(t) &= -\int \tan t \sin t \, dt \\ &= -\int \frac{\sin^2 t}{\cos t} \, dt \\ &= \int \frac{\cos^2 t - 1}{\cos t} \, dt \\ &= \sin t - \ln | \sec t + \tan t| \\ &= \sin t - \ln(\sec t + \tan t) \end{aligned} $$ in the interval $-\frac{\pi}{2} < t < \frac{\pi}{2} $. Similarly, $$u_2(t) = \int \tan t \cos t \, dt = \int \sin t \, dt = -\cos t $$ 
Consequently $$\psi(t) = \cos t[\sin t - \ln(\sec t + \tan t)] + \sin t(-\cos t) = \cos t \ln(\sec t + \tan t) $$ is a particular solution on the interval $-\frac{\pi}{2} < t < \frac{\pi}{2}$. To solve using the initial conditions, $$y(t) = c_1\cos t + c_2\sin t - \cos t\ln(\sec t + \tan t) $$ for some choice of constants $c_1$ and $c_2$. $$\begin{aligned} y(0) &= c_1 + 0c_2 + 0 = 1 \\ y'(0) &= 0c_1 + c_2 - 1 \end{aligned} $$ Therefore $c_1 = 1$ and $c_2 = 2$ and $$y(t) = \cos t + 2\sin t - \cos t\ln(\sec t + \tan t)$$ \end{example} 

\begin{example} Solve: $\ddy - 4\dy+ 4y = te^{2t}$$$\begin{aligned} 
r^2 - 4r + 4 &= 0 \\ (r - 2)^2 &= 0 \to r = 2 \\ 
y_1 &= e^{2t} \\ y_2 &= te^{2t} \\ 
W &= e^{2t}(e^{2t}(2t + 1)) - 2e^{2t}(te^{2t}) = e^{4t} \\ 
u'_1 &= -\frac{te^{2t}te^{2t}}{e^{4t}} = t^2 \\
u_1 = \int 4t \, dt = 2t^2 \\ 
u'_2 &= \frac{te^{2t}e^{2t}}{e^{4t}} = t \\
u_2 = \int t \, dt = \frac{t^2}{2} \\
\psi(t) &= 2t^2e^{2t} + \frac{t^2}{2}te^{2t} 
\end{aligned} $$ \end{example} 

\begin{example} Solve: $\ddy - 3\dy+ 2y = te^{3t} + 1$
$$\begin{aligned} 
r^2 - 3r + 2 &= 0 \\ (r - 1)(r + 2) &= 0 \to r = -2, 1 \\ 
y_1 &= e^t \\ y_2 &= e^{-2t} \\ 
W &= e^t(-2e^{-2t}) - e^t(e^{-2t}) = -3e^{-t} \\ 
u'_1 &= -\frac{(te^{3t} + 1)e^{-2t}}{-3e^{-t}} = \frac{1}{3}e^{-t}(e^{3t}t + 1) \\
u_1 &= \int \frac{1}{3}e^{-t}(e^{3t}t + 1) \, dt = \frac{1}{12}e^{-t}(e^{3t}(2t - 1) - 4) \\ 
u'_2 &= \frac{(te^{3t} + 1)e^t}{-3e^{-t}} = -\frac{1}{3}e^{2t}(e^{3t}t + 1) \\
u_2 &= \int -\frac{1}{3}e^{2t}(e^{3t}t + 1) \, dt = -\frac{1}{150}e^{2t}(2e^{3t}(5t - 1) + 25) \\ 
\psi(t) &= \frac{1}{12}e^{-t}(e^{3t}(2t - 1) - 4)e^t - \frac{1}{150}e^{2t}(2e^{3t}(5t - 1) + 25)e^{-2t} \end{aligned} $$\end{example} 

\begin{example} Solve: $3\ddy + 4\dy+ y = (\sin t)e^{-t}$
$$\begin{aligned} 
3r^2 + 4r + 1 &= 0 \\ (3r + 1)(r + 1) &= 0 \to r = -\frac{1}{3}, -1 \\
y_1 &= e^{-\frac{t}{3}} \\ y_2 &= e^{-t} \\ 
W &= e^{-\frac{t}{3}}(-e^{-t}) - e^{-t}(-\frac{1}{3}e^{-\frac{t}{3}}) = -\frac{2}{3}e^{-\frac{4t}{3}} \\ 
u'_1 &= -\frac{(\sin t)e^{-t}e^{-t}}{-\frac{2}{3}e^{-\frac{4t}{3}}} = \frac{3}{2}e^{-\frac{2t}{3}}\sin t \\
u_1 &= \int \frac{3}{2}e^{-\frac{2t}{3}}\sin t \, dt = -\frac{9}{26}e^{-\frac{2t}{3}}(2\sin t + 3\cos t) \\ 
u'_2 &= \frac{(\sin t)e^{-t}(e^{-\frac{t}{3}})}{-\frac{2}{3}e^{-\frac{4t}{3}}} = -\frac{3}{2}\sin t \\ 
u_2 &= \int -\frac{3}{2}\sin t \, dt = \frac{3}{2}\cos t \\ 
\psi(t) &= -\frac{9}{26}e^{-\frac{2t}{3}}(2\sin t + 3\cos t)e^{-\frac{t}{3}} + \frac{3}{2}e^{-t}\cos t \end{aligned} $$ \end{example} 

\begin{example} Solve: $\ddy - \frac{2t}{1 + t^2}\dy+ \frac{2}{1 + t^2}y = 1 + t^2$ 
$$\begin{aligned} 
y_1 &= t \\ 
u &= \frac{\exp(-\int -\frac{2t}{1 + t^2} \, dt)}{t^2} = \frac{e^{\ln (1 + t^2)}}{t^2} = \frac{1 + t^2}{t^2} = \frac{1}{t^2} + 1 \\ 
v &= \int \frac{1}{t^2} + 1 \, dt = -\frac{1}{t} + t \\ 
y_2 &= t(-\frac{1}{t} + t) = t^2 - 1 \\ 
W &= t(2t) - (t^2 - 1)1 = t^2 + 1 \\ 
u'_1 &= -\frac{(1 + t^2)(t^2 - 1)}{t^2 + 1} = -(t^2 - 1) = 1 - t^2 \\ 
u_1 &= \int 1 - t^2 \, dt = t - \frac{t^3}{3} \\ 
u'_2 &= \frac{(1 + t^2)t}{t^2 + 1} = t \\ 
u_2 &= \int t \, dt = \frac{t^2}{2} \\ 
\psi(t) &= t(t - \frac{t^3}{3}) + \frac{t^2}{2}(t^2 - 1) \end{aligned} $$ \end{example} 

\begin{example} Solve: $\ddy + 4\dy + 4y = t^{\frac{5}{2}}e^{-2t}$, $y(0) = y'(0) = 0$ $$\begin{aligned} 
r^2 + 4r + 4 &= 0 \\ (r + 2)^2 &= 0 \to r = -2 \\ 
y_1 &= e^{-2t} \\ y_2 &= te^{-2t} \\ W &= e^{-2t}(-2te^{-2t} + e^{-2t}) + te^{-2t}(2e^{-2t}) = e^{-4t} \\ 
u'_1 &= -\frac{t^{\frac{5}{2}}e^{-2t} \cdot te^{-2t}}{e^{-4t}} = -t^{\frac{7}{2}} \\ 
u_1 &= \int -t^{\frac{7}{2}} \, dt = -\frac{2}{9}e^{\frac{9}{2}t} \\
u'_2 &= \frac{t^{\frac{5}{2}}e^{-2t}e^{-2t}}{e^{-4t}} = t^{\frac{5}{2}} \\ 
u_2 &= \int t^{\frac{5}{2}} \, dt = \frac{2}{7}e^{\frac{7}{2}} \\ 
\psi(t) &= -\frac{2}{9}t^{\frac{9}{2}}e^{-2t} + \frac{2}{7}t^{\frac{7}{2}}e^{-2t} = 2t^{\frac{9}{2}}e^{-2t}(\frac{1}{7} - \frac{1}{9}) = \frac{4}{63}t^{\frac{9}{2}}e^{-2t} \\ 
y &= c_1e^{-2t} + c_2te^{-2t} + \frac{4}{63}t^{\frac{9}{2}}e^{-2t} \\ 
c_1 = c^2 &= 0 \\ 
y &= \frac{4}{63}t^{\frac{9}{2}}e^{-2t} \end{aligned} $$ \end{example} 

\begin{example} Solve: $\ddy + p(t)\dy + q(t)y = 0$ given that $y_1 = (1 + t)^2$ $$\begin{aligned} 
\ddy + p(t)\dy + q(t)y &= (1 + t)^2 \\ u' &= \frac{1}{(t + 1)^4} \\ v &= \int \frac{1}{(t + 1)^4} \, dt = \frac{1}{3(1 + t)^3} \\ y_2 &= -\frac{1}{3(1 + t)^3} \cdot (1 + t)^2 = -\frac{1}{3(1 +t)} \\ 
W &= (1 + t)(\frac{1}{3(1 + t)^2}) + (\frac{1}{3(1 + t)})(2(1 + t)) = 1 \\
u'_1 &= ((1 + t)(-\frac{1}{3(1 + t)})) = -\frac{1}{3} \\ 
u_1 &= \int -\frac{1}{3} \, dt = -\frac{1}{3}t \\ 
u'_2 &= ((1 + t)(1 + t)^2) = (1 = t)^3 \\ 
u_2 &= \int (1 + t)^3 \, dt = \frac{1}{4}(1 + t)^4 \\ 
\psi(t) &= -\frac{1}{3}t(1 + t)^2 + \frac{1}{4}(1 + t)^4\frac{1}{(1 + t)} \end{aligned} $$ \end{example} 







\begin{example} Solve: $\ddy + \dy + y = t^2 $ 
$$\begin{aligned} r^2 + r + 1 &= 0 \\ r &= \frac{-1 \pm \sqrt{1 - 4}}{2} = -\frac{1}{2} \pm \frac{\sqrt{3}}{2}i \\ 
\psi(t) &= a_0 + a_1t + a_2t^2 \\ \psi'(t) &= a_1 + 2a_2t \\ \psi''(t) &= 2a_2 \\ 
(a_0 + a_1t + a_2t^2) + (a_1 + 2a_2t)t + a_2t^2 &= t^2 + 0 + 0 \\ 
a_2 &= 1 \\ a_1 + 2a_2 &= 0 \\ a_1 &= -2a_2 = -2 \\ a_0 + a_1t + a_2t^2 &= 0 \\ a_0 &= -(a_1t + a_2t^2) = -(1 + -2) = 1 \\ \psi(t) &= t^2 - 2t \end{aligned} $$ \end{example} 

\begin{example} Solve: $\ddy + 3\dy = t^2 - t$ $$\begin{aligned}
\psi(t) &= a_0 + a_1t + a_2t^2 + a_3t^3 \\ \psi'(t) &= a_1 + 2a_2t + 3a_3t^2 \\
\psi''(t) &= 2a_2 + 6a_3t \\ 
(2a_2 + 3a_1) + (6a_3 + 6a_2)t + 9a_3t^2 &= t^2 - t \\
9a_3 &= 1 \to a_3 = \frac{1}{9} \\
6a_23 + 6a_2 &= -1 \\ a_3 + a_2 &= -\frac{1}{6} \\ 
a_3 &= -\frac{1}{6} - \frac{1}{9} = -\frac{5}{18} \\ 
2a_2 + 3a_1 &= 0 \\
3a_1 &= -2a_2 = \frac{10}{18} = \frac{5}{9} \to a_1 = \frac{5}{27} \\ 
\psi(t) &= \frac{5}{27}t - \frac{1}{6}t - \frac{5}{27}t^3
\end{aligned} $$ \end{example} 

\begin{example} Solve $\ddy + 3y = t^3 - 1$ $$\begin{aligned} 
\psi &= a_0 + a_1t + a_2t^2 + a_3t^3 \\ \psi' &= a_1 + 2a_2t + 3a_3t^2 \\ \psi'' &= 2a_2 + 6a_3t \\ 2a_2 + 6a_3t + 3a_1t + 3a_2t^2 + 3a_3t^3 &= t^3 - 1 \\ 3a_3t^3 &= t^3 \to a_3 = \frac{1}{3} \\ 3a_2t^2 &= 0 \to a_2 = 0 \\ 6a_3t + 3a_1t &= 0 \\ 2a_2 + 3a_0 &= -1 \\ 3a_1t &= -6a_3t = -6(\frac{1}{3}t) = -2t \to a_1 = -\frac{2}{3} \\ 3a_0 &= -1 - 2a_2 = -1 - 2(0) = -1 \\ \psi(t) &= -1 - \frac{2}{3}t + \frac{1}{3}t^3 \end{aligned} $$ \end{example} 

\begin{example} Solve $\ddy - y = t^2e^t$ $$\begin{aligned} y &= e^tv \\ A = 1, ~ B = 0 &, ~C = -1, ~\alpha = 1 \\ v'' + 2v'  &= t^2 \\ v &= a_1t + a_2t^2 + a_3t^3 \\ v' &= a_1 + 2a_2t + 3a_3t^2 \\ v'' &= 2a_2 + 6a_3t \\ (2a_2 + 2a_1) + (6a_3 + 4a_2)t + 6a_3t^2 &= t^2 \\ 6a_3 &= 1 \to a_3 = \frac{1}{6} \\ 1 + 4a_2 &= 0 \to a_2 = -\frac{1}{4} \\ -\frac{1}{2} + 2a_1 &= 0 \to a_1 = \frac{1}{4} \\ \psi(t) &= \frac{1}{4}t - \frac{1}{4}t^2 + \frac{1}{6}t^3 \\ y &= c_1e^t + c_2e^{-t} + \Big[\frac{t^3}{6} - \frac{t^2}{4} + \frac{1}{4}\Big]e^t \end{aligned} $$ \end{example} 

\begin{example} Solve $\ddy + 5\dy + 4y = t^2e^{7t}$ $$\begin{aligned} A = 1, ~ B = 5 &, ~C = 4, ~\alpha = 72 \\ v'' + 19v' + 88v &= t^2 \\ v &= a_0 + a_1t + a_2t^2 \\ v' &= a_1 + 2a_2t \\ v'' &= 2a_2 \\ (2a_2 + 19a_1 + 88a_0) + (38a_2 + 88a_1)t + 88a_2t^2 &= t^2 \\ 88a_2t^2 &= t^2 \to a_2 = \frac{1}{88} \\ 38a_2 + 88a_1 &= 0 \to a_1 = -\frac{38}{(88)^2} \\ 2(\frac{1}{88}) + 19(\frac{-38}{(88)^2}) + 88a_0 &= 0 \to a_0 = \frac{-2(88) + 19(38)}{88^3} \\ \psi(t) &= -\frac{2(88) + 19(38)}{88^3} - \frac{38}{88^2}t + \frac{1}{88}t^2 \end{aligned} $$ \end{example} 

\begin{example} Solve $\ddy + 2\dy + y = e^{-t}$ $$\begin{aligned} r^2 + 2r + 1 &= 0 \to (r + 1)^2 = 0 \to r = -1,~ -1 \\ 
v'' &= 1 \\ v' &= t \\ v &= \frac{t^2}{2} \\ y &= c_1e^{-t} + c_2te^{-t} + \frac{t^2}{2}e^{-t} \end{aligned} $$ \end{example} 

\begin{example} Solve $\ddy + 4y = \sin 2t $ $$\begin{aligned} y'' + 4y &= e^{2it} \\ y &= e^{2it}v \\ A = 1, ~ B = 0 &, ~C = 4, ~\alpha = 2i \\ v'' + 4iv' &= 1 \\ v' &= \frac{1}{4i} = -\frac{i}{4} \\ v &= -\frac{i}{4}t \\ \psi(t) &= -\frac{i}{4}t\Big[\cos 2t + i\sin 2t\Big] = \frac{1}{4}t\sin 2t - \frac{i}{4}t\cos 2t \end{aligned} $$ \end{example}

\begin{example} Solve $\ddy + 4y = t\sin 2t$ $$\begin{aligned} y'' + 4y &= te^{2it} \\ A = 1, ~ B = 0 &, ~C = 4, ~\alpha = 2i \\ y &= e^{2it}v \\ v'' + 4iv' &= t \\ v &= a_1t + a_2t^2 \\ v' &= a_1 + 2a_2t \\ v'' &= 2a_2 \\ 2a_2 + 4i(a_1 + 2a_2t) &= t \\ 8ia_2 &= 1 \to a_2 = \frac{1}{8i} = -\frac{1}{8} \\ 2a_2 + 4ia_1 &= 0 \to -\frac{1}{8} + 2ia_1 = 0 \to 2a_1 = \frac{1}{8} \to a_1 = \frac{1}{16} \\ v &= \frac{1}{16}t - \frac{1}{8}t^2 \\ \psi(t) &= (\frac{1}{16}t - \frac{1}{8}t^2)e^{2it} \\ &= (\frac{1}{16}t - \frac{1}{8}t^2)(\cos 2t + i\sin 2t) \\ &= \frac{1}{8}\cos 2t + \frac{1}{16}t\sin 2t \end{aligned} $$ \end{example} 

\begin{example} Solve $\ddy + \dy + 6y = \sin t$ $$\begin{aligned} y'' + y' - 6y &= e^{it} \\ A = 1, ~ B = 1 &, ~C = -6, ~\alpha = i \\ y &= e^{it}v \\ v'' + (1 + 2i)v' + (-1 + i - 6)v &= 1 \\ v'' + (1 + 2i)v' + (i - 7)v &= 1 \\ v &= \frac{1}{i - 7} = -\frac{7}{50} - \frac{i}{50} \\ \psi(t) &= (-\frac{7}{50} - \frac{i}{50})(\cos t + i\sin t) \\ \text{Im}\{\psi\} &= \frac{1}{50}\cos t - \frac{7}{50}\sin t \end{aligned} $$ \end{example} 






\subsection{Method of Judicious Guessing} 
Let $L[y] = a\ddy + b\dy + cy = a_0 + a_1t + \dots + a_nt^n$. Seek a function $\psi(t)$ such that the three functions $a\psi''$, $b\psi'$ and $c\psi$ add up to a given polynomial of degree $n$. Therefore $$\psi(t) = A_0 + A_1t + \dots + A_nt^n$$ and compute $$\begin{aligned} L[\psi](t) &= a\psi''(t) + b\psi'(t) + c\psi(t) \\ &= a[2A_2 + \dots + n(n - 1)A_nt^{n - 2}] + b[A_1 + \dots + nA_nt^{n - 1}] + c[A_0 + A_1t + \dots + A_nt^n] \\ &= cA_nt^n + (cA_{n - 1} + nbA_n)t^{n - 1} + \dots + (cA_0 + bA_1 + 2aA_2) \end{aligned} $$ 
Equating coefficients of like powers of $t$ in the equation $$L[\psi](t) = a_0 + a_1t + \dots + a_nt^n$$ gives 
$$cA_n = a_n,~ cA_{n - 1} + nbA_n = a_{n - 1}, ~ \dots , ~ cA_0 + bA_1 + 2aA_2 = a_0 $$ 
The first equation determines $A_n = \frac{a_n}{C}$ for $c \neq 0$, and the remaining equations then determine $A_{n - 1}, \dots, A_0$ successively. But when $c = 0$, the first equation has no solution $A_n$. This is expected though, for if $c = 0$, then $L[\psi] = a\psi'' + b\psi'$ is a polynomial of degree $n - 1$, which the RHS is a polynomial of degree $n$. To guarantee that $a\psi'' + b\psi'$ is a polynomial of degree $n$, take $\psi$ as a polynomial of degree $n + 1$ Thus $$\psi(t) = t[A_0 + A_1t + \dots + A_nt^n] $$ The coefficients $A_0, A_1, \dots, A_n$ are determined uniquely from the equation $$a\psi'' + b\psi' = a_0 + a_1t + \dots + a_nt^n$$ if $b \neq 0$. If $b = c = 0$, $$\psi(t) = \frac{1}{a}\Big[ \frac{a_0t^2}{1 \cdot 2} + \frac{a_1t^3}{2 \cdot 3} + \dots + \frac{a_nt^{n + 2}}{(n + 1)(n + 2)}\Big] $$ 
In summary, the differential equation $a\ddy + b\dy + cy = a_0 + a_1t + \dots + a_nt^n$ has a solution $\psi(t)$ of the form $$\psi(t) = \begin{cases} A_0 + A_1t + \dots + A_nt^n ~& c \neq 0 \\ t(A_0 + A_1t + \dots + A_nt^n) ~& c = 0, b\neq 0 \\ t^2(A_0 + A_1t + \dots + A_nt^n) ~& c = b = 0 \end{cases} $$ 

\begin{example} Solve: $L[y] = \ddy + \dy + y = t^2$. \\ Set $\psi(t) = A_0 + A_1t + A_2t^2$ and compute 
$$\begin{aligned} L[\psi](t) &= \psi''(t) + \psi'(t) + \psi(t) \\ &= 2A_2 + (A_1+ 2A_2t) + A_0 + A_1t + A_2t^2 \\ &= (A_0 + A_1 + 2A_2) + (A_1 + 2A_2)t + A_2t^2 \end{aligned} $$ 
Equating coefficient of like powers of $t$ in the equation $L[\psi](t) = t^2$ gives $$\begin{aligned} A_2 &= 1 \\ A_1 + 2A_2 &= 0 \\ A_0 + A_1 + 2A_2 &= 0 \end{aligned} $$ 
Therefore $A_0 = 0$,  $A_1 = -2$ and $A_2 = 1$. Thus $$\psi(t) = -2t + t^2 $$ is a particular solution. \end{example}
Consider the differential equation $$L[y] = a\ddy + b\dy + cy = (a_0 + a_1t + \dots + a_nt^n)e^{\alpha t} $$ 
To remove the factor $e^{\alpha t}$ from the RHS, let $y(t) = e^{\alpha t}v(t)$. Then $$\begin{aligned} y' &= e^{\alpha t}(v' + \alpha v) \\ y'' &= e^{\alpha t}(v'' + 2\alpha v' = \alpha^2v) \end{aligned} $$ so that $$L[y] = e^{\alpha t}\Big[av'' + (2a\alpha + b)v' + (a\alpha^2 + b\alpha + c)v\Big] $$ Therefore, $y(t) = e^{\alpha t}v(t)$ is a solution if $$a\frac{d^2 v}{dt^2} + (2a\alpha + b)\frac{dv}{dt} + (a\alpha^2 + b\alpha + c)v = a_0 + a_1t + \dots + a_nt^n $$ 
To find $v(t)$, examine three cases: \\~\\ 
(i) $a\alpha^2 + b\alpha + c \neq 0$: $\alpha$ is not a root of the characteristic equation $ar^2 + br + c = 0$ and so $e^{\alpha t}$ is not a solution of homogeneous equation $L[y] = 0$. \\

(ii) $a\alpha^2 + b\alpha + c = 0$ but $2a\alpha + b \neq 0$: $\alpha$ is a single root of the characteristic equation, implying that $e^{\alpha t}$ is a solution of the homogeneous equation, but $te^{\alpha t}$ is not. \\
(iii) both $a\alpha^2 + b\alpha + c$ and $2a\alpha + b$ equal 0: $\alpha$ is a double root of the characteristic equation therefore both $e^{\alpha t}$ and $te^{\alpha t}$ are solutions of the homogeneous equation. \\~\\
Hence, the differential equation has a particular solution $\psi(t)$ of the form (i) $\psi(t) = (A_0 + \dots + A_nt^n)e^{\alpha t}$, if $e^{\alpha t}$ is not a solution of the homogeneous equation; (ii) $\psi(t) = t(A_0 + \dots + A_nt^n)e^{\alpha t}$, if $e^{\alpha t}$ is a solution of the homogeneous equation but $te^{\alpha t}$ is not; and (iii) $\psi(t) = t^2(A_0 + \dots + A_nt^n)e^{\alpha t}$ if both $e^{\alpha t}$ and $te^{\alpha t}$ are both solutions of the homogenous equation. \\~\\
There are two ways to compute a particular solution $\psi(t)$. Either make the substitution $y = e^{\alpha t}v$ and find $v(t)$ from $$a\frac{d^2v}{dt^2} + (2a\alpha + b)\frac{dv}{dt} + (a\alpha^2 + b\alpha + c)v = a_0 + a_1t + \dots + a_nt^n$$ or guess a solution $\psi(t)$ of the form $e^{\alpha t}$ times a suitable polynomial in $t$. If $\alpha$ is a double root of the characteristic equation, or if $n \geq 2$, then it is advisable to set $y = e^{\alpha t}v$ and then find $v(t)$ from the above differential equation. Otherwise guess $\psi(t)$ directly. 

\begin{example} Solve: $\ddy - 4\dy + 4y = (1 + t + \dots + t^{27})e^{2t}$. \\
$$\begin{aligned} r^2 - 4r + 4 &= 0 \\ (r - 2)^2 &= 0 \\ r &= 2, 2 \\ y_1(t) &= e^{2t} \\ y_2(t) &= te^{2t} \end{aligned} $$ To find a particular solution $\psi(t)$, set $y = e^{2t}v$. Then $$\frac{d^2v}{dt^2} = 1 + t + t^2 + \dots + t^{27}$$ Integrating this equation twice and setting the constants of integration equal to zero gives $$v(t) = \frac{t^2}{1 \cdot 2} + \frac{t^3}{2 \cdot 3} + \dots + \frac{t^{29}}{28 \cdot 29} $$ Therefore the general solution is $$\begin{aligned} y(t) &= c_1e^{2t} + c_2te^{2t} + e^{2t} \Big[ \frac{t^2}{1 \cdot 2} + \frac{t^3}{2 \cdot 3} + \dots + \frac{t^{29}}{28 \cdot 29}\Big] \\ &= e^{2t}\Big[c_1 + c_2t + \frac{t^2}{1 \cdot 2} + \frac{t^3}{2 \cdot 3} + \dots + \frac{t^{29}}{28 \cdot 29}\Big] \end{aligned} $$ \end{example} 

\begin{theorem} Let $y(t) = u(t) + iv(t)$ be a complex-valued solution of the equation $$L[y] = a\ddy + b\dy + cy = g(t) = g_1(t) + ig_2(t)$$ where $a$, $b$ and $c$ are real. This means that $$a\Big[u''(t) + iv''(t)\Big] + b\Big[u'(t) + v'(t)\Big] + c\Big[u(t) + iv(t)\Big] = g_1(t) + ig_2(t)$$ Then $L[u](t) = g_1(t)$ and $L[v](t) = g_2(t)$. \end{theorem}

\begin{example} Solve $L[y] = \ddy - 3\dy + 2y = (1 + t)e^{3t}$. \\ 
$e^{3t}$ is not a solution of the homogeneous equation $y'' - 3y' + 2y = 0$> Thus set $\psi(t) = (A_0 + A_1t)e^{3t}$. Computing $$\begin{aligned} L[\psi](t) &= \psi'' - 3\psi' + 2\psi \\ &= e^{3t}\Big[(9A_0 + 6A_1 + 9A_1t) - 3(3A_0 + A_1 + 3A_1t) + 2(A_0 + A_1t)\Big] \\ &= e^{3t}\Big[(2A_0 + 3A_1) + 2A_1t\Big] \end{aligned} $$ and cancelling off the factor $e^{3t}$ from both sides of the equation $L[\psi](t) = (1 + t)e^{3t}$ gives $$2A_1t + (2A_0 + 3A_1) = 1 + t $$ This implies that $2A_1 = 1$ and $2A_0 + 3A_1 = 1$. Hence $A_1 = \frac{1}{2}$ and $A_0 = -\frac{1}{4}$. Therefore $\psi(t) = (-\frac{1}{4} + \frac{t}{2})e^{3t}$. \end{example}

Consider the differential equation $$L[y] = a\ddy + b\dy + cy = (a_0 + a_1t + a_2t^2 + \dots + a_nt^n) \times \begin{cases} \cos \omega t \\ \sin \omega t \end{cases} $$ 

\begin{theorem} Let $y(t) = u(t) + iv(t)$ be a complex-valued solution of the equation $$L[y] = a\ddy + b\dy + cy = g(t) = g_1(t) + ig_2(t) $$ where $a$, $b$ and $c$ are real. This means that $$a\Big[ u''(t) + iv''(t)\Big] + b\Big[u'(t) + iv'(t)\Big] + c\Big[u(t) + iv(t)] = g_1(t) + ig_2(t)$$ Then $L[u](t) = g_1(t)$ and $L[v](t) = g_2(t)$. \end{theorem} 

\begin{proof} Equating real and imaginary parts gives $$\begin{aligned} au''(t) + bu'(t) + cu(t) &= g_1(t) \\ av''(t) + bv'(t) + cv(t) &= g_2(t) \end{aligned} $$ Let $\psi(t) = u(t) + iv(t)$ be a particular solution of the equation $$a\ddy + b\dy + cy = (a_0 + a_1t + a_2t^2 + \dots + a_nt^n)e^{i\omega t} $$ The real part of the RHS is $(a_0 + a_1t + a_2t^2 + \dots a_nt^n)\cos \omega t$ while the imaginary part is $(a_0 + a_1t + a_2t^2 + \dots + a_nt^n)\sin \omega t$. Therefore $u(t) = \text{Re}\{\phi(t)\}$ is a solution of $ay'' + by' + cy = (a_0 + a_1t + a_2t^2 + a_nt^n)\cos \omega t$ while $v(t) = \text{Im}\{\phi(t)\}$ is a solution of $ay'' + by' + cy = (a_0 + a_1t + a_2t^2 + \dots + a_nt^n) \sin \omega t$. \end{proof} 

\begin{example} Solve $L[y] = \ddy + 4y = \sin 2t$. \\ Find $\psi(t)$ as the imaginary part of a complex-valued solution $\phi(t)$ of the equation $L[y] = \ddy + 4y = e^{2it}$. Observe that the characteristic equation $r^2 + 4 = 0$ has complex roots $r = \pm 2i$. Thus $$\begin{aligned} \phi(t) &= A_0te^{2it} \\ \phi'(t) &= A_0(1 + 2it)e^{2it} \\ \phi''(t) &= A_0(4i - 4t)e^{2it} \end{aligned} $$ 
Thus $L[\phi](t) = \phi''(t) + 4\phi(t) = 4iA_0e^{2it}$. Hence $A_0 = \frac{1}{4i} = -\frac{i}{4}$ and $$\phi(t) = -\frac{it}{4}e^{2it} = -\frac{it}{4}(\cos 2t + i\sin 2t) = \frac{t}{4}\sin 2t - i\frac{t}{4}\cos 2t$$ 
Therefore $\psi(t) = \text{Im}\{\phi(t)\} = -\frac{t}{4}\cos 2t$ is a particular solution. \end{example} 

\begin{example} Solve $\ddy + 4y = \cos 2t$. \\ From the previous example, $\phi(t) = \frac{t}{4}\sin 2t - i\frac{t}{4}\cos 2t$ is a complexed valued solution. Therefore $$\psi(t) = \text{Re}\{\phi(t)\} = \frac{t}{4}\sin 2t$$ is a particular solution here. \end{example} 

\begin{example} Solve $\ddy + 2\dy + y = te^t\cos t$. \\ Note that $te^t\cos t$ is the real part of $te^{(1 + i)t}$. Therefore find $\psi(t)$ as the real part of the complex valued-solution $\phi(t)$ of the equation $$L[y] = \ddy + 2\dy + y = te^{(1 + i)t}$$ 
Observe that $1 + i$ is not a root of the characteristic equation $r^2 + 2r + 1 = 0$. Therefore, particular solution $\phi(t)$ will be of the form $\phi(t) = (A_0 + A_1t)e^{(1 + i)t}$.  $$\begin{aligned} \phi(t) &= (A_0 + A_1t)e^{(1 + i)t} \\ \phi'(t) &= \Big((1 + i)A_0 + A_1 + (1 + i)A_1t\Big)e^{(1 + i)t} \\ \phi''(t) &= 2\Big(iA_1t + iA_0 + (1 + i)A_1\Big)e^{(1 + i)t} \end{aligned} $$ Computing $L[\phi] = \phi''  + 2\phi' + \phi$ and using the identity $$(1 + i)^2 + 2(1 + i) + 1 = \Big[(1 + i) + 1\Big]^2 = (2 + i)^2$$, see that $$\Big[(2 + i)^2 A_1t + (2 + i)^2A_0 + 2(2 + i)A_1\Big] = t$$ Equating coefficients of like powers of $t$ in this equation gives $$\begin{aligned} (2 + i)^2A_1 &= 1 \\ (2 + I)A_0 + 2A_1 &= 0 \end{aligned} $$ 
This implies that $A_0 = \frac{1}{(2 + i)^2}$ and $A_0 = -\frac{2}{(2 + i)^3}$, so that $$\psi(t) = \Bigg[\frac{-2}{(2 + i)^3} + \frac{t}{(2 + i)^2}\Bigg]e^{(1 + i)t} $$ After a little algebra, $$\psi(t) = \frac{e^t}{125}\Big[(15t - 4)\cos t + (20t - 22)\sin t\Big] + i\Big[(22 - 20t)\cos t + (15t - 4)\sin t\Big] $$ Hence $$\phi(t) = \text{Re}\{\psi(t)\} = \frac{e^t}{125} [(15t - 4)\cos t + (20t - 22)\sin t] $$  \end{example}

\subsection{Series Solutions} 
Let $L[y] = P(t)\ddy + Q(t)\dy + R(t)y = 0$ where $P(t)$ is unequal to zero. Consider the case where $P(t), ~ Q(t), ~R(t)$ are polynomials in $t$. If $y(t)$, is a solution of the differential equation and is a polynomial in $t$, then the three functions $P(t)y''(t)$, $Q(t)y'(t)$ and $R(t)y(t)$ are also polynomials in $t$. Therefore a polynomial solution $y(t)$ of the differential equation can be determined by setting the sums of the coefficients of like powers of $t$ in the expression $L[y](t)$ equal to zero. 

\begin{example} Find two linearly independent solutions of the equation $$L[y] = \ddy - 2t\dy - 2y = 0$$ \\ Try to find two polynomial solutions. It is not obvious what degree of any polynomial solution should be therefore set $$y(t) = a_0 + a_1t + a_2t^2 + a_3t^3 + \dots = \sum_{n = 0}^{\infty} a_nt^n $$ From this, $$\begin{aligned} \dy &= a_1 + 2a_2t + 3a_3t^2 + \dots + \sum_{n = 0}^{\infty} na_nt^{n - 1} \\ \ddy &= 2a_2 + 6a_3t + \dots = \sum_{n = 0}^{\infty} n(n - 1)a_nt^{n - 2} \end{aligned} $$ 
Therefore $y(t)$ is a solution if $$\begin{aligned} L[y](t) &= \sumzinf n(n - 1)a_nt^{n - 2} - 2t\sumzinf na_nt^{n - 1} - 2\sumzinf a_nt^n \\ &= \sumzinf n(n - 1)a_nt^{n - 2} - 2\sumzinf na_nt^n - 2\sumzinf a_nt^n = 0 \end{aligned} $$ 
Rewrite the first summation so that the exponent of the general term is $n$ instead of $n - 2$. This is done by increasing every $n$ underneath the summation sign by 2, and decreasing the lower limit by 2; thus $$\sumzinf n(n - 1)a_nt^{n- 2} = \sum_{n = -2}^{\infty} (n + 2)(n + 1)a_{n + 2}t^n $$ 
Verify this by letting $m = n - 2$. When $n$ is zero, $m = -2$ and when $n$ is infinity, $m$ is also infinity. Therefore $$\sumzinf n(n - 1)a_nt^{n - 2} = \sum_{m = -2}^{\infty} (m + 2)(m + 1)a_{m + 2}t^m $$ Observe that the contribution to this sum from $m = -2$ and $m = -1$ is zero. since the factor $(m + 2)(m + 1)$ vanishes in both these instances. Hence $$\sumzinf n(n - 1)a_nt^{n - 2} = \sumzinf (n + 2)(n + 1)a_{n + 2}t^n$$ Now $$L[y](t) = \sumzinf (n + 2)(n + 1)a_{n + 2}t^n - 2\sumzinf na_nt^m - 2\sumzinf a_nt^n = 0$$ Setting the sum of the coefficients of like powers of $t$ in this equation equal to zero gives $$\begin{aligned}  (n + 2)(n + 1)a_{n + 2} - 2na_n - 2a_n &= 0 \\ (n + 2)(n + 1)a_{n + 2} - 2(n + 1)a_n &= 0 \\ (n + 2)(n + 1)a_{n + 2} &= 2(n + 1)a_n \\ a_{n + 2} &= \frac{2(n + 1)a_n}{(n + 2)(n + 1)} = \frac{2}{n + 2}a_n \end{aligned} $$ 
This is a recurrence formula for the coefficients $a_0, a_1, a_2, a_3, \dots$. The coefficient $a_n$ determines the coefficient $a_{n + 2}$. Thus $$\begin{aligned} a_2 &= \frac{2}{2}a_0  = a_0 \\ a_4 &= \frac{2}{2 + 2}a_2 = \frac{2}{4}a_2 = \frac{1}{2}a_2 \end{aligned} $$ In a similar fashion, $$\begin{aligned} a_3 &= \frac{2}{2 + 1}a_1 = \frac{2}{3}a_1 \\ a_5 &= \frac{2}{3 + 2}a_3 = \frac{2}{5} a_3 = \frac{2}{5} \frac{2}{3}a_1 = \frac{4}{3 \cdot 5}a_1 \end{aligned} $$ 
Conequently, all the coefficients are determined uniquely once $a_0$ and $a_1$ are given. Therefore if $y(t) = a_0 + a_1t + a_2t^2 + a_3t^3 + \dots$, then $a_0 = y(0)$ and $a_1 = y'(0)$. To find two solutions, choose two different sets of values of $a_0$ and $a_1$. The simplest possible choices are $$\begin{aligned} (i) ~ a_0 &= 1, ~ a_1 = 0 \\ (ii) ~ a_0 = 0, ~ a_1 = 1 \end{aligned} $$ In the first case, where $a_0 = 1$ and $a_1 = 0$, all the odd coefficients $a_1, a_3, a_5, \dots$ are zero since $a_3 = \frac{2}{3}a_1 = 0$, $a_5 = \frac{2}{5}a_3 = 0$, and so forth. On the other hand, the even coefficients are determined from the relations $$\begin{aligned} a_2 &= a_0 = 1 \\ a_4 &= \frac{2}{4}a_2 = \frac{1}{2} \\ a_6 &= \frac{2}{6}a_4 = \frac{1}{2 \cdot 3} \\ a_8 &= \frac{2}{8}a_6 = \frac{1}{2 \cdot 3 \cdot 4} \end{aligned} $$ and so forth. Thus $$a_{2n} = \frac{1}{2 \cdot 3 \cdot 4 \dots \cdot n} = \frac{1}{n!} $$ Hence $$y_1(t) = 1 + t^2 + \frac{t^4}{2!} + \frac{t^6}{3!} + \dots = e^{t^2} $$ is the first solution. \\~\\ In the second case, where $a_0 = 0$ and $a_1 = 1$. note that all the even coefficients are zero. On the other hand, the odd coefficients are determined from the following relations $$\begin{aligned} a_3 &= \frac{2}{3}a_1 = \frac{2}{3} \\ a_5 &= \frac{2}{5}a_3 = \frac{2}{5} \cdot \frac{2}{3} \\ a_7 &= \frac{2}{7}a_5 = \frac{2}{7} \cdot \frac{2}{5} \cdot \frac{2}{3} \\ a_9 &= \frac{2}{9}a_5 = \frac{2}{9} \cdot \frac{2}{7} \cdot \frac{2}{5} \cdot \frac{2}{3} \end{aligned} $$ and so forth. Thus $$a_{2n + 1} = \frac{2^n}{3 \cdot 5 \cdot \dots \cdot (2n + 1)} $$ There the second solution is $$y_2(t) = t + \frac{2}{3}t^3 + \frac{2^2}{3 \cdot 5}t^5 + \dots = \sumzinf \frac{2^nt^{2n + 1}}{3 \cdot 5 \cdot \dots \cdot (2n + 1)} $$ \end{example}

Properties of Power Series \begin{enumerate} 
\item An infinite series $$y(t) = a_0 + a_1(t - t_0) + a_2(t - t_0)^2 + \dots = \sumzinf a_n(t - t_0)^n $$ is called a power series about $t = t_0$. 
\item All power series have an interval of convergence. This means that there exists a nonnegative number $\rho$ such that the infinite series converges for $|t - t_0| < \rho$ and diverges for $|t - t_0| > \rho$. The number $\rho$ is called the radius of convergence of the power series. 
\item The power series can be differentiated and integrated term by term, and the resultant series have the same interval of convergence.
\item The simplest method for determining the interval of convergence of the power series is the Cauchy ratio test. Suppose that $$\lim_{n \to \infty} \frac{a_{n + 1}}{a_n} = \lambda$$ Then the power series converges for $|t - t_0| < \frac{1}{\lambda}$ and diverges for $|t - t_0| > \frac{1}{\lambda}$. 
\item The product of two power series $\sumzinf a_n (t - t_0)^n$ and $\sumzinf b_n(t - t_0)^n$ is again a power series of the form $\sumzinf c_n(t - t_0)^n$ with $c_n = a_0b_n + a_1b_{n - 1} + \dots + a_nb_0$. The quotient $$\frac{a_0 + a_1t + a_2t^2 + \dots}{b_0 + b_1t + b_2t^2 + \dots} $$ of two power series is again a power series, provided that $b_0 \neq 0$. 
\item Many of the functions $f(t)$ that arise in applications can be expanded in power series; that is, we can find coefficients $a_0, a_1, a_2, \dots$ so that $$f(t) = a_0 + a_1(t - t_0) + a_2(t - t_0)^2 + \dots = \sumzinf a_n(t - t_0)^n$$ 
Such functions are said to be analytic at $t = t_0$, and the series is called the Taylor series of $f$ about $t = t_0$. It can be easily shown that if $f$ admits such an expansion, then, of necessity, $a_n = \frac{f^{(n)}(t_0)}{n!}$ where $f^{(n)}(t) = \frac{d^nf(t)}{dt^n}$. 
\item The interval of convergence of the Taylor series of a function $f(t)$, about $t_0$, can be determined directly through the Cauchy ratio test and other similar methods, or indirectly, through the following theorem of complex analysis. \end{enumerate} 

\begin{theorem} Let the variable $t$ assume complex values and let $z_0$ be the point closest to $t_0$ at which $f$ or one of its derivatives fails to exist. Compute the distance $\rho$, in the complex plane, between $t_0$ and $z_0$. then the Taylor series of $f$ converges for $|t - t_0| < \rho$ and diverges for $|t - t_0| > \rho$. \end{theorem} 

\begin{theorem} Let the functions $\frac{Q(t)}{P(t)}$ and $\frac{R(t)}{P(t)}$ have convergent Taylor series expansions about $t = t_0$, for $| t - t_0| < \rho$. Then every solution $y(t)$ of the differential equation $$P(t)\ddy + Q(t)\dy + R(t)y = 0$$ is analytic at $t = t_0$ and the radius of convergence of its Taylor series expansion about $t = t_0$ is at least $\rho$. The coefficients $a_2, a_3, \dots, $ in the Taylor series expansion $$y(t) = a_0 + a_1(t - t_0) + a_2(t - t_0)^2 + \dots $$ are determined by plugging the series into the differential equation and setting the sum of the coefficients of like powers of $t$ in this expression equal to zero. \end{theorem} 
Note: The interval of convergence of the Taylor series expansion of any solution $y(t)$ is determined, usually, by the interval of convergence of the power series $\frac{Q(t)}{P(t)}$ and $\frac{R(t)}{P(t)}$, rather than by the interval of convergence of the power series $P(t)$, $Q(t)$, and $R(t)$. This is because the differential equation must be put in the standard form $$\ddy + p(t)\dy + q(t)y = 0$$ whenever we examine questions of existence and uniqueness. 

\begin{example} Find two linearly independent solutions of $$L[y] = \ddy + \frac{3t}{1 + t^2}\dy + \frac{1}{1 + t^2}y = 0 $$ In addition, find the solution $y(t)$ which satisfies the initial conditions $y(0) = 2$, $y'(0) = 3$. \\
The right way to do this problem is to multiply both sides by $1 + t^2$ to obtain the equivalent equation $$L[y] = (1 + t^2)\ddy + 3t\dy + y = 0 $$ Setting up $$\begin{aligned} y(t) &= \sumzinf a_nt^n \\ y'(t) &= \sumzinf na_nt^{n - 1} \\ y''(t) &= \sumzinf n(n - 1)a_nt^{n - 2} \end{aligned} $$ Plug these series into the differential equation. $$\begin{aligned} L[y](t) &= (1 + t^2)\sumzinf n(n - 1)a_nt^{n - 2} + 3t \sumzinf na_nt^{n - 1} + \sumzinf a_nt^n \\ &= \sumzinf n(n + 1)a_nt^{n - 2} + \sumzinf [n(n - 1) + 3n + 1]a_nt^n \\ &= \sumzinf (n + 2)(n + 1)a_{n + 2}t^n + \sumzinf (n + 1)^2a_nt^n \end{aligned} $$ Setting the sum of the coefficients of like powers of $t$ equal to zero gives $$(n + 2)(n + 1)a_{n + 2} + (n + 1)^2a_n = 0 $$ or $$a_{n + 2} = -\frac{(n + 1)^2a_n}{(n + 2)(n + 1)} = -\frac{n + 1}{n + 2}a_n $$ This is a recurrence formula for the coefficients $a_2, a_3, \dots$ in terms of $a_0$ and $a_1$. To find two linearly independent solutions, choose the two simplest cases (1) $a_0 = 1,~ a_1 = 0$ and (2) $a_0 = 0;~ a_1 = 1$. \\~\\ In the first case where $a_0 = 1$ and $a_1 = 0$, all the odd coefficients are zero since $a_3 = -\frac{2}{3}a_1 = 0$, $a_5 = -\frac{4}{5}a_3 = 0$, and so on. The even coefficients are determined from the relations $$\begin{aligned} a_2 &= -\frac{1}{2}a_0 = -\frac{1}{2} \\ a_4 &= -\frac{3}{4}a_2 = \frac{1 \cdot 3}{2 \cdot 4} \\ a_6 &= -\frac{5}{6}a_4 = -\frac{1 \cdot 3 \cdot 5}{2 \cdot 4 \cdot 6} \\ a_8 &= \frac{7}{8}a_6 = \frac{1 \cdot 3 \cdot 5 \cdot 7}{2 \cdot 4 \cdot 6 \cdot 8} \end{aligned} $$ and so forth. Proceeding inductively, we find that $$a_{2n} = (-1)^n \frac{1 \cdot 3 \cdot \dots \cdot (2n -1)}{2 \cdot 4 \cdot \dots \cdot 2n} = (-1)^n \frac{1 \cdot 3 \cdot \dots \cdot (2n - 1)}{2^nn!} $$ Thus $$y_1(t) = 1 - \frac{t^2}{2} + \frac{1 \cdot 3}{2 \cdot 4}t^4 + \dots = \sumzinf (-1)^n\frac{1 \cdot 3 \cdot \dots \cdot (2n - 1)}{2^nn!}t^{2n} $$ The ratio of the $(n + 1)$ term to the $n$th term of $y_1(t)$ is $$-\frac{1 \cdot 3 \cdot \dots \cdot (2n - 1)(2n + 1)t^{2n + 2}}{2^{n + 1}(n + 1)!} \times \frac{2^nn!}{1 \cdot 3 \cdot \dots \cdot (2n - 1)t^{2n}} = \frac{-(2n + 1)t^2}{2(n + 1)} $$ and the absolute value of this quantity approaches $t^2$ as $n$ approaches infinity. Hence by the Cauchy ratio test, the infinite series $y_1(t)$ converges for $|t| < 1$ and diverges for $|t| > 1$. \\~\\ In the second case where $a_0 = 0$ and $a_1 = 1$, all the even coefficients are zero and the odd coefficients are determined from the relations $$\begin{aligned} a_3 &= -\frac{2}{3}a_1 = -\frac{2}{3} \\ a_5 &= -\frac{4}{5}a_3 = \frac{2 \cdot 4}{3 \cdot 5} \\ a_7 &= -\frac{6}{7}a_5 = -\frac{2 \cdot 4 \cdot 6}{3 \cdot 5 \cdot 7} \\ a_9 &= -\frac{8}{9}a_7 = \frac{2 \cdot 4 \cdot 6 \cdot 8}{3 \cdot 5 \cdot 7 \cdot 9} \end{aligned} $$ and so on. Proceeding inductively, we find that $$a_{2n + 1} = (-1)^n \frac{2 \cdot 4 \cdot \dots \cdot 2n}{3 \cdot 5 \cdot \dots \cdot (2n + 1)} = \frac{(-1)^n2^nn!}{3 \cdot 5 \cdot \dots \cdot (2n + 1)} $$ Thus $$y_2(t) = t - \frac{2}{3}t^3 + \frac{2 \cdot 4}{3 \cdot 5}t^5 + \dots = \sumzinf \frac{(-1)^n2^nn!}{3 \cdot 5 \cdot \dots \cdot (2n + 1)}t^{2n + 1} $$ is a second solution. It is easily verified that this solution, too, converges for $|t| < 1$ and diverges for $|t| > 1$. This is not every surprising since the Taylor series expansions about $t = 0$ of the functions $\frac{3t}{1 + t^2}$ and $\frac{1}{1 + t^2} $ only converge for $|t| < 1$. \\~\\ The solutions $y_1(t)$ satisfies the initial conditions $y(0) = 1$ and $y'(0) = 0$ while $y_2(t)$ satisfies the initial conditions $y(0) = 0$ and $y'(0) = 1$. Hence $$ y(t) = 2y_1(t) + 3y_2(t) $$ \end{example} 

\begin{example} Solve $ L[y](t) = \ddy + t\dy + y = 0$. $$\begin{aligned} y &= \sumzinf a_nt^n \\ y' &= \sumzinf na_nt^{n - 1} \\ y'' &= \sumzinf n(n - 1)a_nt^{n-  2} \\ L[y](t) &= \sumzinf n(n - 1)a_nt^{n - 2} + \sumzinf na_nt^n + \sumzinf a_nt^n = 0 \\ \sum_{n = -2}^{\infty} (n + 2)(n + 1)a_{n + 2}t^n + \sumzinf (n + 1)a_nt^n &= 0 \\ \sumzinf (n + 2)(n + 1)a_{n + 2}t^n + \sumzinf (n + 1)a_nt^n &= 0 \\ (n + 2)(n + 1)a_{n + 2} &= -(n + 1)a_n \\ a_{n + 2} &= -\frac{1}{n + 2}a_n \end{aligned}$$  Case 1: $a_0 = 1$, $a_1 = 0$ - All odd coefficients are zeros. $$\begin{aligned} a_2 &= -\frac{1}{2} \\ a_4 &= \frac{1}{4 \cdot 2} \\ a_6 &= -\frac{1}{6 \cdot 4 \cdot 2} \\ y_1(t) &= 1 - \frac{1}{2}t^2 + \frac{1}{2 \cdot 4}t^4 - \frac{1}{2 \cdot 4 \cdot 6}t^6 + \dots \end{aligned} $$ 
This series converges everywhere by the theorem. ($Q(t)$ and $R(t)$ converges everywhere.) \\ 
Case 2: $a_0 = 0$, $a_1 = 1$ - All even coefficients are zeros. $$\begin{aligned} a_3 &= -\frac{1}{3} \\ a_5 &= \frac{1}{5 \cdot 3} \\ a_7 &= -\frac{1}{7 \cdot 5 \cdot 3} \\ y_2(t) &= t - \frac{1}{3}t^3 + \frac{1}{5 \cdot 3}t^5 - \frac{1}{7 \cdot 5 \cdot 3}t^7 \end{aligned} $$ \end{example} 

\begin{example} Solve: $(2 + t^2)\ddy - t\dy - 3y = 0$. \\ Note that the radius of convergence is $\sqrt{2}$.  $$\begin{aligned} 
y&= \sumzinf a_nt^n \\ y' &= \sumzinf na_nt^n \\ y'' &= \sumzinf n(n - 1)a_nt^{n - 2} \\ 2\sumzinf n(n - 1)a_nt^{n - 2} + \sumzinf n(n - 1)a_nt^n - \sumzinf na_nt^n - 3\sumzinf a_nt^n &= 0 \\ 2\sum_{n = -2}^{\infty} (n + 2)(n + 1)a_{n + 2}t^n &= \sumzinf [n + 3 - n^2 + n] a_nt^n \\ 2\sumzinf (n + 2)(n + 1)a_{n + 2}t^n &= -[n^2 - 2n - 3]a_nt^n \\ &= -(n - 3)(n + 1)a_nt^n \\ 2(n + 2)(n + 1)a_{n + 2} &= (3 - n)(n + 1)a_n \\ a_{n + 2} &= \frac{3 - n}{2(n + 2)}a_n \end{aligned} $$ Case 1: $a_0 = 1$, $a_1 = 0$ $$\begin{aligned} 
a_2 &= \frac{3}{2 \cdot 2} \\ a_4 &= \frac{1}{2\cdot 4} \cdot \frac{3}{2 \cdot 2} \\ a_6 &= -\frac{1}{2 \cdot 6} \cdot \frac{1}{2 \cdot 4} \cdot \frac{3}{2 \cdot 2} \\ a_8 &= -\frac{3}{2 \cdot 8} \cdot -\frac{1}{2 \cdot 6} \cdot \frac{1}{2 \cdot 4} \cdot \frac{3}{2 \cdot 2} \\ a_{10} &= -\frac{5}{2 \cdot 10} \cdot - \frac{3}{2 \cdot 8} \cdot -\frac{1}{2 \cdot 6} \cdot \frac{1}{2 \cdot 4} \cdot \frac{3}{2 \cdot 2} \\ y_1(t) &= 1 + \frac{3}{2\cdot 2}t^2 + \frac{1 \cdot 3}{2^2 \cdot 2 \cdot 4}t^4 - \frac{1 \cdot 3}{2^3 \cdot 2 \cdot 4 \cdot 6}t^6 + \frac{3 \cdot 1 \cdot 3}{2^4 \cdot 2 \cdot 4 \cdot 6 \cdot 8}t^8 - \frac{5 \cdot 3 \cdot 1 \cdot 3}{2^5 \cdot 2 \cdot 4 \cdot 6 \cdot 8 \cdot 10}t^{10} + \dots \end{aligned} $$ This series converges for $-\sqrt{2} < t < \sqrt{2}$. \\ Case 2: $a_0 = 0$, $a_1 = 1$. $$\begin{aligned} a_3 &= \frac{2}{2 \cdot 3} \\ a_5 = 0 \\ a_7 &= 0 \\ y_2(t) &= t + \frac{1}{3}t^3 \end{aligned} $$ \end{example} 

\begin{example} Solve: $L[y] = \ddy + t^2\dy + 2ty = 0$. $$\begin{aligned} y &= \sumzinf a_nt^n \\ y' &= \sumzinf na_nt^{n - 1} \\ y'' &= n(n -1)a_nt^{n - 2} \\ L[y](t) &= \sumzinf n(n - 1)a_nt^{n - 2} + t^2 \sumzinf na_nt^{n - 1} + 2t\sumzinf a_nt^n \\ &= \sumzinf n(n - 1)a_nt^{n - 2} + \sumzinf na_nt^{n + 1} + 2\sumzinf a_nt^{n + 1} \\ &= \sumzinf n(n - 1)a_nt^{n - 2} + \sumzinf (n + 2)a_nt^{n + 1} \end{aligned} $$ Rewrite the first summation so that the exponent of the general term is $n + 1$ instead of $n - 2$. This is accomplished by increasing every $n$ underneath the summation sign by 3 and decreasing the lower limit by 3. $$\begin{aligned} \sumzinf n(n - 1)a_nt^{n - 2} &= \sum_{n = -3}^{\infty} (n + 3)(n + 2)a_{n + 3}t^{n + 1} \\ &= \sum_{n = -1}^{\infty} (n + 3)(n + 2)a_{n + 3}t^{n + 1} \end{aligned} $$ Thus $$\begin{aligned} L[y](t) &= \sum_{n = -1}^{\infty} (n + 3)(n + 2)a_{n + 3}t^{n + 1} + \sumzinf (n + 2)a_nt^{n + 1} \\ &= 2a_2 + \sumzinf (n + 3)(n + 2)a_{n + 3}t^{n + 1} + \sumzinf (n + 2)a_nt^{n + 1} \end{aligned} $$ Setting the sum of the coefficients of like powers of $t$ equal to zero gives $$ \begin{aligned} 2a_2 &= 0 \\ (n + 3)(n + 2)a_{n + 3} + (n + 2)a_n &= 0 \\ a_2 &= 0 \\ a_{n + 3} &= \frac{1}{n + 3}a_n \end{aligned} $$ 
Case 1: $a_0 = 1$, $a_0 = 1$. $$\begin{aligned} a_3 &= -\frac{1}{3} \\ a_6 &= \frac{1}{3 \cdot 6} \\ a_9 &= -\frac{1}{3 \cdot 6 \cdot 9} \\ a_{3n} &= \frac{(-1)^n}{3 \cdot 6 \cdot \dots \cdot 3n} = \frac{(-1)^n}{3^nn!} \\ y_1(t) &= 1 - \frac{1}{3}t^3 + \frac{1}{3 \cdot 6}t^6 - \frac{1}{3 \cdot 6 \cdot 9}t^9 + \dots = \sumzinf \frac{(-1)^n}{3^nn!}t^{3n} \end{aligned} $$ 
Case 2: $a_0 = 0$, $a_1 = 1$. $$\begin{aligned} a_4 &= -\frac{1}{4} \\ a_7 &= -\frac{1}{4 \cdot 7}  \\ a_{10} &= -\frac{1}{4 \cdot 7 \cdot 10} \\ y_2(t) &= t - \frac{1}{4}t^4 + \frac{1}{4 \cdot 7}t^7 - \frac{1}{4 \cdot 7 \cdot 10}t^{10} + \dots \end{aligned} $$ \end{example} 

\begin{example} Solve: $\ddy - t\dy  = 0$ $$\begin{aligned} y &= \sumzinf a_nt^n \\ y' &= \sumzinf n_nt^{n - 1} \\ y'' &= \sumzinf n(n - 1)a_nt^{n - 2} \\ \sumzinf n(n - 1)a_nt^{n - 2} - \sumzinf a_nt^{n - 1} &= 0 \\ \sum_{n = -3}^{\infty} (n + 3)(n + 2)a_{n + 3} t^{n + 1} &= \sumzinf a_nt^{n - 1} \\ (n + 3)(n + 2) a_{n + 3} &= a_n \\ a_{n + 3} &= \frac{1}{(n + 3)(n + 2)} a_n \end{aligned} $$ 
Case 1: $a_0 = 1$, $a_1 = 0$ $$\begin{aligned} a_3 &= \frac{1}{3 \cdot 2} \\ a_6 &= \frac{1}{6 \cdot 5 \cdot 3 \cdot 2} \\ a_9 &= \frac{1}{9 \cdot 8 \cdot 6 \cdot 5 \cdot 4 \cdot 3} \\ y_1 &= 1 + \frac{1}{3 \cdot 2}t^t + \frac{1}{6 \cdot 5 \cdot 4 \cdot 3} t^6 + \frac{1}{9 \cdot 8 \cdot 6 \cdot 5 \cdot 3 \cdot 2} + \dots \end{aligned} $$ 
Case 2: $a_0 = 0$, $a_1 = 1$ $$\begin{aligned} a_4 &= \frac{1}{4 \cdot 3} \\ a_7 &= \frac{1}{7 \cdot 6 \cdot 4 \cdot 3} \\ a_{10} &= \frac{1}{10 \cdot 9 \cdot 7 \cdot 6 \cdot 4 \cdot 3} \\ y_2 &= t + \frac{1}{4 \cdot 3}t^4 + \frac{1}{7 \cdot 6 \cdot 4 \cdot 3}t^7 + \frac{1}{10 \cdot 9 \cdot 7 \cdot 6 \cdot 4 \cdot 3} t^{10} + \dots \end{aligned} $$ 
This solution converges everywhere by the theorem. \end{example} 

\begin{example} Solve: $\ddy - 2t\dy + \lambda y = 0$ $$\begin{aligned} y &= \sumzinf a_nt^n \\ y' &= \sumzinf na_nt^{n - 1} \\ y'' &=\sumzinf n(n - 1)a_nt^{n - 2} \\ \sumzinf n(n - 1)a_nt^{n - 2} - 2\sumzinf na_nt^n + \lambda \sumzinf a_nt^n &= 0 \\ \sum_{n = -2}^{\infty} (n + 2)(n + 1)a_{n + 2}t^n - 2 \sumzinf na_nt^n + \lambda \sumzinf a_nt^n &= 0 \\ \sumzinf (n + 2)(n + 1)a_{n + 2}t^n - 2\sumzinf na_nt^n + \lambda \sumzinf s_nt^n &= 0 \\ \sumzinf (n + 2)(n + 1)a_{n + 2}t^n - \sumzinf (2n - \lambda) a_nt^n &= 0 \\ (n + 2)(n + 1)a_{n + 2} &= (2n - \lambda)a_n \\ a_{n + 2} &= \frac{2n - \lambda}{(n + 2)(n + 1)}a_n \end{aligned} $$ 
Case 1: $a_0 = 1$, $a_1 = 0$ $$\begin{aligned} a_2 &= \frac{-\lambda}{2 \cdot 1} \\ a_4 &= \frac{(4 - \lambda)(-\lambda)}{4 \cdot 3 \cdot 2 \cdot 1} \\ a_6 &= \frac{(8 - \lambda)(4 - \lambda)(-\lambda)}{6 \cdot 5 \cdot 4 \cdot 3 \cdot 2 \cdot 1} \\ a_8 &= \frac{(12- \lambda)(8 - \lambda)(4 - \lambda)(-\lambda)}{8 \cdot 7 \cdot 6 \cdot 5 \cdot4 \cdot 3 \cdot 2 \cdot 1} \\ y_1 &= 1 + \frac{(-\lambda)}{2!}t^2 + \frac{(4 - \lambda)(-\lambda)}{4!}t^4 + \frac{(8 - \lambda)(4 - \lambda)(-\lambda)}{6!}t^6 + \frac{(12 - \lambda)(8 - \lambda)(4 - \lambda)(-\lambda)}{8!}t^8 + \dots \end{aligned} $$ Case 2: $a_0 = 0$, $a_1 = 1$ $$\begin{aligned} a_3 &= \frac{2 - \lambda}{3 \cdot 2} \\ a_5 &= \frac{(6 - \lambda)(2 - \lambda)}{5 \cdot 4 \cdot 3 \cdot 2} \\ a_7 &= \frac{(10 - \lambda)(6  -\lambda)(2 - \lambda)}{7 \cdot 6 \cdot 5 \cdot 4 \cdot 3 \cdot 2} \\ y_2 &= t + \frac{2 - \lambda}{3!}t^3 + \frac{(6 - \lambda)(2 - \lambda)}{5!}t^5 + \frac{(10 - \lambda)(6 - \lambda)(2 - \lambda)}{7!}t^7 + \dots \end{aligned} $$ This solution converges  everywhere. \end{example} 

\begin{example} Solve: $(1 - t^2)\ddy - 2t\dy + \alpha(\alpha + 1)y = 0 $ $$\begin{aligned} y &= \sumzinf a_nt^n \\ y' &= \sumzinf na_nt^{n - 1} \\ y'' &= \sumzinf n(n - 1)a_nt^{n - 2} \\ &\sumzinf n(n - 1)a_nt^{n - 2} - \sumzinf n(n - 1)a_n - 2\sumzinf na_nt^n + \alpha(\alpha + 1)\sumzinf a_nt^n = 0 \\ &\sumzinf (n + 2)(n + 1)a_{n + 2}t^n = \sumzinf \Big[ n(n - 1) + 2n - \alpha(\alpha + 1)\Big] a_nt^n \\ &= \sumzinf \Big[ n^2 + n - \alpha(\alpha + 1)\Big] a_nt^n \\ &= \sumzinf (n + \alpha + 1)(n - \alpha) a_nt^n \\ a_{n + 2} &= \frac{(n + \alpha + 1)(n - \alpha)}{(n + 2)(n  +1)}a_n \end{aligned} $$ Case 1: $a_0 = 1$, $a_1 = 0$ $$\begin{aligned} a_2 &= \frac{(\alpha + 1)(-\alpha)}{2 \cdot 1} \\ a_4 &= \frac{(3 + \alpha)(2 - \alpha)(\alpha + 1)(-\alpha)}{4 \cdot 3 \cdot 2 \cdot 1} \\ a_6 &= \frac{(5 + \alpha)(3 + \alpha)(1 + \alpha)(-\alpha)(2 - \alpha)(4 - \alpha)}{6 \cdot 5 \cdot 4 \cdot 3 \cdot 2 \cdot 1} \\ y_1 &= 1 + \frac{(\alpha + 1)(-\alpha)}{2!}t^2 + \frac{(\alpha + 3)(\alpha + 1)(-\alpha)(2 - \alpha)}{4!}t^4 \\ &+ \frac{(\alpha + 5)(\alpha + 3)(\alpha + 1)(-\alpha)(2 - \alpha)(4  - \alpha)}{6!}t^6 + \dots \end{aligned} $$ Case 2: $a_0 = 0$, $a_1 = 1$ $$\begin{aligned} a_3 &= \frac{(2 + \alpha)(1 - \alpha)}{3 \cdot 2} \\ a_5 &= \frac{(4 + \alpha)(2 + \alpha)(1 - \alpha)(3 - \alpha)}{5 \cdot 4 \cdot 3 \cdot 2} \\ a_7 &= \frac{(6 + \alpha)(4 + \alpha)(2 + \alpha)(1 - \alpha)(3 - \alpha)(5 - \alpha)}{7 \cdot 6 \cdot 5 \cdot 4 \cdot 3 \cdot 2} \\ y_2 &= t + \frac{(2 + \alpha)(1 - \alpha)}{3!}t^3 + \frac{(4 + \alpha)(2 + \alpha)(1 - \alpha)(3 - \alpha)}{5!}t^5 \\ &+ \frac{(6 + \alpha)(4 + \alpha)(2 + \alpha)(1 - \alpha)(3 - \alpha)(5 - \alpha)}{7!}t^7 + \dots \end{aligned} $$ This solution converges for $-1 < t < 1$. \end{example} 

\begin{example} Solve: $\ddy - t^3y = 0$ $$\begin{aligned} y &= \sumzinf a_nt^n \\ y' &= \sumzinf na_nt^{n - 1} \\ y'' &= \sumzinf n(n - 1)a_nt^{n - 2} \\ \sumzinf n(n - 1)a_nt^{n - 2} - \sumzinf a_nt^{t + 3} &= 0 \\ \sum_{n = -5}^{\infty} (n + 5)(n + 4)a_{n + 5}t^{n - 3} - \sumzinf a_nt^{n + 3} &= 0 \\ \sum_{n = -3}^{\infty} (n + 5)(n + 4)a_{n + 5}t^{n + 3} &= \sumzinf a_nt^{n + 3} \\ a_2 = a_3 = a_4 &= 0 \text{ since there are no } t^2, t^3, t^4 \text{ terms on the RHS} \\ (n + 5)(n + 4)a_{n + 5} &= a_n \\ a_{n + 5} &= \frac{1}{(n + 5)(n + 4)}a_n \end{aligned} $$ 
Case 1: $a_0 = 1$, $a_1 = 0$ $$\begin{aligned} a_5 &= \frac{1}{5 \cdot 4} \\ a_{10} &= \frac{1}{10 \cdot 9 \cdot 5 \cdot 4} \\ a_{15} &= \frac{1}{15 \cdot 14 \cdot 10 \cdot 9 \cdot 5 \cdot 4} \\ y_1 &= 1 + \frac{1}{5 \cdot 4}t^5 + \frac{1}{10 \cdot 9 \cdot 5 \cdot 4}t^{10} + \frac{1}{15 \cdot 14 \cdot 10 \cdot 9 \cdot 5 \cdot 4}t^{15} + \dots \end{aligned} $$ 
Case 2: $a_0 = 0$, $a_1 = 1$ $$\begin{aligned} a_6 &= \frac{1}{6 \cdot 5} \\ a_{11} &= \frac{1}{11 \cdot 10 \cdot 6 \cdot 5} \\ a_{16} &= \frac{1}{16 \cdot 15 \cdot 11 \cdot 10 \cdot 6 \cdot 5} \\ y_2 &= t + \frac{1}{6 \cdot 5}t^6 + \frac{1}{11 \cdot 10 \cdot 6 \cdot 5}t^{11} + \frac{1}{16 \cdot 15 \cdot 11 \cdot 10 \cdot 6 \cdot 5} + \dots \end{aligned} $$ This solution converges everywhere. \end{example} 

\begin{example} Solve $(1 - t^2)\ddy - t\dy + \alpha^2y = 0$ $$\begin{aligned} y &= \sumzinf a_nt^n \\ y' &= \sumzinf na_nt^{n - 1} \\ y'' &= \sumzinf n(n - 1)a_nt^{n - 2} \\ &\sumzinf n(n - 1)a_nt^{n - 2} - \sumzinf n(n - 1)a_nt^n - \sumzinf na_nt^n + \alpha^2 \sumzinf a_nt^n = 0 \\ &\sum_{n = -2}^{\infty} (n + 2)(n + 1)a_{n + 2}t^n - \sumzinf n(n - 1)a_nt^n - \sumzinf na_nt^n + \alpha^2 \sumzinf a_nt^n = 0 \\ &\sumzinf (n + 2)(n + 1)a_{n + 2}t^n - \sumzinf n(n - 1)a_nt^n - \sumzinf na_nt^n + \alpha^2 \sumzinf a_nt^n = 0 \\ &\sumzinf (n + 2)(n + 1)a_{n + 2}t^n = \sumzinf n(n - 1)a_nt^n + \sumzinf na_nt^n - \alpha^2 \sumzinf a_nt^n  \\ &= \sumzinf \Big[ n(n - 1) + n - \alpha^2\Big] a_nt^n \\ &= \sumzinf (n^2 - \alpha^2)a_nt^n \\ (n + 2)(n + 1)a_{n + 2} &= (n^2 - \alpha^2)a_n \\ a_{n + 2} &= \frac{n^2 - \alpha^2}{(n + 2)(n + 1)}a^n \end{aligned} $$ 
Case 1: $a_0 = 1$, $a_1 = 0$ $$\begin{aligned} a_2 &= \frac{(-\alpha^2)}{2 \cdot 1} \\ a_4 &= \frac{(2^2 - \alpha^2)(-\alpha^2)}{4 \cdot 3 \cdot 2 \cdot 1} \\ a_6 &= \frac{(4^2 - \alpha^2)(2^2 - \alpha^2)(-\alpha^2)}{6 \cdot 5 \cdot 4 \cdot 3 \cdot 2 \cdot 1} \\ a_8 &= \frac{(6^2 - \alpha^2)(4^2 - \alpha^2)(2^2 - \alpha^2)(-\alpha^2)}{8 \cdot 7 \cdot 6 \cdot 5 \cdot 4 \cdot 3 \cdot 2 \cdot 1} \\ y_1 &= 1 + \frac{(-\alpha^2)}{2!}t^2 + \frac{(2^2 - \alpha^2)(-\alpha^2)}{4!}t^4 \\ &+ \frac{(4^2 - \alpha^2)(2^2 - \alpha^2)(-\alpha^2)}{6!}t^6 + \frac{(6^2 - \alpha^2)(4^2 - \alpha^2)(2^2 - \alpha^2)(-\alpha^2)}{8!}t^8 + \dots \end{aligned} $$ 
Case 2: $a_0 = 0$, $a_1 = 1$ $$\begin{aligned} a_3 &= \frac{(1^2 - \alpha^2)}{3 \cdot 2} \\ a_5 &= \frac{(3^2 - \alpha^2)(1^2 - \alpha^2)}{5 \cdot 4 \cdot 3 \cdot 2} \\ a_7 &= \frac{(5^2 - \alpha^2)(3^2 - \alpha^2)(1^2 - \alpha^2)}{7 \cdot 6 \cdot 5 \cdot 4 \cdot 3 \cdot 2} \\ a_9 &= \frac{(7^2 - \alpha^2)(5^2 - \alpha^2)(3^2 - \alpha^2)(1^2 - \alpha^2)}{9 \cdot 8 \cdot 7 \cdot 6 \cdot 5 \cdot 4 \cdot 3 \cdot 2} \\ y_2 &= t + \frac{(1^2 - \alpha^2)}{3!}t^3 + \frac{(3^2 - \alpha^2)(1^2 - \alpha^2)}{5!}t^5 \\ &+ \frac{(5^2 - \alpha^2)(3^2 - \alpha^2)(1^2 - \alpha^2)}{7!}t^7 + \frac{(7^2 - \alpha^2)(5^2 - \alpha^2)(3^2 - \alpha^2)(1^2 - \alpha^2)}{9!}t^9 + \dots \end{aligned} $$ 
This solution converges from $-1 < t < 1$. \end{example} 




























\end{document}