\documentclass[12pt]{article}
\usepackage[letterpaper, portrait, margin=1in]{geometry}
\usepackage{amsmath, amsthm, amssymb, mathrsfs}

\usepackage{listings}
\usepackage{color}

\definecolor{dkgreen}{rgb}{0,0.6,0}
\definecolor{gray}{rgb}{0.5,0.5,0.5}
\definecolor{mauve}{rgb}{0.58,0,0.82}

\lstset{frame=tb,
  language=Java,
  aboveskip=3mm,
  belowskip=3mm,
  showstringspaces=false,
  columns=flexible,
  basicstyle={\small\ttfamily},
  numbers=none,
  numberstyle=\tiny\color{gray},
  keywordstyle=\color{blue},
  commentstyle=\color{dkgreen},
  stringstyle=\color{mauve},
  breaklines=true,
  breakatwhitespace=true,
  tabsize=3
}

\usepackage{fancyhdr}
\pagestyle{fancy}
\fancyhf{}
\lhead{Darshan Patel}
\rhead{Computer Science 212: Object-Oriented Programming in Java}
\renewcommand{\footrulewidth}{0.4pt}
\cfoot{\thepage}

\begin{document}

\title{Computer Science 212: Object-Oriented Programming in Java}
\author{Darshan Patel}
\date{Fall 2016}
\maketitle

\tableofcontents

\section{Introduction to Java}
\begin{itemize} 
\item Java code can be written once and run anywhere
\item Java source code goes to a compiler for a machine and made into source code which can be run on any machine 
\item API: application program interface- class libraries (MATH, GUI, Database...) 
\item JRI: Java Runtime Environment: JVM plus API (can run Java programs)
\item JDK: Java Development Kit: JRE plus compiler, javadoc, ...
\item Example: Java source code is written and stored in a .java file, compiler then calls "javac name.java" making a class file "name.class" which can then be run "run name" 
\item The Java virtual machine ensures a program will run the same way on any computer 
\item When running on native hardware, you are dependent on the machine's architecture 
\item The JVM levels the playing field

\end{itemize}

\section{Classes and Objects} 
\begin{itemize} 
\item Primitives: single-valued data items, not objects \begin{itemize} 
\item byte: 8 bit signed two's complement integer
\item short: 16 bit signed two's complement integer 
\item int: 32 bit signed two's complement integer 
\item long: 64 bit signed two's complement integer 
\item float: 32 bit single precision IEEE 754 
\item double: 64 bit single precision IEEE 754
\item boolean: true/false 
\item char: 16 bit Unicode character \end{itemize} 
\item The number of bits and the range of values: $$n = 32$$ $$-(2^{n- 1})\dots(2^{n - 1}) + 1$$
$$-2^{31}\dots2^{31} - 1 $$ $$-2,147,483,648\dots2,147,483,647$$ 
\item ASCII: 8 bit char code $$2^8 = 256 \text{ char} $$
\item Unicode: 16 char code $$2^{16} = 2^6 \times 2^{10} = 64,000 \text{ char}$$
\item Class: a blueprint or template that describes the properties and behaviors of an object 
\item Object: an instance of a class which has specific properties 
\item Instantiation: making an instance of a class
\item Static variables belong to a class - only one variable for every instance of the class
\item Instance variables belong to an object - the object is an instance of the class 
\item Static variables can be changed 
\item Use the final modifier to make constants 
\item Methods: defines the behavior of an object 

\end{itemize}

\section{Arrays and Sorting}
\begin{itemize} 
\item \begin{lstlisting}
private static int inputFromFile(String filename, short[] numbers){
      TextFileInput in = new TextFileInput(filename);
      int lengthFilled = 0;
      String line = in.readLine();
      while ( lengthFilled < numbers.length && line != null ) {
         numbers[lengthFilled++] = Short.parseShort(line);
         line = in.readLine();
      } // while 
      if ( line != null ) {
         System.out.println("File contains too many numbers.");
         System.out.println("This program can process only " +
                            numbers.length + " numbers.");
         System.exit(1);
      } // if
      in.close();
      return lengthFilled; 
   }  // method inputFromFile 
      
   
   private static void selectionSort (short[] array, int length) { 
      for ( int i = 0; i < length - 1; i++ ) { 
         int indexLowest = i; 
         for ( int j = i + 1; j < length; j++ ) 
            if ( array[j] < array[indexLowest] ) 
               indexLowest = j;
         if ( array[indexLowest] != array[i] ) { 
            short temp = array[indexLowest];
            array[indexLowest] = array[i]; 
            array[i] = temp; 
         }  // if
      } // for i 
   } // method selectionSort 
\end{lstlisting}
\end{itemize}

\section{Methods and Parameter Passing} 
\begin{itemize} 
\item Primitive type parameters are passed by value 
\item Pass by value means a copy of the value of the parameter is given to the method
\item Each variable in the main program and the method has its own memory location
\item Object type parameters are passed by reference 
\item Pass by reference means a reference to the parameter is given to the method 
\item Each variable in the main program and the method has its own memory location for the reference 
\item Since we have a reference to an object in a method, any changes made to that object are permanent 
\item Changes happen to the object whose reference is used in a method call
\item \begin{lstlisting} 
private static void selectionSort
                         (short[] array, int length) { 
for ( int i = 0; i < length - 1; i++ ) { 
  int indexLowest = i; 
  for ( int j = i + 1; j < length; j++ ) 
    if ( array[j] < array[indexLowest] ) 
       indexLowest = j;
       if ( array[indexLowest] < array[i] ) { 
         swap( array,indexLowest,i); 
       }
   } // for i 
} // method selectionSort 

\end{lstlisting}
\item \begin{lstlisting}
public static void swap 
                (int[] a, int x, int y) {
  int temp;
  temp = a[x];
  a[x]=a[y];
  a[y]=temp;
}

\end{lstlisting}
\end{itemize}

\section{Program Modularity and Error Checking} 
\begin{itemize} 
\item Program modularity: break a program down into smaller parts (methods), test each method separately, understand the relationship between the methods: parameters and their expected values
\item Date validation: data entered by a use (at a prompt), data entered by a user (into a file), data values received from other methods
\item A method should verify what it receives and returns 
\item System.exit(0) causes the program to end
\item "Run-time" error messages can provide useful information to the programmer or to the user 
\item Errors could be due to bad code or bad data 
\item Bad code and testing: test the program with all kinds of possible data
\item From method to method, all data variables in the program should be in a "correct" state
\item Ex: a method that sorts should produce a sorted object 
\item Assertions: used during program development, testing for errors in the logic of the program
\item They are usually "turned off" when a program is run by the user 
\item Such errors should not occur in the final version of the program 
\item Once an assertion is "thrown," the program terminates 
\item These errors should not happen in "real life"
\item The string in the assertion statement is printed along with a stack trace 
\item Exception: an error that can be "thrown" by a method
\item Ex: Integer.parseInt("abc") will throw an exception called "IllegalArgumentException"
\item Our program can also throw these common exceptions and terminate 
\item To throw an exception: \begin{lstlisting}
throw new IllegalArgumentException
                          ("SSN must have only digits.");
\end{lstlisting}

\end{itemize}

\section{GUIs and Inheritance} 
\begin{itemize} 
\item It is common for the "main" application and the GUI to be separate classes
\item By using inheritance, all things a JFrame can do is brought to the class \begin{lstlisting} 
public class SSNGUI extends JFrame{?
	
}
\end{lstlisting}
\item Non-Application Classes (no "main" method) \begin{itemize} 
\item Classes without "main" methods are true objects 
\item They cannot exist without being instantiated 
\item They may inherit from other classes so little extra code must be written
\item When an object is instantiated, a special method called a "constructor" is automatically executed 
\item The name of the constructor is the same as the name of the class
\item The constructor has no "return" attributes
\item The constructor is the initialization method for the object 
\item The constructor may take parameters from the instantiating method for initial values
\end{itemize} 
\item \begin{lstlisting} 
import javax.swing.*;
import java.awt.*;
public class SSNJFrame3{
   static final int LIST_SIZE = 10;
   static String ssn;
   static String[] ssnList;
   static int ssnSize;
   static TextFileInput inFile;
   static String inFileName = "SSN.txt";
   static JFrame myFrame;
   
   public static void main(String[] args) {
      initialize();
      readSSNsFromFile(inFileName);
      printSSNList(ssnList,ssnSize);
      printSSNtoJFrame(myFrame,ssnList,ssnSize);
   }   
   public static void initialize() {
	   ssn="";
	   ssnList= new String[LIST_SIZE];
	   ssnSize=0;
       inFile = new TextFileInput(inFileName);
       myFrame=new JFrame();
       myFrame.setSize(400,200);
       myFrame.setLocation(100, 100);
       myFrame.setTitle("Social Security Numbers");
       myFrame.setDefaultCloseOperation(JFrame.EXIT_ON_CLOSE);
   }
   }
   \end{lstlisting}

\end{itemize}

\section{Defining a Simple Class}
\begin{itemize} 
\item An object is defined by its attributes and behavior 
\item Attributes: data values 
\item Behavior: defined by the methods
\item The reason the data value is private is to ensure that the object contains correct data values 
\item Public data values can be changed by the user 
\item Methods that assign values to the data values should check for validity and throw an exception if they are not valid 
\item A method is private if it is not to be called from outside the class
\item A method is static if it is not the behavior of an object
\item Since all classes inherit from class Object, it automatically has methods: \begin{lstlisting} 
equals(Object o);
toString();
\end{lstlisting}
\item These methods may not behave the way we want to 
\item The "this" operator is a reference to the class in which it is used

\end{itemize}

\section{Dynamic vs. Static Structures: Linked Lists} 
\begin{itemize} 
\item Static storage structures: once declared, they are fixed in size
\item the size may be a "variable" or a constant but once declared, the size is fixed
\item Dynamic structures use only as much memory as they need 
\item Dynamic structures rely on themselves to determine the order of the data items they contain 
\item Dynamic structures are not stored in contiguous memory 
\item A node has 2 components: the actual data of the node and a reference linking it to the next node in the list 
\item It is helpful to have an empty dummy node at the beginning of the list 
\item \begin{lstlisting}
public class ListNode {
   String data;
   ListNode next;
   
   public ListNode(String data, ListNode next)  {
      this.data = data;
      this.next = next;
   }  // constructor
   
   public ListNode()  {
      this.data = null;
      this.next = null;
   }  // constructor
   
   public ListNode(String data)  {
      this.data = data;
      this.next = null;
   }  // constructor

}
\end{lstlisting}
\item \begin{lstlisting}
public class LinkedListIterator {

   private ListNode node;
   public LinkedListIterator(ListNode first)  {
        node = first;
   }

   public boolean hasNext()  {
      return ( node != null );
   }
   
   public String next()  {
      if ( node == null )
         throw new NullPointerException("Linked list empty.");
      String currentData = node.data;
      node = node.next;
      return currentData;
   }

}

\end{lstlisting}
\item \begin{lstlisting}
public void append (String s) {
  ListNode n = new ListNode(s);
  last.next = n;
  last = n;
  length++;
}
\end{lstlisting}
\item \begin{lstlisting}
public void printList () {
  ListNode p = first.next;
  while (p != null) {
    System.out.println(p.data);
    p = p.next;
  }
  }
\end{lstlisting}
\item \begin{lstlisting}
public ListNode find (String s) {
  ListNode p = first.next;
  while (p != null && !(p.data).equals(s)) {
    p = p.next;
  } // while
  return p;
}
\end{lstlisting}

\end{itemize}

\section{Inheritance and Polymorphism} 
\begin{itemize} 
\item Extending a class is based on the "is a" relationship 
\item The protected modifier grants access only from descendant classes 
\item Public grants access from any class 
\item Private grants access only to instances of the same class 
\item When a class is instantiated, the first thing it must do is "construct" its super class 
\item Calling one of the constructors of the super class is done using the method super($<$optional parameters$>$) 
\item The name of the constructor is the same as the name of the class and has no return type 
\item An abstract class cannot be instantiated
\item  A class is abstract if \begin{itemize} 
\item it is declared as abstract 
\item it contains an abstract method 
\item it inherits an abstract method and does not overload it \end{itemize} 
\item The instanceof operator tests if an object (instance) is a subtype of a given type 

\end{itemize}

\section{GUIs and Event-Driven Programming} 
\begin{itemize} 
\item JFrames: complete Window objects, they don't do anything until told to
\item An event, such as a menu choice, signals the JFrame to respond 
\item Three Tier Architecture: graphical user interface, program logic - decisions based on the user's choices, data (files) - data available for the user's choice 
\item All the main program needs to do is instantiate the GUI
\item An event is something that happens while the program is running 
\item An Event Handler is a method that is automatically called when an event, such as choosing a menu item, occurs 
\item An Event Handler can handle more than one event but each event needs a handler (or nothing will happen) 
\item Event Handlers are written in a class that implements an interface called ActionListener 
\item An Interface is a collection of method headings only (not bodies) 
\item Interfaces are implemented by a Java class 
\item If an interface is implemented, all methods specified in the interface must be provided by that class 
\item An interface, if implemented, guarantees that all methods will be defined 
\item The interface ActionListener contains a method called actionPerformed(ActionEvent) 
\item The actionPerformed method is called when an event happens 
\item Each event needs to be registered with some ActionListener 
\item When an ActionEvent object is created, the actionPerformed method of the handler that is registered with the event is called and the ActionEvent is passed to it as a parameter 
\item \begin{lstlisting} 
import javax.swing.*;
import java.awt.*;
public class SSNGUI extends JFrame {
      
   public SSNGUI(String title, int height, int width) {
	    setTitle(title);
	    setSize(height,width);
       setLocation  (400,200);
	    createFileMenu();
	    setDefaultCloseOperation(EXIT_ON_CLOSE);
       setVisible(true);
   } //SSNGUI

   private void createFileMenu( ) {
      JMenuItem   item;
      JMenuBar    menuBar  = new JMenuBar();
      JMenu       fileMenu = new JMenu("File");
      FileMenuHandler fmh  = new FileMenuHandler(this);

      item = new JMenuItem("Open");    //Open...
      item.addActionListener( fmh );
      fileMenu.add( item );

      fileMenu.addSeparator();           //add a horizontal separator line
    
      item = new JMenuItem("Quit");       //Quit
      item.addActionListener( fmh );
      fileMenu.add( item );

      setJMenuBar(menuBar);
      menuBar.add(fileMenu);
    
   } //createMenu

} //SSNGUI\end{lstlisting} 
\item \begin{lstlisting} 
import java.awt.event.*;
import java.io.*;
public class FileMenuHandler implements ActionListener {
   JFrame jframe;
   public FileMenuHandler (JFrame jf) {
      jframe = jf;
   }
   public void actionPerformed(ActionEvent event) {
      String  menuName;
      menuName = event.getActionCommand();
      if (menuName.equals("Open"))
         openFile( ); 
      else if (menuName.equals("Quit"))
         System.exit(0);
   } //actionPerformed
   }
\end{lstlisting} 



\end{itemize}

\section{Exception Handling} 
\begin{itemize} 
\item To throw an exception: \begin{lstlisting} 
public SSN (String s) {
      if (isValidSSN(s) )
         SSNumber = s;
      else 
         throw new IllegalSSNException("Invalid SSN: "+s);
   }
\end{lstlisting} 
\item Extending an existing exception: \begin{lstlisting}
public class IllegalSSNException
        extends IllegalArgumentException {
   public IllegalSSNException(String message) {
      super (message);
   }
}
\end{lstlisting} 
\item When an exception is thrown, the RunTime System looks for a method that can handle the exception
\item If no such method is found, the Runtime System handles the exception and terminates the program 
\item The Runtime System looks at the most recently called method and backs up all the way to the main program 
\item If any one of the previous methods knew how to handle the exception, it would be an exception catcher 
\item Multiple exceptions can be taught by a try-catch block
\item The JVM will go through the catch blocks top to bottom until a matching error is found 
\item This is why the order of the exceptions listed is important, because of the class hierarchy and inheritance
\item When an exception is not caught, the program terminates and does not continue 
\item The finally block is executed whether or not an exception occurs 
\item It is executed even if there is a return statement prior to the final code 
\item Excluding exceptions in the class RuntimeException, the compiler must find a catcher or a propagator for every exception
\item RuntimeException is an unchecked exception while all other exceptions are checked exceptions 

\end{itemize}

\section{Regular Expressions} 
\begin{itemize} 
\item A Regular Expression (regex) is a pattern that can be matched against a string 
\item \begin{lstlisting} 
import java.util.regex.*;

public static isValidSSN(String ssn) {
	Pattern p;
	Matcher m;
	String SSN_PATTERN =              ;
	p = Pattern.compile(SSN_PATTERN);
	m = p.matcher(ssn);
	return matcher.matches();

}
\end{lstlisting} 
\item Constants: match exactly the string inside the regex 
\item Character Classes (match any character inside [] \begin{itemize} 
\item [abc] - a, b, or c (simple class) 
\item [\^abc] - any character except a, b, or c (negation) 
\item [a-zA-Z] - a through z, or A through Z, inclusive (range) 
\item [a-d[m-p]] - a through d or m through p: [a -dm-p] (union) 
\item [a-z\&\&[def]] - d, e or f (intersection) 
\item [a-z\&\&[\^bc]] - a through z, except for b and c: [ad-z]
\item [a-z\&\&[\^m-p]] - a through z, and not m through p: [a-lq-z] \end{itemize} 
\item Predefined Character Classes \begin{itemize} 
\item . - any character (may or may not match line end) 
\item \textbackslash d - a digit: [0-9]
\item \textbackslash D - a non-digit: [\^0-9]
\item \textbackslash s - a whitespace character: [\textbackslash t\textbackslash n\textbackslash x0B\textbackslash f\textbackslash r]
\item \textbackslash S - a non-whitespace character: [\^\textbackslash s]
\item \textbackslash w - a word character: [a-zA-Z\_0-9]
\item \textbackslash W - a non-word character: [\^\textbackslash w] \end{itemize} 
\item Quantifiers \begin{itemize} 
\item X? - X, once or not at all 
\item X* - X, zero or more times 
\item X+ - one or more times 
\item X{n} - X, exactly n times 
\item X{n,} - X, at least n times 
\item X{n,m} - X, at least n but not more than m times \end{itemize} 
\item \^ - beginning of regex, \$ - end of regex 

\end{itemize}

\section{Maps}
\begin{itemize} 
\item In a hashmap, a "hash function" maps key to index 
\item Issues: search in time O(c), collisions, growth 
\item \begin{lstlisting} 
import java.util.HashMap;
import java.util.Iterator;
 
public class HashMapExample{
       public static void main(String args[]){
          HashMap hashMap = new HashMap();
          hashMap.put("One", new Integer(1)); 
          hashMap.put("Two", new Integer(2));
          hashMap.put("Three", new Integer(3));

        Integer myInt = hashMap.get("Two");
      
        if(hashMap.containsValue(new Integer(1)))
            System.out.println("HashMap contains 1 as value");
         else{
            System.out.println("HashMap does not contain 1 as value");
       
        if( hashMap.containsKey("One") )
            System.out.println("HashMap contains One as key");
        else
            System.out.println("HashMap does not contain One as value");
}
}
}
\end{lstlisting} 
\item To get the keys and values out of the HashMap
\item \begin{lstlisting} 
       Iterator itr;
       System.out.println("Retrieving all keys from the HashMap"); 
       
       itr = hashMap.keySet().iterator();
       while(itr. hasNext()){        
          System.out.println(itr.next());
        }
       
           
        System.out.println("Retrieving all values from the HashMap");

        itr = hashMap.entrySet().iterator();
        while(itr. hasNext()){        
           System.out.println(itr.next());
       }
       \end{lstlisting}
\item The order items went in is not the same as how they come out 
\item A TreeMap arranges the data keys so they come out in order when using the iterator 
\item \begin{lstlisting} 
TreeMap <String, String> french  = 
			new TreeMap<String, String> ( );
\end{lstlisting} 
\item entrySet() returns a collection of key/value pairs 
\item interface Map.Entry is a key/value pair 
\item \begin{lstlisting} 
Set set = french.entrySet(); 
Iterator i = set.iterator(); 
Map.Entry <String,String> me;

while(i.hasNext()) { 
   me = (Map.Entry)i.next(); 
   System.out.print(me.getKey() + ": "); 
   System.out.println(me.getValue()); 
} 
\end{lstlisting} 
\item Order can be assumed if the class implements Comparable 
\item \begin{lstlisting} 
TreeMap <String, String> french  = 
			new TreeMap<String, String> ( );
\end{lstlisting} 
\item For user-defined objects, the TreeMap needs to know how to order the keys 
\item \begin{lstlisting} 
TreeMap <SSN, Integer> treeMap  = 
			new TreeMap (new SSNComparator());\end{lstlisting} 
\item Comparator is a class that implements Comparator which has a method int compare (Object, Object) 
\item \begin{lstlisting} 
import java.util.Comparator;

public class SSNComparator implements Comparator <SSN>  {
   public int compare(SSN num1, SSN num2) {
      return num1.compareTo(num2);
   }\end{lstlisting}
\item The TreeMap is efficient because it keeps item in order and has O(n) for adding new values 
\item The TreeMap is based on the Red-Black tree 
\end{itemize}

\section{File Input and Output}
\begin{itemize} 
\item File I/O is done through the operating system: file system to operating system to program
\item Files can be stored on a variety of devices, read/written by many operating systems
\item Constructor for TextFileInput \begin{lstlisting} 
public TextFileInput(String filename)
   {
      this.filename = filename;
      try  {
         br = new BufferedReader(
                  new InputStreamReader(
                      new FileInputStream(filename)));
      } catch ( IOException ioe )  {
         throw new RuntimeException(ioe);
      }  // catch
   }  // constructor
}
\end{lstlisting}
\item public FileInputStream(String�name) throws FileNotFoundException \newline

Creates a FileInputStream by opening a connection to an actual file, 
the file named by the path name name in the file system. A new FileDescriptor 
object is created to represent this file connection. \newline

Parameters:
name - the system-dependent file name. \newline 
Throws: 
FileNotFoundException - if the file does not exist, is a directory rather than 
a regular file, or for some other reason cannot be opened for reading. 
\item public int read() throws IOException \newline
Reads a byte of data from this input stream. This method blocks if no input is yet available.
\item An InputStreamReader is a bridge from byte streams to character streams: It reads bytes and decodes them into characters using a specified charset. The charset that it uses may be specified by name or may be given explicitly, or the platform's default charset may be accepted. \begin{lstlisting}
public TextFileInput(String filename)
   {
      this.filename = filename;
      try  {
         br = new BufferedReader(
                  new InputStreamReader(
                      new FileInputStream(filename)));
      } catch ( IOException ioe )  {
         throw new RuntimeException(ioe);
      }  // catch
   }  // constructor
\end{lstlisting} 
\item The FileInputStream reads ASCII bytes from the file and delivers a stream of 32-bit int values 
\item The InputStreamReader converts the ints to a stream of Unicode characters (the default character set)
\item public String readLine() throws IOException \newline
Read a line of text. A line is considered to be terminated by any one of a line feed ('\textbackslash n'), a carriage return ('\r'), or a carriage return followed immediately by a linefeed. \newline
Returns: 
A String containing the contents of the line, not including any line-termination characters, or null if the end of the stream has been reached \newline
Throws: 
IOException - If an I/O error occurs
\item The BufferedReader separates the stream of Unicode characters into "lines" of the file (a line is terminated with lineFeed \textbackslash n, carriageReturn or \textbackslash n\textbackslash r) 
\item The class File is an abstract representation of file and directory pathnames 
\item \begin{lstlisting} 
import java.io.File;
import javax.swing.*;

public class SingleFile {
   public static void main (String args[]){
   JFileChooser fileChooser = new JFileChooser();
   fileChooser.showOpenDialog(null);
   File myFile = fileChooser.getSelectedFile();
   System.out.println("getName(): "+myFile.getName());
   System.out.println("getParent(): "+myFile.getParent());
   System.out.println("getPath(): "+myFile.getPath());
   System.out.println("lastModified(): "+myFile.lastModified());
   System.out.println("length(): "+myFile.length());
   }
}
\end{lstlisting} 
\item \begin{lstlisting} 
import java.io.File;
import javax.swing.*;
public class ListFiles {

	public static void main(String[] args) {
        JFileChooser fd = new JFileChooser();
//        mode - the type of files to be displayed:
//            * JFileChooser.FILES_ONLY
//            * JFileChooser.DIRECTORIES_ONLY
//            * JFileChooser.FILES_AND_DIRECTORIES 
        fd.setFileSelectionMode(JFileChooser.DIRECTORIES_ONLY);
        fd.showOpenDialog(null);
		File f = fd.getSelectedFile();
		listFiles(f,"");
	}
	public static void listFiles(File f, String indent) {
		File files[] = f.listFiles();
		
		for (int i = 0; i<files.length; i++) {
			File f2 = files[i];
			System.out.print(f2.getName());
			if (f2.isDirectory())
				listFiles(f2, indent+"   ");
			System.out.print("...");
			System.out.print(f2.length());
			System.out.println();
		}
	}
}
\end{lstlisting} 
\end{itemize}

\section{Generics}
\begin{itemize} 
\item Generic: of, applicable to, or referring to all the members of a genus, class, group, or kind; general
\item The advantage of using generics is to make generalized variables so it can be widely used without edited 
\item \begin{lstlisting} 
public class ListNode <E> {
   E data;
   ListNode next;
   public ListNode(E myData) {
      data=myData;
      next=null;
   }
   public ListNode() {
      data=null;
      next=null;
   }
}
\end{lstlisting} 
\item \begin{lstlisting} 
public class LinkedList<E> {
   private ListNode first;
   private ListNode last;
   private int length;
   
   public LinkedList() {
      ListNode ln = new ListNode();
      first = ln;
      last = ln;
      length = 0;
   }
   public void append ( E myData){
      ListNode n = new ListNode(myData);
      last.next = n;
      last = n;
      length++;
   }
   ....
}
\end{lstlisting} 
\item To make a LinkedList, write "LinkedList$<$String$>$ stringList = new LinkedList$<$String$>$(); "
 

\end{itemize}

\section{Recursion and the Run Time Stack} 
\begin{itemize} 
\item A recursive method is a method that calls itself 
\item A recursive algorithm is one that solves a problem by using the same algorithm to solve a smaller part of the problem 
\item Ex: to list all the presidents of the US, identify the first president and then list the rest of the presidents 
\item \begin{lstlisting} 

public class BinomialCoefficient {

   public static void main(String[] args) {
      int[] n = {4,4,4,4,4};
      int[] r = {0,1,2,3,4};
      int x,y;
      for (int i=0; i<n.length; i++) {
         x=n[i];
         y=r[i];
         System.out.println(x+" choose "+y+" is "+bc(x,y));
      }
   }
   private static int bc (int n, int r) {
       if (n==0 || r==0 ||n==r) {
            return 1;
      }
      else
         return bc(n-1,r)+bc(n-1,r-1);
   }
}\end{lstlisting} 
\item \begin{lstlisting} 

public class EuclidianGCD {

   public static void main(String[] args) {
      int[] testNumerators = {4,6,1,0,15,20};
      int[] testDenominators = {8,18,2,5,225,225};
      int n,d;
      for (int i=0; i<testNumerators.length; i++) {
         n=testNumerators[i];
         d=testDenominators[i];
         System.out.println("The greatest common divisor of "+n+" and "+d+
               " is "+gcd(n,d));
      }
   }
   private static int gcd (int n, int d) {
         if (d == 0) return n;
         else return gcd(d, n % d);
   }
}\end{lstlisting} 
\item \begin{lstlisting} 

public class Factorial {

   public static void main(String[] args) {
      int[] testValues = {4,6,1,0};
      int n;
      for (int i=0; i<testValues.length; i++) {
         n=testValues[i];
         System.out.println("Factorial("+n+") = "+factorial(n));
      }
   }
   private static int factorial (int n) {
      if (n==0)
         return 1;
      else
         return n*factorial(n-1);
   }
}\end{lstlisting} 
\item \begin{lstlisting} 
public class Fibonacci {

   public static void main(String[] args) {
      int[] testValues = {4,6,1,0,9,18};
      int n;
      for (int i=0; i<testValues.length; i++) {
         n=testValues[i];
         System.out.println("The "+n+suffix(n)+" Fibonacci number is "+fibonacci(n));
      }
   }
   private static int fibonacci (int n) {
      if (n==0)
         return 0;
      if
         (n==1)
         return 1;
      return fibonacci(n-1)+fibonacci(n-2);
   }
}
\end{lstlisting}
\item \begin{lstlisting} 
import javax.swing.*;
public class Pascal {

      public static void main (String[] args){
         int numRows = Integer.parseInt(JOptionPane.showInputDialog(null,"How many rows?"))-1;
         for (int i=0;i<= numRows;i++){
            for (int j=0; j<=i; j++)
               System.out.print(bc(i,j)+" ");
            System.out.println();
         }
      }
      private static int bc (int n, int r) {
         if (n==0 || r==0 ||n==r) {
              return 1;
        }
        else
           return bc(n-1,r)+bc(n-1,r-1);
     }
      
}\end{lstlisting} 
\item \begin{lstlisting} 

public class TowersOfHanoi {

   public static void main(String[] args) {
      moveRings(3,"A","B","C");

   }
   private static void moveRings(int numberOfRings, String fromTower,
                                 String toTower, String tempTower) {
      if (numberOfRings == 0) return;
      moveRings (numberOfRings-1, fromTower, tempTower, toTower);
      System.out.println("Move a ring from tower "+fromTower+" to tower "+
                           toTower);
      moveRings (numberOfRings-1, tempTower, toTower, fromTower);
   }
}\end{lstlisting} 

\end{itemize}

\section{Model-View-Controller (MVC)}
\begin{itemize} 
\item Model: a representation of the data with no concern for how it will appear to the user - extends Observable 
\item View: displays the data using GUI components by observing the Model - implements Observer 
\item Controller: a Listener that responds to events and updates the Model 
\item Regular Program \begin{lstlisting} 
public class TemperatureModel  {
    private double temperatureF = 32.0;
    public double getF() {
      return temperatureF;
  }
  public double getC(){
      return (temperatureF - 32.0) * 5.0 / 9.0;
  }
  public void setF(double tempF)
  {  
      temperatureF = tempF;
  }
  public void setC(double tempC)
  {  temperatureF = tempC*9.0/5.0 + 32.0;
  }
}
\end{lstlisting} 
\item Model Class \begin{lstlisting} 
import java.util.Observable;
public class TemperatureModel extends Observable {
    private double temperatureF = 32.0;
    public double getF() {
      return temperatureF;
   }
  public double getC(){
      return (temperatureF - 32.0) * 5.0 / 9.0;
   }
  public void setF(double tempF)
   {  
      temperatureF = tempF;
      setChanged();
      notifyObservers();
   }
  public void setC(double tempC)
   {  temperatureF = tempC*9.0/5.0 + 32.0;
      setChanged();
      notifyObservers();
   }
}
\end{lstlisting}
\item Controller (Listener) \begin{lstlisting} 
import java.awt.event.ActionEvent;
import java.awt.event.ActionListener;

  class UpListener implements ActionListener  {
     TemperatureModel model;
   
      public UpListener(TemperatureModel m) {
      model = m;
   }

      public void actionPerformed(ActionEvent e) {  
         model.setF(model.getF() + 1.0);
      }
   }\end{lstlisting} 
\item GUI class \begin{lstlisting} 
import java.awt.*;
import java.awt.event.*;
abstract class TemperatureGUI implements java.util.Observer {
   private String label;
   private TemperatureModel model;
   private Frame temperatureFrame;
   private TextField display = new TextField();
   private Button upButton = new Button("Raise");
   private Button downButton = new Button("Lower");
   
   TemperatureGUI(String theLabel, TemperatureModel tModel, int h, int v) {  
      label = theLabel;
      model = tModel;
      Frame temperatureFrame;
      temperatureFrame = new Frame(label);
      temperatureFrame.add("North", new Label(label));
      temperatureFrame.add("Center", display);
      Panel buttons = new Panel();
      buttons.add(upButton);
      buttons.add(downButton);      
      temperatureFrame.add("South", buttons);      
      temperatureFrame.addWindowListener(new CloseListener()); 
      model.addObserver(this); // Connect the View to the Model
      temperatureFrame.setSize(200,100);
      temperatureFrame.setLocation(h, v);
      temperatureFrame.setVisible(true); 
   
   public void setDisplay(String s){ 
      display.setText(s);}
   public double getDisplay()  {  
      return Double.valueOf(display.getText()).doubleValue();
   }
      continued?
}
}
\end{lstlisting}
\item If it is to operate in Fahrenheit \begin{lstlisting} 
import java.awt.*;
import java.awt.event.*;
import java.util.Observable;

public class FarenheitGUI extends TemperatureGUI {
   
   public FarenheitGUI(TemperatureModel model, int h, int v) {  
      super("Farenheit Temperature", model, h, v);
      setDisplay(""+model.getF());
      addUpListener(new UpListener(model));
      addDownListener(new DownListener(model));
      addDisplayListener(new DisplayListener(model,this));
   }
   
   public void update(Observable t, Object o) 
	// automatically called when the model is changed
   {  
      setDisplay("" + model().getF());
   }

}\end{lstlisting} 

\end{itemize}

\section{Threads} 
\begin{itemize} 
\item Thread: an instance of program execution - generalized as a Process 
\item A single Java application could be considered a Thread but a Java application may contain multiple threads 
\item The value of threads can be seen by looking at process states that are linked in a continuous loop
\item Process States: \begin{itemize} 
\item Ready: threads that are ready to run
\item Running: threads that are running on a CPU
\item Waiting: threads that are waiting for something 
\end{itemize} 
\item The operating system (or the JVM) is responsible for moving processes (threads) from state to state 
\item To instantiate a new thread: "Thread t = new Thread();"
\item t.start() will make the thread go from ready to running 
\item t.sleep(ms), t.suspend(),t.wait() will make the thread go from running to waiting
\item t.stop() will kill the thread
\item t.resume() will make the thread go from waiting to ready 
\item Making a timer work \begin{lstlisting} 
import java.awt.*;
import javax.swing.*;
public class TimerJFrame extends JFrame implements Runnable {
   private int secondsRemaining;
   private JTextArea text = new JTextArea();
   public TimerJFrame (int seconds) {
      secondsRemaining = seconds;
      setTitle("Time Remaining...");
      setSize(150,150);
      setLocation  (400,200);
      Container cp = getContentPane();
      text.setFont(new Font("Arial",2,72));
      cp.add(text);
      text.append(Integer.toString(secondsRemaining));
      setVisible(true);
      setDefaultCloseOperation(EXIT_ON_CLOSE);
      Thread timer = new Thread(this);
      timer.start();
   }
   public void run() {
  System.out.println("The game has started...");      
  while (secondsRemaining > 0) {
    try {
         Thread.sleep(1000);
         secondsRemaining--;
         text.setText(Integer.toString(secondsRemaining));
         setVisible(true);
      }
        catch (InterruptedException ie) {
           System.out.println("Timer is interrupted");
        }
   }
   JOptionPane.showMessageDialog(null,"Time is up!");
   }
   }
   \end{lstlisting} 
\item Another example \begin{lstlisting} 
public class LoggingThread extends Thread {
  private LinkedList linesToLog = new LinkedList();
  private volatile boolean terminateRequested;

  public void run() {
    try {
      while (!terminateRequested) {
        String line;
        synchronized (linesToLog) {
          while (linesToLog.isEmpty())
            linesToLog.wait();
          line = (String) linesToLog.removeFirst();
        }
        doLogLine(line);
      }
    } catch (InterruptedException ex) {
      Thread.currentThread().interrupt();
    }
  }
  private void doLogLine(String line) {
    // ... write to wherever
  }
  public void log(String line) {
    synchronized (linesToLog) {
      linesToLog.add(line);
      linesToLog.notify();
    }
  }
}
\end{lstlisting}
\end{itemize}
\end{document}